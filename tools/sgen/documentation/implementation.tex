Much of the documentation of the implementation is contained in the interface files of the source code.\\\\
TODO: make a graph illustrating the workflow.\\\\
Some notes:
\begin{itemize}
\item See \texttt{sgen/source/README.md} for a broad overview of the workflow and dependencies
\item Several of the modules make extensive use of the AST0 visitor\footnote{\texttt{coccinelle/parsing\_cocci/ast0\_visitor.ml}}. It is used because it abstracts away a lot of the boilerplace code needed for accessing the components of the abstract syntax tree.
\item An easy way to debug in \texttt{sgen/source/rule\_body.ml}: add
\begin{verbatim}
>> Snap.add "debug msg" >>
\end{verbatim}
in some function sequence. Then "debug message" will appear in the exact same place in the generated script.
\item Absolutely not optimised for performance (in particular, memory). For instance, snapshot need not be purely functional; could have used a hashtable instead of map in Snapshot.t.result, in a mutable record field. This way, there would only be one rule mapper throughout the full generation. However, using a map might be beneficial later on if we want to keep various copies for rule splitting.\\
Also, the rule mapper is converted to a string list before printing; would be more efficient to directly print the rule mapper without the middle man. However, this is easier to justify interface-wise. \\
Most importantly, the generally small size of Coccinelle scripts means that performance is not actually a problem in practice.
\item sgen needs its own flag in the Coccinelle parser: \texttt{Flag\_parsing\_cocci.generating\_mode}. This ensures that dependencies are not optimised away in the parser, as we need that information for printing the rules properly. It cannot be substituted for the \texttt{ignore\_patch\_or\_match} option, because that option also affects other parts of the parser.
\end{itemize}

%Some notes for later:
%\begin{itemize}
%\item Workflow graph (based on the workflow described in the source/README.md file
%\item Dependency graph (based on the dependencies described in the source/README.md file) HASSE DIAGRAM!
%\end{itemize}




