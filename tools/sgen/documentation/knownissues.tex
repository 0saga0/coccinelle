\begin{itemize}
\item \textbf{Missing information in rule headers}: Rule headers will not be generated correctly if the original rules contain \texttt{extends} or \texttt{expression}. Those qualifiers will be missing in the generated rule since they are not included in the output of the parser.\newline
Fix: Change the parser to include this information.
\item \textbf{Typedefs in rule headers}: If there are meta typedefs in the original rule headers, they will be included in every generated rule that uses the type. This causes an error when using the generated script since meta typedefs can only be declared once.\newline
Fix: Remove the error check in the parser since this should not cause an error.
\item Disjunction generation has a number of issues:
\begin{itemize}
  \item \textbf{Selecting wrong position in statement dots cases}: There must be the same number of positions in each disjunction case, otherwise \texttt{org} and \texttt{report} will only match when all positions can be found. In statement dots disjunctions, this is currently solved by putting the position at the first possible statement. The issue here is that, if the case contains many statements, only the first surrounding statement will be highlighted instead of the important part.\newline
Fix: Implement support for finding a single best statement in a statement dots (list of statements).
  \item \textbf{Nested disjunctions}: The position counter is frozen within a disjunction. But if there is a nested disjunction inside it, the same position will be used in both disjunction levels, causing a nonsensical script.\newline
Fix: Keep track of the current nest and name the position accordingly.
\end{itemize}
\item \textbf{No format string metavariable check}: The user can specify metavariables to be used in the messages for \texttt{org} and \texttt{report} mode. The program currently does not check if the declared metavariables actually exist in the original rule.\newline
Fix: Implement check of metavariables.
\item \textbf{No special-character rulename check}: In \texttt{Coccinelle}, allowed rulenames are limited to allowed identifiers in the \texttt{C} language, ie. certain characters are not allowed. In \texttt{sgen} when naming nameless rules, any rulename that was not already used/does not contain whitespaces will be accepted.\newline
Fix: Limit the allowed user-specified rulenames to \texttt{C} identifiers.
\item \textbf{Dependencies between patch rules}: It is possible to make patch rules that depend on other patch rules modifying the code. E.g. if one patch rule transforms f(0) and one transforms f(e), then f(0) will only match the first, since it is transformed to something else when it reaches the f(e) rule. But in \texttt{context} mode, both rules will print the f(0) occurrence.\newline
Fix: ??? Somehow detect that two rules will match the same case and insert constraints such that any match in subsequent rules do not match the first one.
\item \textbf{Dependencies in context rules}: In a \texttt{context} script, if there are dependencies between rules, they might be mixed up. This happens if there are rules dependent on the generated rules since they will then be printed before the rules on which they are dependent!\newline
Fix: ???
\item \textbf{Type and switch case disjunctions}: Currently, the program fails if attempting to generate a \texttt{Coccinelle} script with type or switch case disjunctions. The failure happens in the position generator. The reason it is not implemented is that it requires quite a lot of code for a case that seldomly appears.\newline
Fix: Implement full position generation for types and switch cases.
\end{itemize}
