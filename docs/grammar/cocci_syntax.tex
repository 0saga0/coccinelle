\documentclass{article}
%\usepackage[latin9]{inputenc}
\usepackage{hevea}
\usepackage{fullpage}
\usepackage{alltt}
\usepackage{xspace}
\usepackage[pdfborder={0 0 0}]{hyperref}
\usepackage{listings}
\usepackage[usenames,dvipsnames]{color}
\usepackage[T1]{fontenc}
\usepackage{multirow}

\lstset{basicstyle=\ttfamily,numbers=left, numberstyle=\tiny, stepnumber=1, numbersep=5pt,language=C,commentstyle=\color{OliveGreen},keywordstyle=\color{blue},stringstyle=\color{BrickRed}}

%
% You must prefix the +/- lines of
% cocci files with @+/@- respectively.
% This will enable the automatic coloration.
%
% Note: You need at least the following version of hevea
% http://hevea.inria.fr/distri/unstable/hevea-2008-12-17.tar.gz
%
\ifhevea
\lstdefinelanguage{Cocci}{
morekeywords={idexpression,expression,statement,identifier,
parameter,list,when,strict,any,forall,exists,disable,unlikely,
local,position,typedef,script,python},
keywordstyle=\color{Bittersweet}\bfseries,
sensitive=false,
moredelim=[is][\color{OliveGreen}]{@M}{@M},
moredelim=[il][\color{blue}]{@+},
moredelim=[il][\color{BrickRed}]{@-}}
\else
\lstdefinelanguage{Cocci}{
morekeywords={idexpression,expression,statement,identifier,
parameter,list,when,strict,any,forall,exists,disable,unlikely,
local,position,typedef,script,python},
keywordstyle=\color{Bittersweet}\bfseries,
sensitive=false,
moredelim=*[is][\color{OliveGreen}]{@M}{@M},
moredelim=[il][\color{blue}]{@+},
moredelim=[il][\color{BrickRed}]{@-}
}
\fi

\newif\iflanguagestyle
\languagestylefalse
% Definition of a grammar (BNF style) package for Latex and Hevea


\ifhevea
% Definition for Hevea (HTML generation)
\def\T#1{{\sf #1}}
\def\NTS#1{{\maroon #1\/}}
\def\KW#1{{\blue #1}}
\def\gramor{{\black $|$}}   
\def\grameq{{\black \quad::=\quad}}
\def\lparen{{\black (}}
\def\rparen{{\black )}}
\def\lbracket{{\black [}}
\def\rbracket{{\black ]}}
\def\plus{{\black +}}
\def\questionmark{{\black ?}}
\def\etoile{{\black *}}
\else
% Definition for LaTeX
\def\T#1{{\sf #1}}
\def\NTS#1{{\it #1\/}}
\def\KW#1{{\mtt{#1}}}
%\def\gramor{$\vert$}
\def\gramor{$\mid$}
\def\grameq{\,\,\,::=\,\,\,\,\,}
\def\lparen{(}
\def\rparen{)}
\def\lbracket{[}
\def\rbracket{]}
\def\plus{+}
\def\questionmark{?}
\def\etoile{*}
\fi

\def\NT#1{\hyperlink{#1}{\NTS{#1}}}
\def\group#1{{\rm\lparen}#1{\rm\rparen}}
\def\range#1#2{#1{..}#2}
\def\any#1{#1$^{\etoile}$}
\def\some#1{#1$^{\plus}$}
\def\ANY#1{\any{{\rm\lparen}#1{\rm\rparen}}}
\def\SOME#1{\some{{\rm\lparen}#1{\rm\rparen}}}
\def\OR{\gramor\ }

\iflanguagestyle
% Option notation : [ xxx ] versus (xxx)^?
\def\opt#1{#1$^{\questionmark}$}
\def\OPT#1{\opt{{\rm\lparen}#1{\rm\rparen}}}
\else
\def\opt#1{{\lbracket}#1{\rbracket}}
\def\OPT#1{\opt{#1}}
\fi

\newenvironment{grammar}{\begin{center}\begin{tabular}{l@{}c@{}l}}{\end{tabular}\end{center}}
\def\RULE#1\CASE#2{\hypertarget{#1}{\NTS{#1}} & \grameq & \KW{#2} \\}
\def\CASE#1{& \gramor & \KW{#1} \\}


\def\lb{\char123}
\def\rb{\char125}


\newcommand{\sizecodebis}[0]{\scriptsize}

\newcommand{\mita}[1]{\mbox{\it{{#1}}}}
\newcommand{\mtt}[1]{\mbox{\tt{{#1}}}}
\newcommand{\msf}[1]{\mbox{\sf{{#1}}}}
\newcommand{\stt}[1]{\mbox{\scriptsize\tt{{#1}}}}
\newcommand{\ssf}[1]{\mbox{\scriptsize\sf{{#1}}}}
\newcommand{\sita}[1]{\mbox{\scriptsize\it{{#1}}}}
\newcommand{\mrm}[1]{\mbox{\rm{{#1}}}}
\newcommand{\mth}[1]{\({#1}\)}
\newcommand{\entails}[2]{\begin{array}{@{}c@{}}{#1}\\\hline{#2}\end{array}}
\newcommand{\ttlb}{\mbox{\tt \char'173}}
\newcommand{\ttrb}{\mbox{\tt \char'175}}
\newcommand{\ttmid}{\mbox{\tt \char'174}}
\newcommand{\tttld}{\mbox{\tt \char'176}}

\newcommand{\fixme}[1]{{\color{red} #1}}

\ifhevea
\newcommand{\phantom}{}
\newcommand{\air}{   }
\else
\newcommand{\air}{\phantom{xxx}}
\fi

\title{The SmPL Grammar}
\author{Research group on Coccinelle}
\date{\today}

\begin{document}
\maketitle

%\section{The SmPL Grammar}

% This section presents the SmPL grammar.  This definition follows closely
% our implementation using the Menhir parser generator \cite{menhir}.

The grammar uses some rules where the left-hand side is in all capital
letters.  These are macros, which take one or more grammar rule
right-hand-sides as arguments.  The grammar also uses some unspecified
nonterminals, such as {\sf id}, {\sf const}, etc.  These refer to the
sets suggested by the name, {\em i.e.}, {\sf id} refers to the set of
possible C-language identifiers, while {\sf const} refers to the set
of possible C-language constants. \ifhevea A PDF version of this
documention is available at
\url{http://localhost:8080/coccinelle/cocci_syntax.pdf}.\else A HTML
version of this documention is online at
\url{http://localhost:8080/coccinelle/cocci_syntax.html}. \fi

\section{Program}

\begin{grammar}
  \RULE{\rt{program}}
  \CASE{\any{\NT{include\_cocci}} \some{\NT{changeset}}}

  \RULE{\rt{include\_cocci}}
  \CASE{using \NT{string}}
  \CASE{using \NT{pathToIsoFile}}

  \RULE{\rt{changeset}}
  \CASE{\NT{metavariables} \ANY{--- filename +++ filename} \NT{transformation}}

\end{grammar}

Between the metavariables and the transformation rule, there can be a
specification of constraints on the names of the old and new files,
analogous to the filename specifications in the standard patch syntax.
%(see Figure \ref{scsiglue_patch}).

\section{Metavariables}

Fresh metavariables must only be used in {\tt +} code.  Metavariables
must occur at least once in the transformation immediately following
their declaration.  These properties are not expressed in the grammar,
but are checked by a subsequent analysis.  The metavariables are
designated according to the kind of terms they can match, such as a
statement, an identifier, or an expression.  An expression
metavariable can be further constrained by its type.

\begin{grammar}
  \RULE{\rt{metavariables}}
  \CASE{@@ \any{\NT{metadecl}} @@}
  \CASE{@ \NT{rulename} @ \any{\NT{metadecl}} @@}

  \RULE{\rt{rulename}}
  \CASE{\T{id} \OPT{extends \T{id}} \OPT{depends on \NT{dep}} \opt{\NT{iso}}
    \opt{\NT{disable}} \opt{\NT{exists}} \opt{expression}}
  \CASE{script:\T{language} \OPT{depends on \NT{dep}}}

  \RULE{\rt{dep}}
  \CASE{\NT{pnrule}}
  \CASE{\NT{dep} \&\& \NT{dep}}
  \CASE{\NT{dep} || \NT{dep}}

  \RULE{\rt{pnrule}}
  \CASE{\T{id}}
  \CASE{!\T{id}}
  \CASE{ever \T{id}}
  \CASE{never \T{id}}
  \CASE{(\NT{dep})}

  \RULE{\rt{iso}}
  \CASE{using "\T{str}" \ANY{, "\T{str}"}}

  \RULE{\rt{disable}}
  \CASE{disable \T{id} \ANY{, \T{id}}}

  \RULE{\rt{exists}}
  \CASE{exists}
  \CASE{\opt{reverse} forall}
\end{grammar}

\begin{grammar}
  \RULE{\rt{metadecl}}
  \CASE{fresh identifier \NT{ids} ;}
  \CASE{parameter \opt{list} \NT{ids} ;}
  \CASE{expression list \NT{ids} ;}
  \CASE{type \NT{ids} ;}
  \CASE{statement \opt{list} \NT{ids} ;}
  \CASE{typedef \NT{ids} ;}
  \CASE{declarer name \NT{ids} ;}
  \CASE{iterator name \NT{ids} ;}
  \CASE{identifier \NT{pmid\_with\_not\_eq\_list} ;}
  \CASE{\opt{local} function \NT{pmid\_with\_not\_eq\_list} ;}
  \CASE{declarer \NT{pmid\_with\_not\_eq\_list} ;}
  \CASE{iterator \NT{pmid\_with\_not\_eq\_list} ;}
  \CASE{error \NT{pmid\_with\_not\_eq\_list} ; }
  \CASE{\opt{local} idexpression \opt{\NT{ctype}} \NT{pmid\_with\_not\_eq\_list} ;}
  \CASE{\opt{local} idexpression \OPT{\ttlb \NT{ctypes} \ttrb \any{*}} \NT{pmid\_with\_not\_eq\_list} ;}
  \CASE{\opt{local} idexpression \some{*} \NT{pmid\_with\_not\_eq\_list} ;}
  \CASE{expression \some{*} \NT{pmid\_with\_not\_eq\_list} ;}
  \CASE{\NT{ctype} [ ] \NT{pmid\_with\_not\_eq\_list} ;}
  \CASE{\ttlb \NT{ctypes} \ttrb \any{*} [ ] \NT{pmid\_with\_not\_eq\_list} ;}
  \CASE{constant \opt{\NT{ctype}} \NT{pmid\_with\_not\_eq\_list} ;}
  \CASE{constant \OPT{\ttlb \NT{ctypes} \ttrb \any{*}} \NT{pmid\_with\_not\_eq\_list} ;}
  \CASE{expression \NT{pmid\_with\_not\_ceq\_list} ;}
  \CASE{\NT{ctype} \NT{pmid\_with\_not\_ceq\_list} ;}
  \CASE{\ttlb \NT{ctypes} \ttrb \any{*} \NT{pmid\_with\_not\_ceq\_list} ;}
  \CASE{position \opt{any} \NT{pmid\_with\_not\_eq\_mid\_list} ;}
  \CASE{parameter list [ ident ] \NT{ids} ;}
  \CASE{expression list [ ident ] \NT{ids} ;}
\end{grammar}

\begin{grammar}
  \RULE{\rt{ids}}
  \CASE{\NT{pmid} \ANY{, \NT{pmid}}}

  \RULE{\rt{pmid}}
  \CASE{\T{id}}
  \CASE{\NT{mid}}
  \CASE{list}
  \CASE{error}
  \CASE{type}

  \RULE{\rt{mid}}  \CASE{\T{rulename\_id}.\T{id}}

  \RULE{\rt{pmid\_with\_not\_eq\_list}}
  \CASE{\NT{pmid\_with\_not\_eq} \ANY{, \NT{pmid\_with\_not\_eq}}}

  \RULE{\rt{pmid\_with\_not\_eq}}
  \CASE{\NT{pmid} \OPT{!= \T{id}}}
  \CASE{\NT{pmid} \OPT{!= \ttlb \T{id} \ANY{, \T{id}} \ttrb}}

  \RULE{\rt{pmid\_with\_not\_ceq\_list}}
  \CASE{\NT{pmid\_with\_not\_ceq} \ANY{, \NT{pmid\_with\_not\_ceq}}}

  \RULE{\rt{pmid\_with\_not\_ceq}}
  \CASE{\NT{pmid} \OPT{!= \NT{id\_or\_cst}}}
  \CASE{\NT{pmid} \OPT{!= \ttlb \NT{id\_or\_cst} \ANY{, \NT{id\_or\_cst}} \ttrb}}

  \RULE{\rt{id\_or\_cst}}
  \CASE{\T{id}}
  \CASE{\T{integer}}

  \RULE{\rt{pmid\_with\_not\_eq\_mid\_list}}
  \CASE{\NT{pmid\_with\_not\_eq\_mid} \ANY{, \NT{pmid\_with\_not\_eq\_mid}}}

  \RULE{\rt{pmid\_with\_not\_eq\_mid}}
  \CASE{\NT{pmid} \OPT{!= \NT{mid}}}
  \CASE{\NT{pmid} \OPT{!= \ttlb \NT{mid} \ANY{, \NT{mid}} \ttrb}}
\end{grammar}

Subsequently, we refer to arbitrary metavariables as
\mth{\msf{metaid}^{\mbox{\scriptsize{\it{ty}}}}}, where {\it{ty}} indicates
the {\it metakind} used in the declaration of the variable.  For example,
\mth{\msf{metaid}^{\ssf{Type}}} refers to a metavariable that stands for
any type.

The {\it type} nonterminal is used by both the grammar of metavariable
declarations and the grammar of transformations, and is defined on
page~\pageref{types}.

\section{Transformation}

The transformation specification essentially has the form of C code,
except that lines to remove are annotated with \verb+-+ in the first
column, and lines to add are annotated with \verb-+-.  A
transformation specification can also use {\em dots}, ``\verb-...-'',
describing an arbitrary sequence of function arguments or instructions
within a control-flow path.  Dots may be modified with a {\tt when}
clause, indicating a pattern that should not occur anywhere within the
matched sequence.  Finally, specify a disjunction of patterns, of the
form \mtt{( \mth{\mita{pat}_1} | \mita{\ldots} | \mth{\mita{pat}_n}
  )}.

The grammar of the transformation is not actually the grammar of the SmPL
code that can be written by the programmer, but the grammar of the slice of
this consisting of the {\tt -} annotated and the unannotated code (the
context of the transformed lines), or the {\tt +} annotated code and the
unannotated code.  For example, for parsing purposes, the transformation
%presented in Section \ref{sec:seq2}
is split into the two variants shown below and each is parsed
separately.

\begin{center}
\begin{tabular}{c}
\begin{lstlisting}[language=Cocci]
  proc_info_func(...) {
    <...
@--    hostno
@++    hostptr->host_no
    ...>
 }
\end{lstlisting}\\
\end{tabular}
\end{center}

{%\sizecodebis
\begin{center}
\begin{tabular}{p{5cm}p{3cm}p{5cm}}
\begin{lstlisting}[language=Cocci]
  proc_info_func(...) {
    <...
@--    hostno
    ...>
 }
\end{lstlisting}
&&
\begin{lstlisting}[language=Cocci]
  proc_info_func(...) {
    <...
@++    hostptr->host_no
    ...>
 }
\end{lstlisting}
\end{tabular}
\end{center}
}

\noindent
Requiring that both slices parse correctly ensures that the rule matches
syntactically valid C code and that it produces syntactically valid C code.
The generated parse trees are then merged for use in the subsequent
matching and transformation process.

The grammar rule for the minus or plus slice of a transformation is as follows:

\begin{grammar}

  \RULE{\rt{transformation}}
  \CASE{\any{\NT{include}} \opt{\NT{OPTDOTSEQ}\mth{(}\some{\NT{fun\_decl\_stmt}} \OR \NT{expr}, \NT{when}\mth{)}}}

  \RULE{\rt{include}}
  \CASE{\#include \T{include\_string}}

  \RULE{\rt{fun\_decl\_stmt}}
  \CASE{\NT{decl\_stmt}}
  \CASE{\NT{fundecl}}

%  \CASE{\NT{fundecl}}
%  \CASE{\NT{ctype}}
%  \CASE{\ttlb \NT{initialize\_list} \ttrb}
%  \CASE{\NT{toplevel\_seq\_start\_after\_dots\_init}}
%
%  \RULE{\rt{toplevel\_seq\_start\_after\_dots\_init}}
%  \CASE{\NT{stmt\_dots} \NT{toplevel\_after\_dots}}
%  \CASE{\NT{expr} \opt{\NT{toplevel\_after\_exp}}}
%  \CASE{\NT{decl\_stmt\_expr} \opt{\NT{toplevel\_after\_stmt}}}
%
%  \RULE{\rt{stmt\_dots}}
%  \CASE{... \any{\NT{when}}}
%  \CASE{<... \any{\NT{when}} \NT{nest\_after\_dots} ...>}
%  \CASE{<+... \any{\NT{when}} \NT{nest\_after\_dots} ...+>}

  \RULE{\rt{when}}
  \CASE{when != \NT{when\_code}}
  \CASE{when = \NT{rule\_elem\_stmt}}
  \CASE{when \NT{any\_strict} \ANY{, \NT{any\_strict}}}
  \CASE{when true != \NT{expr}}
  \CASE{when false != \NT{expr}}

  \RULE{\rt{when\_code}}
  \CASE{\NT{OPTDOTSEQ}\mth{(}\some{\NT{decl\_stmt}}, \NT{when}\mth{)}}
  \CASE{\NT{OPTDOTSEQ}\mth{(}\NT{expr}, \NT{when}\mth{)}}

  \RULE{\rt{rule\_elem\_stmt}}
  \CASE{\NT{one\_decl}}
  \CASE{\NT{expr};}
  \CASE{return \opt{\NT{expr}};}
  \CASE{break;}
  \CASE{continue;}
  \CASE{(\NT{rule\_elem\_stmt} \SOME{| \NT{rule\_elem\_stmt}})}

  \RULE{\rt{any\_strict}}
  \CASE{any}
  \CASE{strict}
  \CASE{forall}
  \CASE{exists}

%  \RULE{\rt{nest\_after\_dots}}
%  \CASE{\NT{decl\_stmt\_exp} \opt{\NT{nest\_after\_stmt}}}
%  \CASE{\opt{\NT{exp}} \opt{\NT{nest\_after\_exp}}}
%
%  \RULE{\rt{nest\_after\_stmt}}
%  \CASE{\NT{stmt\_dots} \NT{nest\_after\_dots}}
%  \CASE{\NT{decl\_stmt} \opt{\NT{nest\_after\_stmt}}}
%
%  \RULE{\rt{nest\_after\_exp}}
%  \CASE{\NT{stmt\_dots} \NT{nest\_after\_dots}}
%
%  \RULE{\rt{toplevel\_after\_dots}}
%  \CASE{\opt{\NT{toplevel\_after\_exp}}}
%  \CASE{\NT{exp} \opt{\NT{toplevel\_after\_exp}}}
%  \CASE{\NT{decl\_stmt\_expr} \NT{toplevel\_after\_stmt}}
%
%  \RULE{\rt{toplevel\_after\_exp}}
%  \CASE{\NT{stmt\_dots} \opt{\NT{toplevel\_after\_dots}}}
%
%  \RULE{\rt{decl\_stmt\_expr}}
%  \CASE{TMetaStmList$^\ddag$}
%  \CASE{\NT{decl\_var}}
%  \CASE{\NT{stmt}}
%  \CASE{(\NT{stmt\_seq} \ANY{| \NT{stmt\_seq}})}
%
%  \RULE{\rt{toplevel\_after\_stmt}}
%  \CASE{\NT{stmt\_dots} \opt{\NT{toplevel\_after\_dots}}}
%  \CASE{\NT{decl\_stmt} \NT{toplevel\_after\_stmt}}

\end{grammar}

\begin{grammar}
  \RULE{\rt{OPTDOTSEQ}\mth{(}\rt{grammar}, \rt{when\_ds}\mth{)}}
  \CASE{}\multicolumn{3}{r}{\hspace{1cm}
  \KW{\opt{... \opt{\NT{when\_ds}}} \NT{grammar}
    \ANY{... \opt{\NT{when\_ds}} \NT{grammar}}
    \opt{... \opt{\NT{when\_ds}}}}
  }

%  \CASE{\opt{... \opt{\NT{when\_ds}}} \NT{grammar}
%    \ANY{... \opt{\NT{when\_ds}} \NT{grammar}}
%    \opt{... \opt{\NT{when\_ds}}}}
%  \CASE{<... \any{\NT{when\_ds}} \NT{grammar} ...>}
%  \CASE{<+... \any{\NT{when\_ds}} \NT{grammar} ...+>}

\end{grammar}

\noindent
Lines may be annotated with an element of the set $\{\mtt{-},
\mtt{+}\}$ or an element of the set $\{\mtt{*}, \mtt{?}\}$, or one of
each. \mtt{?} and \mtt{*} represent respectively at most one, and at
least one match of the given pattern.  There are some constraints on
the use of these annotations:
\begin{itemize}
\item Dots, {\em i.e.} \texttt{...}, cannot occur on a line marked
  \texttt{+}.
\item Nested dots, {\em i.e.} dots enclosed in {\tt <} and {\tt >}, cannot
  occur on a line with any marking.
\end{itemize}

\section{Types}
\label{types}

\begin{grammar}

  \RULE{\rt{ctypes}}
  \CASE{\NT{ctype} \ANY{, \NT{ctype}}}

  \RULE{\rt{ctype}}
  \CASE{\opt{\NT{const\_vol}} \NT{generic\_ctype} \any{*}}
  \CASE{\opt{\NT{const\_vol}} void \some{*}}
  \CASE{(\NT{ctype} \ANY{| \NT{ctype}})}

  \RULE{\rt{const\_vol}}
  \CASE{const}
  \CASE{volatile}

  \RULE{\rt{generic\_ctype}}
  \CASE{\NT{ctype\_qualif}}
  \CASE{\opt{\NT{ctype\_qualif}} char}
  \CASE{\opt{\NT{ctype\_qualif}} short}
  \CASE{\opt{\NT{ctype\_qualif}} int}
  \CASE{\opt{\NT{ctype\_qualif}} long}
  \CASE{double}
  \CASE{float}
  \CASE{\OPT{struct\OR union} \T{id} \OPT{\{ \any{\NT{struct\_decl\_list}} \}}}

  \RULE{\rt{ctype\_qualif}}
  \CASE{unsigned}
  \CASE{signed}

  \RULE{\rt{struct\_decl\_list}}
  \CASE{\NT{struct\_decl\_list\_start}}

  \RULE{\rt{struct\_decl\_list\_start}}
  \CASE{\NT{struct\_decl}}
  \CASE{\NT{struct\_decl} \NT{struct\_decl\_list\_start}}
  \CASE{... \opt{when != \NT{struct\_decl}}$^\dag$ \opt{\NT{continue\_struct\_decl\_list}}}

  \RULE{\rt{continue\_struct\_decl\_list}}
  \CASE{\NT{struct\_decl} \NT{struct\_decl\_list\_start}}
  \CASE{\NT{struct\_decl}}

  \RULE{\rt{struct\_decl}}
  \CASE{\NT{ctype} \NT{d\_ident};}
  \CASE{\NT{fn\_ctype} (* \NT{d\_ident}) (\NT{PARAMSEQ}\mth{(}\NT{name\_opt\_decl}, \mth{\varepsilon)});)}
  \CASE{\opt{\NT{const\_vol}} \T{id} \NT{d\_ident};}

  \RULE{\rt{d\_ident}}
  \CASE{\NT{id} \any{[\opt{\NT{expr}}]}}

  \RULE{\rt{fn\_ctype}}
  \CASE{\NT{generic\_ctype} \any{*}}
  \CASE{void \any{*}}

  \RULE{\rt{name\_opt\_decl}}
  \CASE{\NT{decl}}
  \CASE{\NT{ctype}}
  \CASE{\NT{fn\_ctype}}
\end{grammar}

$^\dag$ The optional \texttt{when} construct ends at the end of the line.

\section{Function declarations}

\begin{grammar}

  \RULE{\rt{fundecl}}
  \CASE{\opt{\NT{fn\_ctype}} \any{\NT{funinfo}} \NT{funid}
    (\opt{\NT{PARAMSEQ}\mth{(}\NT{param},\mth{\varepsilon)}})
    \ttlb~\opt{\NT{stmt\_seq}} \ttrb}

  \RULE{\rt{funinfo}}
  \CASE{static}
  \CASE{auto}
  \CASE{register}
  \CASE{extern}
  \CASE{inline}
  \CASE{\color{red} \NT{attr}}

  \RULE{\rt{funid}}
  \CASE{\T{id}}
  \CASE{\mth{\T{metaid}^{\ssf{Func}}}}
  \CASE{\mth{\T{metaid}^{\ssf{LocalFunc}}}}

  \RULE{\rt{param}}
  \CASE{\NT{type} \T{id}}
  \CASE{\mth{\T{metaid}^{\ssf{Param}}}}
  \CASE{\mth{\T{metaid}^{\ssf{ParamList}}}}
\end{grammar}

\begin{grammar}
  \RULE{\rt{PARAMSEQ}\mth{(}\rt{grammar\_p}, \rt{when\_p}\mth{)}}
  \CASE{\mth{(}\NT{grammar\_p}\OR \ldots \opt{\NT{when\_p}}\mth{)} \ANY{, \NT{grammar\_p}\OR , \ldots \opt{\NT{when\_p}}}}
\end{grammar}

%\newpage

\section{Declarations}

\begin{grammar}
  \RULE{\rt{decl\_var}}
  \CASE{\NT{type} \opt{\NT{id} \opt{[\opt{\NT{dot\_expr}}]}
      \ANY{, \NT{id} \opt{[ \opt{\NT{dot\_expr}}]}}};}
  \CASE{\NT{one\_decl}}

  \RULE{\rt{one\_decl}}
  \CASE{\NT{type} \NT{id} \opt{[\opt{\NT{dot\_expr}}]}= \NT{nest\_expr};}

  \RULE{\rt{decl}}
  \CASE{\NT{ctype} \NT{id}}
  \CASE{\NT{fn\_ctype} (* \NT{id}) (\NT{PARAMSEQ}\mth{(}\NT{name\_opt\_decl}, \mth{\varepsilon)})}
  \CASE{void}
  \CASE{\mth{\T{metaid}^{\ssf{Param}}}}
\end{grammar}

\section{Statements}

The first rule {\em statement} describes the various forms of a statement.
The remaining rules implement the constraints that are sensitive to the
context in which the statement occurs: {\em single\_statement} for a
context in which only one statement is allowed, and {\em decl\_statement}
for a context in which a declaration, statement, or sequence thereof is
allowed.

\begin{grammar}
  \RULE{\rt{stmt}}
  \CASE{\NT{include}}
  \CASE{\mth{\T{metaid}^{\ssf{Stmt}}}}
  \CASE{\NT{expr};}
  \CASE{if (\NT{dot\_expr}) \NT{single\_stmt} \opt{else \NT{single\_stmt}}}
  \CASE{for (\opt{\NT{dot\_expr}}; \opt{\NT{dot\_expr}}; \opt{\NT{dot\_expr}})
    \NT{single\_stmt}}
  \CASE{while (\NT{dot\_expr}) \NT{single\_stmt}}
  \CASE{do \NT{single\_stmt} while (\NT{dot\_expr});}
  \CASE{\NT{iter\_ident} (\any{\NT{dot\_expr}}) \NT{single\_stmt}}
  \CASE{switch (\opt{\NT{dot\_expr}}) \ttlb \any{\NT{case\_line}} \ttrb}
  \CASE{return \opt{\NT{dot\_expr}};}
  \CASE{\ttlb~\opt{\NT{stmt\_seq}} \ttrb}
  \CASE{\NT{NEST}\mth{(}\some{\NT{decl\_stmt}}, \NT{when}\mth{)}}
  \CASE{\NT{NEST}\mth{(}\NT{expr}, \NT{when}\mth{)}}
  \CASE{break;}
  \CASE{continue;}
  \CASE{\NT{id}:}
  \CASE{goto \NT{id};}
  \CASE{\ttlb \NT{stmt\_seq} \ttrb}

  \RULE{\rt{single\_stmt}}
  \CASE{\NT{stmt}}
  \CASE{\NT{OR}\mth{(}\NT{stmt}\mth{)}}

  \RULE{\rt{decl\_stmt}}
  \CASE{\mth{\T{metaid}^{\ssf{StmtList}}}}
  \CASE{\NT{decl\_var}}
  \CASE{\NT{stmt}}
  \CASE{\NT{OR}\mth{(}\NT{stmt\_seq}\mth{)}}

  \RULE{\rt{stmt\_seq}}
  \CASE{\any{\NT{decl\_stmt}}
    \opt{\NT{DOTSEQ}\mth{(}\some{\NT{decl\_stmt}},
      \NT{when}\mth{)} \any{\NT{decl\_stmt}}}}
  \CASE{\any{\NT{decl\_stmt}}
    \opt{\NT{DOTSEQ}\mth{(}\NT{expr},
      \NT{when}\mth{)} \any{\NT{decl\_stmt}}}}

  \RULE{\rt{case\_line}}
  \CASE{default : \NT{stmt\_seq}}
  \CASE{case \NT{dot\_expr} : \NT{stmt\_seq}}

  \RULE{\rt{iter\_ident}}
  \CASE{\T{IteratorId}}
  \CASE{\T{MetaIterator}}
\end{grammar}

\begin{grammar}
  \RULE{\rt{OR}\mth{(}\rt{grammar\_o}\mth{)}}
  \CASE{( \NT{grammar\_o} \ANY{\ttmid \NT{grammar\_o}})}

  \RULE{\rt{DOTSEQ}\mth{(}\rt{grammar\_d}, \rt{when\_d}\mth{)}}
  \CASE{\ldots \opt{\NT{when\_d}} \ANY{\NT{grammar\_d} \ldots \opt{\NT{when\_d}}}}

  \RULE{\rt{NEST}\mth{(}\rt{grammar\_n}, \rt{when\_n}\mth{)}}
  \CASE{<\ldots \NT{grammar\_n} \ANY{\ldots \opt{\NT{when\_n}} \NT{grammar\_n}} \ldots>}
  \CASE{<+\ldots \NT{grammar\_n} \ANY{\ldots \opt{\NT{when\_n}} \NT{grammar\_n}} \ldots+>}
\end{grammar}

\noindent
OR is a macro that generates a disjunction of patterns.  The three
tokens \T{(}, \T{\ttmid}, and \T{)} must appear in the leftmost
column, to differentiate them from the parentheses and bit-or tokens
that can appear within expressions (and cannot appear in the leftmost
column).  These tokens are furthermore different from (, \(\mid\), and
), which are part of the grammar metalanguage.

\section{Expressions}

A nest or a single ellipsis is allowed in some expression contexts, and
causes ambiguity in others.  For example, in a sequence \mtt{\ldots
\mita{expr} \ldots}, the nonterminal \mita{expr} must be instantiated as an
explicit C-language expression, while in an array reference,
\mtt{\mth{\mita{expr}_1} \mtt{[} \mth{\mita{expr}_2} \mtt{]}}, the
nonterminal \mth{\mita{expr}_2}, because it is delimited by brackets, can
be also instantiated as \mtt{\ldots}, representing an arbitrary expression.  To
distinguish between the various possibilities, we define three nonterminals
for expressions: {\em expr} does not allow either top-level nests or
ellipses, {\em nest\_expr} allows a nest but not an ellipsis, and {\em
dot\_expr} allows both.  The EXPR macro is used to express these variants
in a concise way.

\begin{grammar}
  \RULE{\rt{expr}}
  \CASE{\NT{EXPR}\mth{(}\NT{expr}\mth{)}}

  \RULE{\rt{nest\_expr}}
  \CASE{\NT{EXPR}\mth{(}\NT{nest\_expr}\mth{)}}
  \CASE{\NT{NEST}\mth{(}\NT{nest\_expr}, \NT{exp\_whencode}\mth{)}}

  \RULE{\rt{dot\_expr}}
  \CASE{\NT{EXPR}\mth{(}\NT{dot\_expr}\mth{)}}
  \CASE{\NT{NEST}\mth{(}\NT{dot\_expr}, \NT{exp\_whencode}\mth{)}}
  \CASE{... \opt{\NT{exp\_whencode}}}

  \RULE{\rt{EXPR}\mth{(}\rt{exp}\mth{)}}
  \CASE{\NT{exp} \NT{assign\_op} \NT{exp}}
  \CASE{\NT{exp}++}
  \CASE{\NT{exp}--}
  \CASE{\NT{unary\_op} \NT{exp}}
  \CASE{\NT{exp} \NT{bin\_op} \NT{exp}}
  \CASE{\NT{exp} ? \NT{dot\_expr} : \NT{exp}}
  \CASE{(\NT{type}) \NT{exp}}
  \CASE{\NT{exp} [\NT{dot\_expr}]}
  \CASE{\NT{exp} . \NT{id}}
  \CASE{\NT{exp} -> \NT{id}}
  \CASE{\NT{exp}(\opt{\NT{PARAMSEQ}\mth{(}\NT{arg}, \NT{exp\_whencode}\mth{)}})}
  \CASE{\NT{id}}
  \CASE{\mth{\T{metaid}^{\ssf{Func}}}}
  \CASE{\mth{\T{metaid}^{\ssf{LocalFunc}}}}
  \CASE{\mth{\T{metaid}^{\ssf{Exp}}}}
  \CASE{\mth{\T{metaid}^{\ssf{Err}}}}
  \CASE{\mth{\T{metaid}^{\ssf{Const}}}}
  \CASE{\NT{const}}
  \CASE{(\NT{dot\_expr})}
  \CASE{\NT{OR}\mth{(}\NT{exp}\mth{)}}

  \RULE{\rt{arg}}
  \CASE{\NT{nest\_expr}}
  \CASE{\mth{\T{metaid}^{\ssf{ExpList}}}}

  \RULE{\rt{exp\_whencode}}
  \CASE{when != \NT{dot\_expr}}

  \RULE{\rt{assign\_op}}
  \CASE{= \OR -= \OR += \OR *= \OR /= \OR \%=}
  \CASE{\&= \OR |= \OR \caret= \OR \lt\lt= \OR \gt\gt=}

  \RULE{\rt{bin\_op}}
  \CASE{* \OR / \OR \% \OR + \OR -}
  \CASE{\lt\lt \OR \gt\gt \OR \caret\xspace \OR \& \OR \ttmid}
  \CASE{< \OR > \OR <= \OR >= \OR == \OR != \OR \&\& \OR \ttmid\ttmid}

  \RULE{\rt{unary\_op}}
  \CASE{++ \OR -- \OR \& \OR * \OR + \OR - \OR !}

\end{grammar}

\section{Constant, Identifiers and Types for Transformations}

\begin{grammar}
  \RULE{\rt{const}}
  \CASE{\NT{string}}
  \CASE{[0-9]+}
  \CASE{\mth{\cdots}}

  \RULE{\rt{string}}
  \CASE{"\any{[\^{}"]}"}

  \RULE{\rt{id}}
  \CASE{\T{id} \OR \mth{\T{metaid}^{\ssf{Id}}}}

  \RULE{\rt{type}}
  \CASE{\NT{ctype} \OR \mth{\T{metaid}^{\ssf{Type}}}}

  \RULE{\rt{pathToIsoFile}}
  \CASE{<.*>}
\end{grammar}


\section{Examples}
%\label{sec:examples}

This section presents a range of examples.  Each
example is presented along with some C code to which it is
applied. The description explains the rules and the matching process.

\subsection{Function renaming}

One of the primary goals of Coccinelle is to perform software
evolution.  For instance, Coccinelle could be used to perform function
renaming. In the following example, every occurrence of a call to the
function \texttt{foo} is replaced by a call to the
function \texttt{bar}.\\

\begin{tabular}{ccc}
Before & Semantic patch & After \\
\begin{minipage}[t]{.3\linewidth}
\begin{lstlisting}
#DEFINE TEST "foo"

printf("foo");

int main(int i) {
//Test
  int k = foo();

  if(1) {
    foo();
  } else {
    foo();
  }

  foo();
}
\end{lstlisting}
\end{minipage}
&
\begin{minipage}[t]{.3\linewidth}
\begin{lstlisting}[language=Cocci]
@M@@

@@@M


@-- foo()
@++ bar()
\end{lstlisting}
\end{minipage}
&
\begin{minipage}[t]{.3\linewidth}
\begin{lstlisting}
#DEFINE TEST "foo"

printf("foo");

int main(int i) {
//Test
  int k = bar();

  if(1) {
    bar();
  } else {
    bar();
  }

  bar();
}
\end{lstlisting}
\end{minipage}\\
\end{tabular}

\newpage
\subsection{Removing a function argument}

Another important kind of evolution is the introduction or deletion of a
function argument. In the following example, the rule \texttt{rule1} looks
for definitions of functions having return type \texttt{irqreturn\_t} and
two parameters. A second \emph{anonymous} rule then looks for calls to the
previously matched functions that have three arguments. The third argument
is then removed to correspond to the new function prototype.\\

\begin{tabular}{c}
\begin{lstlisting}[language=Cocci,name=arg]
@M@ rule1 @
identifier fn;
identifier irq, dev_id;
typedef irqreturn_t;
@@@M

static irqreturn_t fn (int irq, void *dev_id)
{
   ...
}

@M@@
identifier rule1.fn;
expression E1, E2, E3;
@@@M

 fn(E1, E2
@--  ,E3
   )
\end{lstlisting}\\
\end{tabular}

\vspace{1cm}

\begin{tabular}{c}
  \texttt{drivers/atm/firestream.c} at line 1653 before transformation\\
\begin{lstlisting}[language=PatchC]
static void fs_poll (unsigned long data)
{
        struct fs_dev *dev = (struct fs_dev *) data;

@-        fs_irq (0, dev, NULL);
        dev->timer.expires = jiffies + FS_POLL_FREQ;
        add_timer (&dev->timer);
}
\end{lstlisting}\\
\vspace{1cm}
\\


  \texttt{drivers/atm/firestream.c} at line 1653 after transformation\\
\begin{lstlisting}[language=PatchC]
static void fs_poll (unsigned long data)
{
        struct fs_dev *dev = (struct fs_dev *) data;

@+        fs_irq (0, dev);
        dev->timer.expires = jiffies + FS_POLL_FREQ;
        add_timer (&dev->timer);
}
\end{lstlisting}\\
\end{tabular}

\newpage
\subsection{Introduction of a macro}

To avoid code duplication or error prone code, the kernel provides
macros such as \texttt{BUG\_ON}, \texttt{DIV\_ROUND\_UP} and
\texttt{FIELD\_SIZE}. In these cases, the semantic patches look for
the old code pattern and replace it by the new code.\\

A semantic patch to introduce uses of the \texttt{DIV\_ROUND\_UP} macro
looks for the corresponding expression, \emph{i.e.}, $(n + d - 1) /
d$. When some code is matched, the metavariables \texttt{n} and \texttt{d}
are bound to their corresponding expressions. Finally, Coccinelle rewrites
the code with the \texttt{DIV\_ROUND\_UP} macro using the values bound to
\texttt{n} and \texttt{d}, as illustrated in the patch that follows.\\

\begin{tabular}{c}
Semantic patch to introduce uses of the \texttt{DIV\_ROUND\_UP} macro\\
\begin{lstlisting}[language=Cocci,name=divround]
@M@ haskernel @
@@@M

#include <linux/kernel.h>

@M@ depends on haskernel @
expression n,d;
@@@M

(
@-- (((n) + (d)) - 1) / (d))
@++ DIV_ROUND_UP(n,d)
|
@-- (((n) + ((d) - 1)) / (d))
@++ DIV_ROUND_UP(n,d)
)
\end{lstlisting}
\end{tabular}\\

\vspace{1cm}

\begin{tabular}{c}
Example of a generated patch hunk\\
\begin{lstlisting}[language=PatchC]
@---- a/drivers/atm/horizon.c
@++++ b/drivers/atm/horizon.c
@M@@ -698,7 +698,7 @@ got_it:
                if (bits)
                        *bits = (div<<CLOCK_SELECT_SHIFT) | (pre-1);
                if (actual) {
@--                       *actual = (br + (pre<<div) - 1) / (pre<<div);
@++                       *actual = DIV_ROUND_UP(br, pre<<div);
                        PRINTD (DBG_QOS, "actual rate: %u", *actual);
                }
                return 0;
\end{lstlisting}
\end{tabular}\\

\newpage

The \texttt{BUG\_ON} macro makes a assertion about the value of an
expression. However, because some parts of the kernel define
\texttt{BUG\_ON} to be the empty statement when debugging is not wanted,
care must be taken when the asserted expression may have some side-effects,
as is the case of a function call. Thus, we create a rule introducing
\texttt{BUG\_ON} only in the case when the asserted expression does not
perform a function call.

On particular piece of code that has the form of a function call is a use
of \texttt{unlikely}, which informs the compiler that a particular
expression is unlikely to be true.  In this case, because \texttt{unlikely}
does not perform any side effects, it is safe to use \texttt{BUG\_ON}.  The
second rule takes care of this case.  It furthermore disables the
isomorphism that allows a call to \texttt{unlikely} be replaced with its
argument, as then the second rule would be the same as the first one.\\

\begin{tabular}{c}
\begin{lstlisting}[language=Cocci,name=bugon]
@M@@
expression E,f;
@@@M

(
  if (<+... f(...) ...+>) { BUG(); }
|
@-- if (E) { BUG(); }
@++ BUG_ON(E);
)

@M@ disable unlikely @
expression E,f;
@@@M

(
  if (<+... f(...) ...+>) { BUG(); }
|
@-- if (unlikely(E)) { BUG(); }
@++ BUG_ON(E);
)
\end{lstlisting}\\
\end{tabular}\\

For instance, using the semantic patch above, Coccinelle generates
patches like the following one.

\begin{tabular}{c}
\begin{lstlisting}[language=PatchC]
@---- a/fs/ext3/balloc.c
@++++ b/fs/ext3/balloc.c
@M@@ -232,8 +232,7 @@ restart:
                prev = rsv;
        }
        printk("Window map complete.\n");
@--       if (bad)
@--               BUG();
@++       BUG_ON(bad);
 }
 #define rsv_window_dump(root, verbose) \
        __rsv_window_dump((root), (verbose), __FUNCTION__)
\end{lstlisting}
\end{tabular}

\newpage
\subsection{Look for \texttt{NULL} dereference}

This SmPL match looks for \texttt{NULL} dereferences. Once an
expression has been compared to \texttt{NULL}, a dereference to this
expression is prohibited unless the pointer variable is reassigned.\\

\begin{tabular}{c}
    Original \\

\begin{lstlisting}
foo = kmalloc(1024);
if (!foo) {
  printk ("Error %s", foo->here);
  return;
}
foo->ok = 1;
return;
\end{lstlisting}\\
  \end{tabular}

\vspace{1cm}

\begin{tabular}{c}
  Semantic match\\

\begin{lstlisting}[language=Cocci]
@M@@
expression E, E1;
identifier f;
statement S1,S2,S3;
@@@M

@+* if (E == NULL)
{
  ... when != if (E == NULL) S1 else S2
      when != E = E1
@+* E->f
  ... when any
  return ...;
}
else S3
\end{lstlisting}\\
\end{tabular}

\vspace{1cm}

\begin{tabular}{c}
  Matched lines\\

\begin{lstlisting}[language=PatchC]
foo = kmalloc(1024);
@-if (!foo) {
@-  printk ("Error %s", foo->here);
  return;
}
foo->ok = 1;
return;
\end{lstlisting}\\
\end{tabular}

\newpage
\subsection{Reference counter: the of\_xxx API}

Coccinelle can embed Python code. Python code is used inside special
SmPL rule annotated with \texttt{script:python}.  Python rules inherit
metavariables, such as identifier or token positions, from other SmPL
rules. The inherited metavariables can then be manipulated by Python
code.

The following semantic match looks for a call to the
\texttt{of\_find\_node\_by\_name} function. This call increments a
counter which must be decremented to release the resource. Then, when
there is no call to \texttt{of\_node\_put}, no new assignment to the
\texttt{device\_node} variable \texttt{n} and a \texttt{return}
statement is reached, a bug is detected and the position \texttt{p1}
and \texttt{p2} are initialized. As the Python only depends on the
positions \texttt{p1} and \texttt{p2}, it is evaluated. In the
following case, some emacs Org mode data are produced.  This example
illustrates the various fields that can be accessed in the Python code from
a position variable.

\begin{tabular}{c}
\begin{lstlisting}[language=Cocci,breaklines=true]
@M@ r exists @
local idexpression struct device_node *n;
position p1, p2;
statement S1,S2;
expression E,E1;
@@@M

(
if (!(n@p1 = of_find_node_by_name(...))) S1
|
n@p1 = of_find_node_by_name(...)
)
<... when != of_node_put(n)
    when != if (...) { <+... of_node_put(n) ...+> }
    when != true !n  || ...
    when != n = E
    when != E = n
if (!n || ...) S2
...>
(
  return <+...n...+>;
|
return@p2 ...;
|
n = E1
|
E1 = n
)

@M@ script:python @
p1 << r.p1;
p2 << r.p2;
@@@M

print "* TODO [[view:%s::face=ovl-face1::linb=%s::colb=%s::cole=%s][inc. counter:%s::%s]]" % (p1[0].file,p1[0].line,p1[0].column,p1[0].column_end,p1[0].file,p1[0].line)
print "[[view:%s::face=ovl-face2::linb=%s::colb=%s::cole=%s][return]]" % (p2[0].file,p2[0].line,p2[0].column,p2[0].column_end)
\end{lstlisting}
\end{tabular}


\newpage

Lines 13 to 17 list a variety of constructs that should not appear
between a call to \texttt{of\_find\_node\_by\_name} and a buggy return
site. Examples are a call to \texttt{of\_node\_put} (line 13) and a
transition into the then branch of a conditional testing whether
\texttt{n} is \texttt{NULL} (line 15). Any number of conditionals
testing whether \texttt{n} is \texttt{NULL} are allowed as indicated
by the use of a nest \texttt{<...~~...>} to describe the path between
the call to \texttt{of\_find\_node\_by\_name}, the return and the
conditional in the pattern on line 18.\\

The previously semantic match has been used to generate the following
lines. They may be edited using the emacs Org mode to navigate in the code
from a site to another.

\begin{lstlisting}[language=,breaklines=true]
* TODO [[view:/linux-next/arch/powerpc/platforms/pseries/setup.c::face=ovl-face1::linb=236::colb=18::cole=20][inc. counter:/linux-next/arch/powerpc/platforms/pseries/setup.c::236]]
[[view:/linux-next/arch/powerpc/platforms/pseries/setup.c::face=ovl-face2::linb=250::colb=3::cole=9][return]]
* TODO [[view:/linux-next/arch/powerpc/platforms/pseries/setup.c::face=ovl-face1::linb=236::colb=18::cole=20][inc. counter:/linux-next/arch/powerpc/platforms/pseries/setup.c::236]]
[[view:/linux-next/arch/powerpc/platforms/pseries/setup.c::face=ovl-face2::linb=245::colb=3::cole=9][return]]
\end{lstlisting}

Note~: Coccinelle provides some predefined Python functions,
\emph{i.e.}, \texttt{cocci.print\_main}, \texttt{cocci.print\_sec} and
\texttt{cocci.print\_secs}. One could alternatively write the following
SmPL rule instead of the previously presented one.

\begin{tabular}{c}
\begin{lstlisting}[language=Cocci]
@M@ script:python @
p1 << r.p1;
p2 << r.p2;
@@@M

cocci.print_main(p1)
cocci.print_sec("return",p2)
\end{lstlisting}
\end{tabular}\\

The function \texttt{cocci.print\_secs} is used when there are several
positions which are matched by a single position variable and that
every matched position should be printed.

Any metavariable could be inherited in the Python code. However,
accessible fields are not currently equally supported among them.

% \begin{tabular}{ccc}
% Before & Semantic patch & After \\
% \begin{minipage}[t]{.3\linewidth}
% \begin{lstlisting}
% \end{lstlisting}
% \end{minipage}
% &
% \begin{minipage}[t]{.3\linewidth}
% \begin{lstlisting}[language=Cocci]
% \end{lstlisting}
% \end{minipage}
% &
% \begin{minipage}[t]{.3\linewidth}
% \begin{lstlisting}
% \end{lstlisting}
% \end{minipage}\\
% \end{tabular}

%%% Local Variables:
%%% mode: LaTeX
%%% TeX-master: "cocci_syntax"
%%% coding: latin-9
%%% TeX-PDF-mode: t
%%% ispell-local-dictionary: "american"
%%% End:

\end{document}

%%% Local Variables:
%%% mode: LaTeX
%%% TeX-master: "cocci_syntax"
%%% coding: latin-9
%%% TeX-PDF-mode: t
%%% ispell-local-dictionary: "english"
%%% End:
