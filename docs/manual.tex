\title{spatch user manual}


\chapter{Tutorial}

\section{SmPL piece by pieces}





\chapter{spatch command line options}





\chapter{Advanced features}

\subsection{Position}
%src: a mail from julia

Tu peux maintenant mettre des positions partout.  C'est interdit d'heriter
une position a travers une regle qui fait une modif.

Tu peux declarer une position avec des contraintes, eg

position p1 != {x.p2, y.p3};

Entre les premiers @@ d'une regle, tu peux mettre "expression" et avoir
une meilleure parsing.  Ca permet par exemple de faire:

<... f() ...> + <... g() ...>

pour decrire un + qui a f() et g() quelquepart comme arguements.

Pour les }, j'ai mis un champ "is_fake" dans les noeuds, avec is_loop,
etc.

\subsection{Embeded Python scripting}




\chapter{Developing a semantic patch}

\subsection{emacs mode}

\subsection{process}

\subsection{Linux workflow}

