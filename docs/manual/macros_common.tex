
% Very convenient to add comments on the paper. Just set the boolean
% to false before sending the paper:
\newboolean{showcomments}
\setboolean{showcomments}{true}
\ifthenelse{\boolean{showcomments}}
{ \newcommand{\mynote}[2]{
    \fbox{\bfseries\sffamily\scriptsize#1}
    {\small$\blacktriangleright$\textsf{\emph{#2}}$\blacktriangleleft$}}}
{ \newcommand{\mynote}[2]{}}

\newcommand\jl[1]{\mynote{Julia}{#1}}



\newcommand{\sizecodebis}[0]{\scriptsize}

\newcommand{\mita}[1]{\mbox{\it{{#1}}}}
\newcommand{\mtt}[1]{\mbox{\tt{{#1}}}}
\newcommand{\msf}[1]{\mbox{\sf{{#1}}}}
\newcommand{\stt}[1]{\mbox{\scriptsize\tt{{#1}}}}
\newcommand{\ssf}[1]{\mbox{\scriptsize\sf{{#1}}}}
\newcommand{\sita}[1]{\mbox{\scriptsize\it{{#1}}}}
\newcommand{\mrm}[1]{\mbox{\rm{{#1}}}}
\newcommand{\mth}[1]{\({#1}\)}
\newcommand{\entails}[2]{\begin{array}{@{}c@{}}{#1}\\\hline{#2}\end{array}}
\newcommand{\ttlb}{\mbox{\tt \char'173}}
\newcommand{\ttrb}{\mbox{\tt \char'175}}
\newcommand{\ttmid}{\mbox{\tt \char'174}}
\newcommand{\tttld}{\mbox{\tt \char'176}}

\newcommand{\fixme}[1]{{\color{red} #1}}


%------------------------------------------------------------------------------
%%%Squeeze space in bibliographybg
%  \let\oldthebibliography=\thebibliography
%  \let\endoldthebibliography=\endthebibliography
%  \renewenvironment{thebibliography}[1]{%
%    \begin{oldthebibliography}{#1}%
%      \setlength{\parskip}{0ex}%
%      \setlength{\itemsep}{0ex}%
%  }%
%  {%
%    \end{oldthebibliography}%
%  }


%------------------------------------------------------------------------------

%christian lindig tricks: 
% http://www.st.cs.uni-saarland.de/~lindig/tex.html

%\newif\ifdraft\drafttrue

%see overfull 
%\ifdraft
%\overfullrule3pt
%\fi



%\def\<#1>{\texttt{#1}}
%Now I can write \<some code here>, which feels more natural 
% in the editor than \texttt{some code here}.
