
%\section{The SmPL Grammar}

% This section presents the SmPL grammar.  This definition follows closely
% our implementation using the Menhir parser generator \cite{menhir}.

This document presents the grammar of the SmPL language used by the
\href{http://coccinelle.lip6.fr/}{Coccinelle tool}.  For the most
part, the grammar is written using standard notation.  In some rules,
however, the left-hand side is in all uppercase letters.  These are
macros, which take one or more grammar rule right-hand-sides as
arguments.  The grammar also uses some unspecified nonterminals, such
as \T{id}, \T{const}, etc.  These refer to the sets suggested by
the name, {\em i.e.}, \T{id} refers to the set of possible
C-language identifiers, while \T{const} refers to the set of
possible C-language constants.
%
\ifhevea
A PDF version of this documentation is available at
\url{http://coccinelle.lip6.fr/docs/main_grammar.pdf}.
\else
A HTML version of this documentation is available online at
\url{http://coccinelle.lip6.fr/docs/main_grammar.html}.
\fi

\section{Program}

\begin{grammar}
  \RULE{\rt{program}}
  \CASE{\any{\NT{include\_cocci}} \some{\NT{changeset}}}

  \RULE{\rt{include\_cocci}}
  \CASE{using \NT{string}}
  \CASE{using \NT{pathToIsoFile}}
  \CASE{virtual \T{id} \ANY{, \T{id}}}

  \RULE{\rt{changeset}}
  \CASE{\NT{metavariables} \NT{transformation}}
  \CASE{\NT{script\_metavariables} \T{script\_code}}
%  \CASE{\NT{metavariables} \ANY{--- filename +++ filename} \NT{transformation}}
\end{grammar}

\noindent
\T{script\_code} is any code in the chosen scripting language.  Parsing of
the semantic patch does not check the validity of this code; any errors are
first detected when the code is executed.  Furthermore, \texttt{@} should
not be use in this code.  Spatch scans the script code for the next
\texttt{@} and considers that to be the beginning of the next rule, even if
\texttt{@} occurs within e.g., a string or a comment.

\texttt{virtual} keyword is used to declare virtual rules. Virtual
rules may be subsequently used as a dependency for the rules in the
SmPL file. Whether a virtual rule is defined or not is controlled by
the \texttt{-D} option on the command line.

% Between the metavariables and the transformation rule, there can be a
% specification of constraints on the names of the old and new files,
% analogous to the filename specifications in the standard patch syntax.
% (see Figure \ref{scsiglue_patch}).

\section{Metavariables for transformations}

The \NT{rulename} portion of the metavariable declaration can specify
properties of a rule such as its name, the names of the rules that it
depends on, the isomorphisms to be used in processing the rule, and whether
quantification over paths should be universal or existential.  The optional
annotation {\tt expression} indicates that the pattern is to be considered
as matching an expression, and thus can be used to avoid some parsing
problems.

The \NT{metadecl} portion of the metavariable declaration defines various
types of metavariables that will be used for matching in the transformation
section.

\begin{grammar}
  \RULE{\rt{metavariables}}
  \CASE{@@ \any{\NT{metadecl}} @@}
  \CASE{@ \NT{rulename} @ \any{\NT{metadecl}} @@}

  \RULE{\rt{rulename}}
  \CASE{\T{id} \OPT{extends \T{id}} \OPT{depends on \NT{dep}} \opt{\NT{iso}}
    \opt{\NT{disable-iso}} \opt{\NT{exists}} \opt{expression}}

  \RULE{\rt{dep}}
  \CASE{\NT{pnrule}}
  \CASE{\NT{dep} \&\& \NT{dep}}
  \CASE{\NT{dep} || \NT{dep}}

  \RULE{\rt{pnrule}}
  \CASE{\T{id}}
  \CASE{!\T{id}}
  \CASE{ever \T{id}}
  \CASE{never \T{id}}
  \CASE{(\NT{dep})}

  \RULE{\rt{iso}}
  \CASE{using \NT{string} \ANY{, \NT{string}}}

  \RULE{\rt{disable-iso}}
  \CASE{disable \NT{COMMA\_LIST}\mth{(}\T{id}\mth{)}}

  \RULE{\rt{exists}}
  \CASE{exists}
  \CASE{forall}
%  \CASE{\opt{reverse} forall}

  \RULE{\rt{COMMA\_LIST}\mth{(}\rt{elem}\mth{)}}
  \CASE{\NT{elem} \ANY{, \NT{elem}}}
\end{grammar}

The keyword \KW{disable} is normally used with the names of
isomorphisms defined in standard.iso or whatever isomorphism file has been
included.  There are, however, some other isomorphisms that are built into
the implementation of Coccinelle and that can be disabled as well.  Their
names are given below.  In each case, the text describes the standard
behavior.  Using \NT{disable-iso} with the given name disables this behavior.

\begin{itemize}
\item \KW{optional\_storage}: A SmPL function definition that does not
  specify any visibility (i.e., static or extern), or a SmPL variable
  declaration that does not specify any storage (i.e., auto, static,
  register, or extern), matches a function declaration or variable
  declaration with any visibility or storage, respectively.
\item \KW{optional\_qualifier}: This is similar to \KW{optional\_storage},
  except that here is it the qualifier (i.e., const or volatile) that does
  not have to be specified in the SmPL code, but may be present in the C code.
\item \KW{value\_format}: Integers in various formats, e.g., 1 and 0x1, are
  considered to be equivalent in the matching process.
\item \KW{comm\_assoc}: An expression of the form \NT{exp} \NT{bin\_op}
  \KW{...}, where \NT{bin\_op} is commutative and associative, is
  considered to match any top-level sequence of \NT{bin\_op} operators
  containing \NT{exp} as the top-level argument.
\end{itemize}

The possible types of metavariable declarations are defined by the grammar
rule below.  Metavariables should occur at least once in the transformation
immediately following their declaration.  Fresh metavariables must only be
used in {\tt +} code.  These properties are not expressed in the grammar,
but are checked by a subsequent analysis.  The metavariables are designated
according to the kind of terms they can match, such as a statement, an
identifier, or an expression.  An expression metavariable can be further
constrained by its type.  A declaration metavariable matches the
declaration of one or more variables, all sharing the same type
specification ({\em e.g.}, {\tt int a,b,c=3;}).  A field metavariable does
the same, but for structure fields.

\begin{grammar}
  \RULE{\rt{metadecl}}
  \CASE{fresh identifier \NT{ids} ;}
  \CASE{identifier \NT{COMMA\_LIST}\mth{(}\NT{pmid\_with\_regexp}\mth{)} ;}
  \CASE{identifier \NT{COMMA\_LIST}\mth{(}\NT{pmid\_with\_virt\_or\_not\_eq}\mth{)} ;}
  \CASE{parameter \opt{list} \NT{ids} ;}
  \CASE{parameter list [ \NT{id} ] \NT{ids} ;}
  \CASE{parameter list [ \NT{const} ] \NT{ids} ;}
  \CASE{type \NT{ids} ;}
  \CASE{statement \opt{list} \NT{ids} ;}
  \CASE{declaration \opt{list} \NT{ids} ;}
  \CASE{field \opt{list} \NT{ids} ;}
  \CASE{typedef \NT{ids} ;}
  \CASE{declarer name \NT{ids} ;}
%  \CASE{\opt{local} function \NT{pmid\_with\_not\_eq\_list} ;}
  \CASE{declarer \NT{COMMA\_LIST}\mth{(}\NT{pmid\_with\_regexp}\mth{)} ;}
  \CASE{declarer \NT{COMMA\_LIST}\mth{(}\NT{pmid\_with\_not\_eq}\mth{)} ;}
  \CASE{iterator name \NT{ids} ;}
  \CASE{iterator \NT{COMMA\_LIST}\mth{(}\NT{pmid\_with\_regexp}\mth{)} ;}
  \CASE{iterator \NT{COMMA\_LIST}\mth{(}\NT{pmid\_with\_not\_eq}\mth{)} ;}
%  \CASE{error \NT{pmid\_with\_not\_eq\_list} ; }
  \CASE{\opt{local} idexpression \opt{\NT{ctype}} \NT{COMMA\_LIST}\mth{(}\NT{pmid\_with\_not\_eq}\mth{)} ;}
  \CASE{\opt{local} idexpression \OPT{\ttlb \NT{ctypes}\ttrb~\any{*}} \NT{COMMA\_LIST}\mth{(}\NT{pmid\_with\_not\_eq}\mth{)} ;}
  \CASE{\opt{local} idexpression \some{*} \NT{COMMA\_LIST}\mth{(}\NT{pmid\_with\_not\_eq}\mth{)} ;}
  \CASE{expression list \NT{ids} ;}
  \CASE{expression \some{*} \NT{COMMA\_LIST}\mth{(}\NT{pmid\_with\_not\_eq}\mth{)} ;}
  \CASE{expression enum \any{*} \NT{COMMA\_LIST}\mth{(}\NT{pmid\_with\_not\_eq}\mth{)} ;}
  \CASE{expression struct \any{*} \NT{COMMA\_LIST}\mth{(}\NT{pmid\_with\_not\_eq}\mth{)} ;}
  \CASE{expression union \any{*} \NT{COMMA\_LIST}\mth{(}\NT{pmid\_with\_not\_eq}\mth{)} ;}
  \CASE{expression \NT{COMMA\_LIST}\mth{(}\NT{pmid\_with\_not\_ceq}\mth{)} ;}
  \CASE{expression list [ \NT{id} ] \NT{ids} ;}
  \CASE{expression list [ \NT{const} ] \NT{ids} ;}
  \CASE{\NT{ctype} [ ] \NT{COMMA\_LIST}\mth{(}\NT{pmid\_with\_not\_eq}\mth{)} ;}
  \CASE{\NT{ctype} \NT{COMMA\_LIST}\mth{(}\NT{pmid\_with\_not\_ceq}\mth{)} ;}
  \CASE{\ttlb \NT{ctypes}\ttrb~\any{*} \NT{COMMA\_LIST}\mth{(}\NT{pmid\_with\_not\_ceq}\mth{)} ;}
  \CASE{\ttlb \NT{ctypes}\ttrb~\any{*} [ ] \NT{COMMA\_LIST}\mth{(}\NT{pmid\_with\_not\_eq}\mth{)} ;}
  \CASE{constant \opt{\NT{ctype}} \NT{COMMA\_LIST}\mth{(}\NT{pmid\_with\_not\_eq}\mth{)} ;}
  \CASE{constant \OPT{\ttlb \NT{ctypes}\ttrb~\any{*}} \NT{COMMA\_LIST}\mth{(}\NT{pmid\_with\_not\_eq}\mth{)} ;}
  \CASE{position \opt{any} \NT{COMMA\_LIST}\mth{(}\NT{pmid\_with\_not\_eq\_mid}\mth{)} ;}
\end{grammar}

It is possible to specify that an expression list or a parameter list
metavariable should match a specific number of expressions or parameters.

\begin{grammar}
  \RULE{\rt{ids}}
  \CASE{\NT{COMMA\_LIST}\mth{(}\NT{pmid}\mth{)}}

  \RULE{\rt{pmid}}
  \CASE{\T{id}}
  \CASE{\NT{mid}}
%   \CASE{list}
%   \CASE{error}
%   \CASE{type}

  \RULE{\rt{mid}}  \CASE{\T{rulename\_id}.\T{id}}

  \RULE{\rt{pmid\_with\_regexp}}
  \CASE{\NT{pmid} \~{}= \NT{regexp}}

  \RULE{\rt{pmid\_with\_not\_eq}}
  \CASE{\NT{pmid} \OPT{!= \NT{id\_or\_meta}}}
  \CASE{\NT{pmid}
     \OPT{!= \ttlb~\NT{COMMA\_LIST}\mth{(}\NT{id\_or\_meta}\mth{)} \ttrb}}

  \RULE{\rt{pmid\_with\_virt\_or\_not\_eq}}
  \CASE{virtual.\T{id}}
  \CASE{\NT{pmid\_with\_not\_eq}}

  \RULE{\rt{pmid\_with\_not\_ceq}}
  \CASE{\NT{pmid} \OPT{!= \NT{id\_or\_cst}}}
  \CASE{\NT{pmid} \OPT{!= \ttlb~\NT{COMMA\_LIST}\mth{(}\NT{id\_or\_cst}\mth{)} \ttrb}}

  \RULE{\rt{id\_or\_cst}}
  \CASE{\T{id}}
  \CASE{\T{integer}}

  \RULE{\rt{id\_or\_meta}}
  \CASE{\T{id}}
  \CASE{\T{rulename\_id}.\T{id}}

  \RULE{\rt{pmid\_with\_not\_eq\_mid}}
  \CASE{\NT{pmid} \OPT{!= \NT{mid}}}
  \CASE{\NT{pmid} \OPT{!= \ttlb~\NT{COMMA\_LIST}\mth{(}\NT{mid}\mth{)} \ttrb}}
\end{grammar}

Subsequently, we refer to arbitrary metavariables as
\mth{\msf{metaid}^{\mbox{\scriptsize{\it{ty}}}}}, where {\it{ty}}
indicates the {\it metakind} used in the declaration of the variable.
For example, \mth{\msf{metaid}^{\ssf{Type}}} refers to a metavariable
that was declared using \texttt{type} and stands for any type.

The \NT{ctype} and \NT{ctypes} nonterminals are used by both the grammar of
metavariable declarations and the grammar of transformations, and are
defined on page~\pageref{types}.

An identifier metavariable with {\tt virtual} as its ``rule name'' is given
a value on the command line.  For example, if a semantic patch contains a
rule that declares an identifier metavariable with the name {\tt
  virtual.alloc}, then the command line could contain {\tt -D
  alloc=kmalloc}.  There should not be space around the {\tt =}.  An
example is in {\tt demos/vm.cocci} and {\tt demos/vm.c}.

\section{Metavariables for scripts}

Metavariables for scripts can only be inherited from transformation rules.
In the spirit of scripting languages such as Python that use dynamic
typing, metavariables for scripts do not include type declarations.

\begin{grammar}
  \RULE{\rt{script\_metavariables}}
  \CASE{@ script:\NT{language} \OPT{\NT{rulename}} \OPT{depends on \NT{dep}} @
        \any{\NT{script\_metadecl}} @@}
  \CASE{@ initialize:\NT{language} \OPT{depends on \NT{dep}} @}
  \CASE{@ finalize:\NT{language} \OPT{depends on \NT{dep}} @}

  \RULE{\rt{language}} \CASE{python} \CASE{ocaml}

  \RULE{\rt{script\_metadecl}}
  \CASE{\T{id} <{}< \T{rulename\_id}.\T{id} ;}
  \CASE{\T{id} ;}
\end{grammar}

Currently, the only scripting languages that are supported are Python and
OCaml, indicated using {\tt python} and {\tt ocaml}, respectively.  The
set of available scripting languages may be extended at some point.

Script rules declared with \KW{initialize} are run before the treatment of
any file.  Script rules declared with \KW{finalize} are run when the
treatment of all of the files has completed.  There can be at most one of
each per scripting language (thus currently at most one of each).
Initialize and finalize script rules do not have access to SmPL
metavariables.  Nevertheless, a finalize script rule can access any
variables initialized by the other script rules, allowing information to be
transmitted from the matching process to the finalize rule.

A script metavariable that does not specify an origin, using \texttt{<<},
is newly declared by the script.  This metavariable should be assigned to a
string and can be inherited by subsequent rules as an identifier.  In
Python, the assignment of such a metavariable $x$ should refer to the
metavariable as {\tt coccinelle.\(x\)}.  Examples are in the files
\texttt{demos/pythontococci.cocci} and \texttt{demos/camltococci.cocci}.

In an ocaml script, the following extended form of \textit{script\_metadecl}
may be used:

\begin{grammar}
  \RULE{\rt{script\_metadecl}}
  \CASE{(\T{id},\T{id}) <{}< \T{rulename\_id}.\T{id} ;}
  \CASE{\T{id} <{}< \T{rulename\_id}.\T{id} ;}
  \CASE{\T{id} ;}
\end{grammar}

\noindent
In a declaration of the form \texttt{(\T{id},\T{id}) <{}<
  \T{rulename\_id}.\T{id} ;}, the left component of \texttt{(\T{id},\T{id})}
receives a string representation of the value of the inherited metavariable
while the right component receives its abstract syntax tree.  The file
\texttt{parsing\_c/ast\_c.ml} in the Coccinelle implementation gives some
information about the structure of the abstract syntax tree.  Either the
left or right component may be replaced by \verb+_+, indicating that the
string representation or abstract syntax trees representation is not
wanted, respectively.

\section{Transformation}

The transformation specification essentially has the form of C code,
except that lines to remove are annotated with \verb+-+ in the first
column, and lines to add are annotated with \verb-+-.  A
transformation specification can also use {\em dots}, ``\verb-...-'',
describing an arbitrary sequence of function arguments or instructions
within a control-flow path.  Dots may be modified with a {\tt when}
clause, indicating a pattern that should not occur anywhere within the
matched sequence.  Finally, a transformation can specify a disjunction
of patterns, of the form \mtt{( \mth{\mita{pat}_1} | \mita{\ldots} |
  \mth{\mita{pat}_n} )} where each \texttt{(}, \texttt{|} or
\texttt{)} is in column 0 or preceded by \texttt{\textbackslash}.

The grammar that we present for the transformation is not actually the
grammar of the SmPL code that can be written by the programmer, but is
instead the grammar of the slice of this consisting of the {\tt -}
annotated and the unannotated code (the context of the transformed lines),
or the {\tt +} annotated code and the unannotated code.  For example, for
parsing purposes, the following transformation
%presented in Section \ref{sec:seq2}
is split into the two variants shown below and each is parsed
separately.

\begin{center}
\begin{tabular}{c}
\begin{lstlisting}[language=Cocci]
  proc_info_func(...) {
    <...
@--    hostno
@++    hostptr->host_no
    ...>
 }
\end{lstlisting}\\
\end{tabular}
\end{center}

{%\sizecodebis
\begin{center}
\begin{tabular}{p{5cm}p{3cm}p{5cm}}
\begin{lstlisting}[language=Cocci]
  proc_info_func(...) {
    <...
@--    hostno
    ...>
 }
\end{lstlisting}
&&
\begin{lstlisting}[language=Cocci]
  proc_info_func(...) {
    <...
@++    hostptr->host_no
    ...>
 }
\end{lstlisting}
\end{tabular}
\end{center}
}

\noindent
Requiring that both slices parse correctly ensures that the rule matches
syntactically valid C code and that it produces syntactically valid C code.
The generated parse trees are then merged for use in the subsequent
matching and transformation process.

The grammar for the minus or plus slice of a transformation is as follows:

\begin{grammar}

  \RULE{\rt{transformation}}
  \CASE{\some{\NT{include}}}
  \CASE{\NT{OPTDOTSEQ}\mth{(}\NT{expr}, \NT{when}\mth{)}}
  \CASE{\NT{OPTDOTSEQ}\mth{(}\some{\NT{decl\_stmt}}, \NT{when}\mth{)}}
  \CASE{\NT{OPTDOTSEQ}\mth{(}\NT{fundecl}, \NT{when}\mth{)}}

  \RULE{\rt{include}}
  \CASE{\#include \T{include\_string}}

%  \RULE{\rt{fun\_decl\_stmt}}
%  \CASE{\NT{decl\_stmt}}
%  \CASE{\NT{fundecl}}

%  \CASE{\NT{ctype}}
%  \CASE{\ttlb \NT{initialize\_list} \ttrb}
%  \CASE{\NT{toplevel\_seq\_start\_after\_dots\_init}}
%
%  \RULE{\rt{toplevel\_seq\_start\_after\_dots\_init}}
%  \CASE{\NT{stmt\_dots} \NT{toplevel\_after\_dots}}
%  \CASE{\NT{expr} \opt{\NT{toplevel\_after\_exp}}}
%  \CASE{\NT{decl\_stmt\_expr} \opt{\NT{toplevel\_after\_stmt}}}
%
%  \RULE{\rt{stmt\_dots}}
%  \CASE{... \any{\NT{when}}}
%  \CASE{<... \any{\NT{when}} \NT{nest\_after\_dots} ...>}
%  \CASE{<+... \any{\NT{when}} \NT{nest\_after\_dots} ...+>}

  \RULE{\rt{when}}
  \CASE{when != \NT{when\_code}}
  \CASE{when = \NT{rule\_elem\_stmt}}
  \CASE{when \NT{COMMA\_LIST}\mth{(}\NT{any\_strict}\mth{)}}
  \CASE{when true != \NT{expr}}
  \CASE{when false != \NT{expr}}

  \RULE{\rt{when\_code}}
  \CASE{\NT{OPTDOTSEQ}\mth{(}\some{\NT{decl\_stmt}}, \NT{when}\mth{)}}
  \CASE{\NT{OPTDOTSEQ}\mth{(}\NT{expr}, \NT{when}\mth{)}}

  \RULE{\rt{rule\_elem\_stmt}}
  \CASE{\NT{one\_decl}}
  \CASE{\NT{expr};}
  \CASE{return \opt{\NT{expr}};}
  \CASE{break;}
  \CASE{continue;}
  \CASE{\bs(\NT{rule\_elem\_stmt} \SOME{\bs| \NT{rule\_elem\_stmt}}\bs)}

  \RULE{\rt{any\_strict}}
  \CASE{any}
  \CASE{strict}
  \CASE{forall}
  \CASE{exists}

%  \RULE{\rt{nest\_after\_dots}}
%  \CASE{\NT{decl\_stmt\_exp} \opt{\NT{nest\_after\_stmt}}}
%  \CASE{\opt{\NT{exp}} \opt{\NT{nest\_after\_exp}}}
%
%  \RULE{\rt{nest\_after\_stmt}}
%  \CASE{\NT{stmt\_dots} \NT{nest\_after\_dots}}
%  \CASE{\NT{decl\_stmt} \opt{\NT{nest\_after\_stmt}}}
%
%  \RULE{\rt{nest\_after\_exp}}
%  \CASE{\NT{stmt\_dots} \NT{nest\_after\_dots}}
%
%  \RULE{\rt{toplevel\_after\_dots}}
%  \CASE{\opt{\NT{toplevel\_after\_exp}}}
%  \CASE{\NT{exp} \opt{\NT{toplevel\_after\_exp}}}
%  \CASE{\NT{decl\_stmt\_expr} \NT{toplevel\_after\_stmt}}
%
%  \RULE{\rt{toplevel\_after\_exp}}
%  \CASE{\NT{stmt\_dots} \opt{\NT{toplevel\_after\_dots}}}
%
%  \RULE{\rt{decl\_stmt\_expr}}
%  \CASE{TMetaStmList$^\ddag$}
%  \CASE{\NT{decl\_var}}
%  \CASE{\NT{stmt}}
%  \CASE{(\NT{stmt\_seq} \ANY{| \NT{stmt\_seq}})}
%
%  \RULE{\rt{toplevel\_after\_stmt}}
%  \CASE{\NT{stmt\_dots} \opt{\NT{toplevel\_after\_dots}}}
%  \CASE{\NT{decl\_stmt} \NT{toplevel\_after\_stmt}}

\end{grammar}

\begin{grammar}
  \RULE{\rt{OPTDOTSEQ}\mth{(}\rt{grammar\_ds}, \rt{when\_ds}\mth{)}}
  \CASE{}\multicolumn{3}{r}{\hspace{1cm}
  \KW{\opt{... \opt{\NT{when\_ds}}} \NT{grammar\_ds}
    \ANY{... \opt{\NT{when\_ds}} \NT{grammar\_ds}}
    \opt{... \opt{\NT{when\_ds}}}}
  }

%  \CASE{\opt{... \opt{\NT{when\_ds}}} \NT{grammar}
%    \ANY{... \opt{\NT{when\_ds}} \NT{grammar}}
%    \opt{... \opt{\NT{when\_ds}}}}
%  \CASE{<... \any{\NT{when\_ds}} \NT{grammar} ...>}
%  \CASE{<+... \any{\NT{when\_ds}} \NT{grammar} ...+>}

\end{grammar}

\noindent
Lines may be annotated with an element of the set $\{\mtt{-}, \mtt{+},
\mtt{*}\}$ or the singleton $\mtt{?}$, or one of each set. \mtt{?}
represents at most one match of the given pattern. \mtt{*} is used for
semantic match, \emph{i.e.}, a pattern that highlights the fragments
annotated with \mtt{*}, but does not perform any modification of the
matched code. \mtt{*} cannot be mixed with \mtt{-} and \mtt{+}.  There are
some constraints on the use of these annotations:
\begin{itemize}
\item Dots, {\em i.e.} \texttt{...}, cannot occur on a line marked
  \texttt{+}.
\item Nested dots, {\em i.e.}, dots enclosed in {\tt <} and {\tt >}, cannot
  occur on a line with any marking.
\end{itemize}

Each element of a disjunction must be a proper term like an
expression, a statement, an identifier or a declaration. Thus, the
rule on the left below is not a syntactically correct SmPL rule. One may
use the rule on the right instead.

\begin{center}
  \begin{tabular}{l@{\hspace{5cm}}r}
\begin{lstlisting}[language=Cocci]
@@
type T;
T b;
@@

(
 writeb(...,
|
 readb(
)
@--(T)
 b)
\end{lstlisting}
    &
\begin{lstlisting}[language=Cocci]
@@
type T;
T b;
@@

(
read
|
write
)
 (...,
@-- (T)
  b)
\end{lstlisting}
    \\
  \end{tabular}
\end{center}

\section{Types}
\label{types}

\begin{grammar}

  \RULE{\rt{ctypes}}
  \CASE{\NT{COMMA\_LIST}\mth{(}\NT{ctype}\mth{)}}

  \RULE{\rt{ctype}}
  \CASE{\opt{\NT{const\_vol}} \NT{generic\_ctype} \any{*}}
  \CASE{\opt{\NT{const\_vol}} void \some{*}}
  \CASE{(\NT{ctype} \ANY{| \NT{ctype}})}

  \RULE{\rt{const\_vol}}
  \CASE{const}
  \CASE{volatile}

  \RULE{\rt{generic\_ctype}}
  \CASE{\NT{ctype\_qualif}}
  \CASE{\opt{\NT{ctype\_qualif}} char}
  \CASE{\opt{\NT{ctype\_qualif}} short}
  \CASE{\opt{\NT{ctype\_qualif}} int}
  \CASE{\opt{\NT{ctype\_qualif}} long}
  \CASE{\opt{\NT{ctype\_qualif}} long long}
  \CASE{double}
  \CASE{float}
  \CASE{size\_t} \CASE{ssize\_t} \CASE{ptrdiff\_t}
  \CASE{enum \NT{id} \{ \NT{PARAMSEQ}\mth{(}\NT{dot\_expr}, \NT{exp\_whencode}\mth{)} \OPT{,} \}}
  \CASE{\OPT{struct\OR union} \T{id} \OPT{\{ \any{\NT{struct\_decl\_list}} \}}}

  \RULE{\rt{ctype\_qualif}}
  \CASE{unsigned}
  \CASE{signed}

  \RULE{\rt{struct\_decl\_list}}
  \CASE{\NT{struct\_decl\_list\_start}}

  \RULE{\rt{struct\_decl\_list\_start}}
  \CASE{\NT{struct\_decl}}
  \CASE{\NT{struct\_decl} \NT{struct\_decl\_list\_start}}
  \CASE{... \opt{when != \NT{struct\_decl}}$^\dag$ \opt{\NT{continue\_struct\_decl\_list}}}

  \RULE{\rt{continue\_struct\_decl\_list}}
  \CASE{\NT{struct\_decl} \NT{struct\_decl\_list\_start}}
  \CASE{\NT{struct\_decl}}

  \RULE{\rt{struct\_decl}}
  \CASE{\NT{ctype} \NT{d\_ident};}
  \CASE{\NT{fn\_ctype} (* \NT{d\_ident}) (\NT{PARAMSEQ}\mth{(}\NT{name\_opt\_decl}, \mth{\varepsilon)});)}
  \CASE{\opt{\NT{const\_vol}} \T{id} \NT{d\_ident};}

  \RULE{\rt{d\_ident}}
  \CASE{\T{id} \any{[\opt{\NT{expr}}]}}

  \RULE{\rt{fn\_ctype}}
  \CASE{\NT{generic\_ctype} \any{*}}
  \CASE{void \any{*}}

  \RULE{\rt{name\_opt\_decl}}
  \CASE{\NT{decl}}
  \CASE{\NT{ctype}}
  \CASE{\NT{fn\_ctype}}
\end{grammar}

$^\dag$ The optional \texttt{when} construct ends at the end of the line.

\section{Function declarations}

\begin{grammar}

  \RULE{\rt{fundecl}}
  \CASE{\opt{\NT{fn\_ctype}} \any{\NT{funinfo}} \NT{funid}
    (\opt{\NT{PARAMSEQ}\mth{(}\NT{param}, \mth{\varepsilon)}})
    \ttlb~\opt{\NT{stmt\_seq}} \ttrb}

  \RULE{\rt{funproto}}
  \CASE{\opt{\NT{fn\_ctype}} \any{\NT{funinfo}} \NT{funid}
    (\opt{\NT{PARAMSEQ}\mth{(}\NT{param}, \mth{\varepsilon)}});}

  \RULE{\rt{funinfo}}
  \CASE{inline}
  \CASE{\NT{storage}}
%   \CASE{\NT{attr}}

  \RULE{\rt{storage}}
  \CASE{static}
  \CASE{auto}
  \CASE{register}
  \CASE{extern}

  \RULE{\rt{funid}}
  \CASE{\T{id}}
  \CASE{\mth{\T{metaid}^{\ssf{Id}}}}
%   \CASE{\mth{\T{metaid}^{\ssf{Func}}}}
%   \CASE{\mth{\T{metaid}^{\ssf{LocalFunc}}}}

  \RULE{\rt{param}}
  \CASE{\NT{type} \T{id}}
  \CASE{\mth{\T{metaid}^{\ssf{Param}}}}
  \CASE{\mth{\T{metaid}^{\ssf{ParamList}}}}

  \RULE{\rt{decl}}
  \CASE{\NT{ctype} \NT{id}}
  \CASE{\NT{fn\_ctype} (* \NT{id}) (\NT{PARAMSEQ}\mth{(}\NT{name\_opt\_decl}, \mth{\varepsilon)})}
  \CASE{void}
  \CASE{\mth{\T{metaid}^{\ssf{Param}}}}
\end{grammar}

\begin{grammar}
  \RULE{\rt{PARAMSEQ}\mth{(}\rt{gram\_p}, \rt{when\_p}\mth{)}}
  \CASE{\NT{COMMA\_LIST}\mth{(}\NT{gram\_p} \OR \ldots \opt{\NT{when\_p}}\mth{)}}
\end{grammar}

To match a function it is not necessary to provide all of the annotations
that appear before the function name.  For example, the following semantic
patch:

\begin{lstlisting}[language=Cocci]
@@
@@

foo() { ... }
\end{lstlisting}

\noindent
matches a function declared as follows:

\begin{lstlisting}[language=C]
static int foo() { return 12; }
\end{lstlisting}

\noindent
This behavior can be turned off by disabling the \KW{optional\_storage}
isomorphism.  If one adds code before a function declaration, then the
effect depends on the kind of code that is added.  If the added code is a
function definition or CPP code, then the new code is placed before
all information associated with the function definition, including any
comments preceeding the function definition.  On the other hand, if the new
code is associated with the function, such as the addition of the keyword
{\tt static}, the new code is placed exactly where it appears with respect
to the rest of the function definition in the semantic patch.  For example, 

\begin{lstlisting}[language=Cocci]
@@
@@

+ static
foo() { ... }
\end{lstlisting}

\noindent
causes static to be placed just before the function name.  The following
causes it to be placed just before the type

\begin{lstlisting}[language=Cocci]
@@
type T;
@@

+ static
T foo() { ... }
\end{lstlisting}

\noindent
It may be necessary to consider several cases to ensure that the added ode
is placed in the right position.  For example, one may need one pattern
that considers that the function is declared {\tt inline} and another that
considers that it is not.

%\newpage

\section{Declarations}

\begin{grammar}
  \RULE{\rt{decl\_var}}
%  \CASE{\NT{type} \opt{\NT{id} \opt{[\opt{\NT{dot\_expr}}]}
%      \ANY{, \NT{id} \opt{[ \opt{\NT{dot\_expr}}]}}};}
  \CASE{\NT{common\_decl}}
  \CASE{\opt{\NT{storage}} \NT{ctype} \NT{COMMA\_LIST}\mth{(}\NT{d\_ident}\mth{)} ;}
  \CASE{\opt{\NT{storage}} \opt{\NT{const\_vol}} \T{id} \NT{COMMA\_LIST}\mth{(}\NT{d\_ident}\mth{)} ;}
  \CASE{\opt{\NT{storage}} \NT{fn\_ctype} ( * \NT{d\_ident} ) ( \NT{PARAMSEQ}\mth{(}\NT{name\_opt\_decl}, \mth{\varepsilon)} ) = \NT{initialize} ;}
  \CASE{typedef \NT{ctype} \NT{typedef\_ident} ;}

  \RULE{\rt{one\_decl}}
  \CASE{\NT{common\_decl}}
  \CASE{\opt{\NT{storage}} \NT{ctype} \NT{id};}
%  \CASE{\NT{storage} \NT{ctype} \NT{id} \opt{[\opt{\NT{dot\\_expr}}]} = \NT{nest\\_expr};}
  \CASE{\opt{\NT{storage}} \opt{\NT{const\_vol}} \T{id} \NT{d\_ident} ;}

  \RULE{\rt{common\_decl}}
  \CASE{\NT{ctype};}
  \CASE{\NT{funproto}}
  \CASE{\opt{\NT{storage}} \NT{ctype} \NT{d\_ident} = \NT{initialize} ;}
  \CASE{\opt{\NT{storage}} \opt{\NT{const\_vol}} \T{id} \NT{d\_ident} = \NT{initialize} ;}
  \CASE{\opt{\NT{storage}} \NT{fn\_ctype} ( * \NT{d\_ident} ) ( \NT{PARAMSEQ}\mth{(}\NT{name\_opt\_decl}, \mth{\varepsilon)} ) ;}
  \CASE{\NT{decl\_ident} ( \OPT{\NT{COMMA\_LIST}\mth{(}\NT{expr}\mth{)}} ) ;}

  \RULE{\rt{initialize}}
  \CASE{\NT{dot\_expr}}
  \CASE{\mth{\T{metaid}^{\ssf{Initialiser}}}}
  \CASE{\ttlb~\opt{\NT{COMMA\_LIST}\mth{(}\NT{dot\_expr}\mth{)}}~\ttrb}

  \RULE{\rt{init\_list\_elem}}
  \CASE{\NT{dot\_expr}}
  \CASE{\NT{designator} = \NT{dot\_expr}}
  \CASE{\NT{id} : \NT{dot\_expr}}

  \RULE{\rt{designator}}
  \CASE{. \NT{id}}
  \CASE{[ \NT{dot\_expr} ]}
  \CASE{[ \NT{dot\_expr} ... \NT{dot\_expr} ]}

  \RULE{\rt{decl\_ident}}
  \CASE{\T{DeclarerId}}
  \CASE{\mth{\T{metaid}^{\ssf{Declarer}}}}
\end{grammar}

\section{Statements}

The first rule {\em statement} describes the various forms of a statement.
The remaining rules implement the constraints that are sensitive to the
context in which the statement occurs: {\em single\_statement} for a
context in which only one statement is allowed, and {\em decl\_statement}
for a context in which a declaration, statement, or sequence thereof is
allowed.

\begin{grammar}
  \RULE{\rt{stmt}}
  \CASE{\NT{include}}
  \CASE{\mth{\T{metaid}^{\ssf{Stmt}}}}
  \CASE{\NT{expr};}
  \CASE{if (\NT{dot\_expr}) \NT{single\_stmt} \opt{else \NT{single\_stmt}}}
  \CASE{for (\opt{\NT{dot\_expr}}; \opt{\NT{dot\_expr}}; \opt{\NT{dot\_expr}})
    \NT{single\_stmt}}
  \CASE{while (\NT{dot\_expr}) \NT{single\_stmt}}
  \CASE{do \NT{single\_stmt} while (\NT{dot\_expr});}
  \CASE{\NT{iter\_ident} (\any{\NT{dot\_expr}}) \NT{single\_stmt}}
  \CASE{switch (\opt{\NT{dot\_expr}}) \ttlb \any{\NT{case\_line}} \ttrb}
  \CASE{return \opt{\NT{dot\_expr}};}
  \CASE{\ttlb~\opt{\NT{stmt\_seq}} \ttrb}
  \CASE{\NT{NEST}\mth{(}\some{\NT{decl\_stmt}}, \NT{when}\mth{)}}
  \CASE{\NT{NEST}\mth{(}\NT{expr}, \NT{when}\mth{)}}
  \CASE{break;}
  \CASE{continue;}
  \CASE{\NT{id}:}
  \CASE{goto \NT{id};}
  \CASE{\ttlb \NT{stmt\_seq} \ttrb}

  \RULE{\rt{single\_stmt}}
  \CASE{\NT{stmt}}
  \CASE{\NT{OR}\mth{(}\NT{stmt}\mth{)}}

  \RULE{\rt{decl\_stmt}}
  \CASE{\mth{\T{metaid}^{\ssf{StmtList}}}}
  \CASE{\NT{decl\_var}}
  \CASE{\NT{stmt}}
  \CASE{\NT{OR}\mth{(}\NT{stmt\_seq}\mth{)}}

  \RULE{\rt{stmt\_seq}}
  \CASE{\any{\NT{decl\_stmt}}
    \opt{\NT{DOTSEQ}\mth{(}\some{\NT{decl\_stmt}},
      \NT{when}\mth{)} \any{\NT{decl\_stmt}}}}
  \CASE{\any{\NT{decl\_stmt}}
    \opt{\NT{DOTSEQ}\mth{(}\NT{expr},
      \NT{when}\mth{)} \any{\NT{decl\_stmt}}}}

  \RULE{\rt{case\_line}}
  \CASE{default :~\NT{stmt\_seq}}
  \CASE{case \NT{dot\_expr} :~\NT{stmt\_seq}}

  \RULE{\rt{iter\_ident}}
  \CASE{\T{IteratorId}}
  \CASE{\mth{\T{metaid}^{\ssf{Iterator}}}}
\end{grammar}

\begin{grammar}
  \RULE{\rt{OR}\mth{(}\rt{gram\_o}\mth{)}}
  \CASE{( \NT{gram\_o} \ANY{\ttmid \NT{gram\_o}})}

  \RULE{\rt{DOTSEQ}\mth{(}\rt{gram\_d}, \rt{when\_d}\mth{)}}
  \CASE{\ldots \opt{\NT{when\_d}} \ANY{\NT{gram\_d} \ldots \opt{\NT{when\_d}}}}

  \RULE{\rt{NEST}\mth{(}\rt{gram\_n}, \rt{when\_n}\mth{)}}
  \CASE{<\ldots \opt{\NT{when\_n}} \NT{gram\_n} \ANY{\ldots \opt{\NT{when\_n}} \NT{gram\_n}} \ldots>}
  \CASE{<+\ldots \opt{\NT{when\_n}} \NT{gram\_n} \ANY{\ldots \opt{\NT{when\_n}} \NT{gram\_n}} \ldots+>}
\end{grammar}

\noindent
OR is a macro that generates a disjunction of patterns.  The three
tokens \T{(}, \T{\ttmid}, and \T{)} must appear in the leftmost
column, to differentiate them from the parentheses and bit-or tokens
that can appear within expressions (and cannot appear in the leftmost
column). These token may also be preceded by \texttt{\bs}
when they are used in an other column.  These tokens are furthermore
different from (, \(\mid\), and ), which are part of the grammar
metalanguage.

\section{Expressions}

A nest or a single ellipsis is allowed in some expression contexts, and
causes ambiguity in others.  For example, in a sequence \mtt{\ldots
\mita{expr} \ldots}, the nonterminal \mita{expr} must be instantiated as an
explicit C-language expression, while in an array reference,
\mtt{\mth{\mita{expr}_1} \mtt{[} \mth{\mita{expr}_2} \mtt{]}}, the
nonterminal \mth{\mita{expr}_2}, because it is delimited by brackets, can
be also instantiated as \mtt{\ldots}, representing an arbitrary expression.  To
distinguish between the various possibilities, we define three nonterminals
for expressions: {\em expr} does not allow either top-level nests or
ellipses, {\em nest\_expr} allows a nest but not an ellipsis, and {\em
dot\_expr} allows both.  The EXPR macro is used to express these variants
in a concise way.

\begin{grammar}
  \RULE{\rt{expr}}
  \CASE{\NT{EXPR}\mth{(}\NT{expr}\mth{)}}

  \RULE{\rt{nest\_expr}}
  \CASE{\NT{EXPR}\mth{(}\NT{nest\_expr}\mth{)}}
  \CASE{\NT{NEST}\mth{(}\NT{nest\_expr}, \NT{exp\_whencode}\mth{)}}

  \RULE{\rt{dot\_expr}}
  \CASE{\NT{EXPR}\mth{(}\NT{dot\_expr}\mth{)}}
  \CASE{\NT{NEST}\mth{(}\NT{dot\_expr}, \NT{exp\_whencode}\mth{)}}
  \CASE{...~\opt{\NT{exp\_whencode}}}

  \RULE{\rt{EXPR}\mth{(}\rt{exp}\mth{)}}
  \CASE{\NT{exp} \NT{assign\_op} \NT{exp}}
  \CASE{\NT{exp}++}
  \CASE{\NT{exp}--}
  \CASE{\NT{unary\_op} \NT{exp}}
  \CASE{\NT{exp} \NT{bin\_op} \NT{exp}}
  \CASE{\NT{exp} ?~\NT{dot\_expr} :~\NT{exp}}
  \CASE{(\NT{type}) \NT{exp}}
  \CASE{\NT{exp} [\NT{dot\_expr}]}
  \CASE{\NT{exp} .~\NT{id}}
  \CASE{\NT{exp} -> \NT{id}}
  \CASE{\NT{exp}(\opt{\NT{PARAMSEQ}\mth{(}\NT{arg}, \NT{exp\_whencode}\mth{)}})}
  \CASE{\NT{id}}
%   \CASE{\mth{\T{metaid}^{\ssf{Func}}}}
%   \CASE{\mth{\T{metaid}^{\ssf{LocalFunc}}}}
  \CASE{\mth{\T{metaid}^{\ssf{Exp}}}}
%   \CASE{\mth{\T{metaid}^{\ssf{Err}}}}
  \CASE{\mth{\T{metaid}^{\ssf{Const}}}}
  \CASE{\NT{const}}
  \CASE{(\NT{dot\_expr})}
  \CASE{\NT{OR}\mth{(}\NT{exp}\mth{)}}

  \RULE{\rt{arg}}
  \CASE{\NT{nest\_expr}}
  \CASE{\mth{\T{metaid}^{\ssf{ExpList}}}}

  \RULE{\rt{exp\_whencode}}
  \CASE{when != \NT{expr}}

  \RULE{\rt{assign\_op}}
  \CASE{= \OR -= \OR += \OR *= \OR /= \OR \%=}
  \CASE{\&= \OR |= \OR \caret= \OR \lt\lt= \OR \gt\gt=}

  \RULE{\rt{bin\_op}}
  \CASE{* \OR / \OR \% \OR + \OR -}
  \CASE{\lt\lt \OR \gt\gt \OR \caret\xspace \OR \& \OR \ttmid}
  \CASE{< \OR > \OR <= \OR >= \OR == \OR != \OR \&\& \OR \ttmid\ttmid}

  \RULE{\rt{unary\_op}}
  \CASE{++ \OR -- \OR \& \OR * \OR + \OR - \OR !}

\end{grammar}

\section{Constant, Identifiers and Types for Transformations}

\begin{grammar}
  \RULE{\rt{const}}
  \CASE{\NT{string}}
  \CASE{[0-9]+}
  \CASE{\mth{\cdots}}

  \RULE{\rt{string}}
  \CASE{"\any{[\^{}"]}"}

  \RULE{\rt{id}}
  \CASE{\T{id} \OR \mth{\T{metaid}^{\ssf{Id}}}}

  \RULE{\rt{typedef\_ident}}
  \CASE{\T{id} \OR \mth{\T{metaid}^{\ssf{Type}}}}

  \RULE{\rt{type}}
  \CASE{\NT{ctype} \OR \mth{\T{metaid}^{\ssf{Type}}}}

  \RULE{\rt{pathToIsoFile}}
  \CASE{<.*>}

  \RULE{\rt{regexp}}
  \CASE{"\any{[\^{}"]}"}
\end{grammar}


%%% Local Variables:
%%% mode: LaTeX
%%% TeX-master: "main_grammar"
%%% coding: utf-8
%%% TeX-PDF-mode: t
%%% ispell-local-dictionary: "american"
%%% End:
