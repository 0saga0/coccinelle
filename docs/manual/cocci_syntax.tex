
%\section{The SmPL Grammar}

% This section presents the SmPL grammar.  This definition follows closely
% our implementation using the Menhir parser generator \cite{menhir}.

This document presents the grammar of the SmPL language used by the
\href{http://coccinelle.lip6.fr/}{Coccinelle tool}.  For the most
part, the grammar is written using standard notation.  In some rules,
however, the left-hand side is in all uppercase letters.  These are
macros, which take one or more grammar rule right-hand-sides as
arguments.  The grammar also uses some unspecified nonterminals, such
as \T{id}, \T{const}, etc.  These refer to the sets suggested by
the name, {\em i.e.}, \T{id} refers to the set of possible
C-language identifiers, while \T{const} refers to the set of
possible C-language constants.

A square bracket that is surrounded by spaces in the description of a term
should appear explicitly in the term, as in an array reference.  On the
other hand, square brackets that surround some other term indicat that the
presence of that term is optional.

%
\ifhevea
A PDF version of this documentation is available at
\url{http://coccinelle.lip6.fr/docs/main_grammar.pdf}.
\else
A HTML version of this documentation is available online at
\url{http://coccinelle.lip6.fr/docs/main_grammar.html}.
\fi

\section{Program}

\begin{grammar}
  \RULE{\rt{program}}
  \CASE{\any{\NT{include\_cocci}} \some{\NT{changeset}}}

  \RULE{\rt{include\_cocci}}
  \CASE{include \NT{string}}
  \CASE{using \NT{string}}
  \CASE{using \NT{pathToIsoFile}}
  \CASE{virtual \T{id} \ANY{, \T{id}}}

  \RULE{\rt{changeset}}
  \CASE{\NT{metavariables} \NT{transformation}}
  \CASE{\NT{script\_metavariables} \T{script\_code}}
%  \CASE{\NT{metavariables} \ANY{--- filename +++ filename} \NT{transformation}}
\end{grammar}

\noindent
\T{script\_code} is any code in the chosen scripting language.  Parsing of
the semantic patch does not check the validity of this code; any errors are
first detected when the code is executed.  Furthermore, \texttt{@} should
not be use in this code.  Spatch scans the script code for the next
\texttt{@} and considers that to be the beginning of the next rule, even if
\texttt{@} occurs within e.g., a comment.

\texttt{virtual} keyword is used to declare virtual rules. Virtual
rules may be subsequently used as a dependency for the rules in the
SmPL file. Whether a virtual rule is defined or not is controlled by
the \texttt{-D} option on the command line.

% Between the metavariables and the transformation rule, there can be a
% specification of constraints on the names of the old and new files,
% analogous to the filename specifications in the standard patch syntax.
% (see Figure \ref{scsiglue_patch}).

\section{Metavariables for transformations}

The \NT{rulename} portion of the metavariable declaration can specify
properties of a rule such as its name, the names of the rules that it
depends on, the isomorphisms to be used in processing the rule, and whether
quantification over paths should be universal or existential.  The optional
annotation {\tt expression} indicates that the pattern is to be considered
as matching an expression, and thus can be used to avoid some parsing
problems.

The \NT{metadecl} portion of the metavariable declaration defines various
types of metavariables that will be used for matching in the transformation
section.

\begin{grammar}
  \RULE{\rt{metavariables}}
  \CASE{@@ \any{\NT{metadecl}} @@}
  \CASE{@ \NT{rulename} @ \any{\NT{metadecl}} @@}

  \RULE{\rt{rulename}}
  \CASE{\T{id} \OPT{extends \T{id}} \OPT{depends on \NT{dep}} \opt{\NT{iso}}
    \opt{\NT{disable-iso}} \opt{\NT{exists}} \opt{expression}}

  \RULE{\rt{dep}}
  \CASE{\T{id}}
  \CASE{!\T{id}}
  \CASE{!(\NT{dep})}
  \CASE{ever \T{id}}
  \CASE{never \T{id}}
  \CASE{\NT{dep} \&\& \NT{dep}}
  \CASE{\NT{dep} || \NT{dep}}
  \CASE{file in \NT{string}}
  \CASE{(\NT{dep})}

  \RULE{\rt{iso}}
  \CASE{using \NT{string} \ANY{, \NT{string}}}

  \RULE{\rt{disable-iso}}
  \CASE{disable \NT{COMMA\_LIST}\mth{(}\T{id}\mth{)}}

  \RULE{\rt{exists}}
  \CASE{exists}
  \CASE{forall}
%  \CASE{\opt{reverse} forall}

  \RULE{\rt{COMMA\_LIST}\mth{(}\rt{elem}\mth{)}}
  \CASE{\NT{elem} \ANY{, \NT{elem}}}
\end{grammar}

The keyword \KW{disable} is normally used with the names of
isomorphisms defined in standard.iso or whatever isomorphism file has been
included.  There are, however, some other isomorphisms that are built into
the implementation of Coccinelle and that can be disabled as well.  Their
names are given below.  In each case, the text describes the standard
behavior.  Using \NT{disable-iso} with the given name disables this behavior.

\begin{itemize}
\item \KW{optional\_storage}: A SmPL function definition that does not
  specify any visibility (i.e., static or extern), or a SmPL variable
  declaration that does not specify any storage (i.e., auto, static,
  register, or extern), matches a function declaration or variable
  declaration with any visibility or storage, respectively.
\item \KW{optional\_qualifier}: This is similar to \KW{optional\_storage},
  except that here it is the qualifier (i.e., const or volatile) that does
  not have to be specified in the SmPL code, but may be present in the C code.
\item \KW{optional\_attributes}: This is similar to
  \KW{optional\_attributes}, except that here is it an attribute (e.g.,
  \_\_init) that does not have to be specified in the SmPL code, but may be
  present in the C code.  \textbf{Note that this isomorphism is currently
    useless, because matching of attributes is not supported, due to the
    difficulty of parsing attributes in C code.}
\item \KW{value\_format}: Integers in various formats, e.g., 1 and 0x1, are
  considered to be equivalent in the matching process.
\item \KW{optional\_declarer\_semicolon}: Some declarers (top-level terms
  that look like function calls but serve to declare some variable) don't
  require a semicolon.  This isomorphism allows a SmPL declarer with a semicolon
  to match such a C declarer, if no transformation is specified on the SmPL
  semicolon.
\item \KW{comm\_assoc}: An expression of the form \NT{exp} \NT{bin\_op}
  \KW{...}, where \NT{bin\_op} is commutative and associative, is
  considered to match any top-level sequence of \NT{bin\_op} operators
  containing \NT{exp} as the top-level argument.
\item \KW{prototypes}: A rule for transforming a function prototype is
  generated when a function header changes.
\end{itemize}

The \texttt{depends on} clause indicates conditions under which a semantic
patch rule should be applied.  Most of these conditions relate to the
success or failure of other rules, which may be virtual rules.  Giving the
name of a rule implies that the current rule is applied if the named rule
has succeeded in matching in the current environment.  Giving \texttt{ever}
followed by a rule name implies that the current rule is applied if the
named rule has succeeded in matching in any environment.  Analogously,
\texttt{never} means that the named rule should have succeeded in matching
in no environment.  The boolean and, or and negation operators combine
these declarations in the usual way.  The declaration {\tt file in} checks
that the code being processed comes from the mentioned file, or from a
subdirectory.  The declaration {\tt file in} is only allowed on SmPL
code-matching rules.  Script rules are not applied to any code in
particular, and thus it doesn't make sense to check on the file being
considered.

The possible types of metavariable declarations are defined by the grammar
rule below.  Metavariables should occur at least once in the transformation
code immediately following their declaration.  Fresh identifier
metavariables must only be used in {\tt +} code.  These properties are not
expressed in the grammar, but are checked by a subsequent analysis.  The
metavariables are designated according to the kind of terms they can match,
such as a statement, an identifier, or an expression.  An expression
metavariable can be further constrained by its type.  A declaration
metavariable matches the declaration of one or more variables, all sharing
the same type specification ({\em e.g.}, {\tt int a,b,c=3;}).  A field
metavariable does the same, but for structure fields.  In the minus code, a
statement list metavariable can only appear as a complete function body or
as the complete body of a sequence statement.  In the plus code, a
statement list metavariable can occur anywhere a statement list is allowed,
i.e., including as an element of another statement list.

\begin{grammar}
  \RULE{\rt{metadecl}}
  \CASE{metavariable \NT{ids} ;}
  \CASE{fresh identifier \NT{ids} ;}
  \CASE{identifier \NT{COMMA\_LIST}\mth{(}\NT{pmid\_with\_regexp}\mth{)} ;}
  \CASE{identifier \NT{COMMA\_LIST}\mth{(}\NT{pmid\_with\_virt\_or\_not\_eq}\mth{)} ;}
  \CASE{parameter \opt{list} \NT{ids} ;}
  \CASE{parameter list [ \NT{id} ] \NT{ids} ;}
  \CASE{parameter list [ \NT{const} ] \NT{ids} ;}
  \CASE{type \NT{ids} ;}
  \CASE{statement \opt{list} \NT{ids} ;}
  \CASE{declaration \NT{ids} ;}
  \CASE{field \opt{list} \NT{ids} ;}
  \CASE{typedef \NT{ids} ;}
  \CASE{attribute \NT{ids} ;}
  \CASE{declarer name \NT{ids} ;}
%  \CASE{\opt{local} function \NT{pmid\_with\_not\_eq\_list} ;}
  \CASE{declarer \NT{COMMA\_LIST}\mth{(}\NT{pmid\_with\_regexp}\mth{)} ;}
  \CASE{declarer \NT{COMMA\_LIST}\mth{(}\NT{pmid\_with\_not\_eq}\mth{)} ;}
  \CASE{iterator name \NT{ids} ;}
  \CASE{iterator \NT{COMMA\_LIST}\mth{(}\NT{pmid\_with\_regexp}\mth{)} ;}
  \CASE{iterator \NT{COMMA\_LIST}\mth{(}\NT{pmid\_with\_not\_eq}\mth{)} ;}
%  \CASE{error \NT{pmid\_with\_not\_eq\_list} ; }
  \CASE{\opt{local \mth{\mid} global} idexpression \opt{\NT{ctype}} \NT{COMMA\_LIST}\mth{(}\NT{pmid\_with\_not\_eq}\mth{)} ;}
  \CASE{\opt{local \mth{\mid} global} idexpression \OPT{\ttlb \NT{ctypes}\ttrb~\any{*}} \NT{COMMA\_LIST}\mth{(}\NT{pmid\_with\_not\_eq}\mth{)} ;}
  \CASE{\opt{local \mth{\mid} global} idexpression \some{*} \NT{COMMA\_LIST}\mth{(}\NT{pmid\_with\_not\_eq}\mth{)} ;}
  \CASE{expression list \NT{ids} ;}
  \CASE{expression \some{*} \NT{COMMA\_LIST}\mth{(}\NT{pmid\_with\_not\_eq}\mth{)} ;}
  \CASE{expression enum \any{*} \NT{COMMA\_LIST}\mth{(}\NT{pmid\_with\_not\_eq}\mth{)} ;}
  \CASE{expression struct \any{*} \NT{COMMA\_LIST}\mth{(}\NT{pmid\_with\_not\_eq}\mth{)} ;}
  \CASE{expression union \any{*} \NT{COMMA\_LIST}\mth{(}\NT{pmid\_with\_not\_eq}\mth{)} ;}
  \CASE{expression \NT{COMMA\_LIST}\mth{(}\NT{pmid\_with\_not\_ceq}\mth{)} ;}
  \CASE{expression list [ \NT{id} ] \NT{ids} ;}
  \CASE{expression list [ \NT{const} ] \NT{ids} ;}
  \CASE{\NT{ctype} [ ] \NT{COMMA\_LIST}\mth{(}\NT{pmid\_with\_not\_eq}\mth{)} ;}
  \CASE{\NT{ctype} \NT{COMMA\_LIST}\mth{(}\NT{pmid\_with\_not\_ceq}\mth{)} ;}
  \CASE{\ttlb \NT{ctypes}\ttrb~\any{*} \NT{COMMA\_LIST}\mth{(}\NT{pmid\_with\_not\_ceq}\mth{)} ;}
  \CASE{\ttlb \NT{ctypes}\ttrb~\any{*} [ ] \NT{COMMA\_LIST}\mth{(}\NT{pmid\_with\_not\_eq}\mth{)} ;}
  \CASE{constant \opt{\NT{ctype}} \NT{COMMA\_LIST}\mth{(}\NT{pmid\_with\_not\_eq}\mth{)} ;}
  \CASE{constant \OPT{\ttlb \NT{ctypes}\ttrb~\any{*}} \NT{COMMA\_LIST}\mth{(}\NT{pmid\_with\_not\_eq}\mth{)} ;}
  \CASE{position \opt{any} \NT{COMMA\_LIST}\mth{(}\NT{pmid\_with\_not\_eq\_mid}\mth{)} ;}
  \CASE{symbol \NT{ids};}
  \CASE{format \NT{ids};}
  \CASE{format list [ \NT{id} ] \NT{ids} ;}
  \CASE{format list [ \NT{const} ] \NT{ids} ;}
  \CASE{assignment operator \NT{COMMA\_LIST}\mth{(}\T{assignopdecl}\mth{)} ;}
  \CASE{binary operator \NT{COMMA\_LIST}\mth{(}\T{binopdecl}\mth{)} ;}

  \RULE{\rt{assignopdecl}}
  \CASE{\NT{id} \OPT{ = \NT{assignop\_contraint}}}

  \RULE{\rt{assignop\_contraint}}
  \CASE{\mth{\{}\NT{COMMA\_LIST}\mth{(}\NT{assign\_op}\mth{)}\mth{\}}}
  \CASE{\NT{assign\_op}}

  \RULE{\rt{binopdecl}}
  \CASE{\NT{id} \OPT{ = \NT{binop\_contraint}}}

  \RULE{\rt{binop\_contraint}}
  \CASE{\mth{\{}\NT{COMMA\_LIST}\mth{(}\NT{bin\_op}\mth{)}\mth{\}}}
  \CASE{\NT{bin\_op}}
\end{grammar}

A metavariable declaration local idexpression v means that v is restricted
to be a local variable.  If it should just be a variable, but not
necessarily a local one, then drop local.  A more complex description of a
location, such as a->b is considered to be an expression, not an
idexpression.

Constant is for constants, such as 27.  But it also considers an identifier
that is all capital letters (possibly containing numbers) as a constant as
well, because the names given to macros in Linux usually have this form.

An identifier is the name of a structure field, a macro, a function, or a
variable.  Is is the name of something rather than an expression that has a
value.  But an identifier can be used in the position of an expression as
well, where it represents a variable.

It is possible to specify that an expression list or a parameter list
metavariable should match a specific number of expressions or parameters.

It is possible to specify some information about the definition of a fresh
identifier.  See the wiki.

A symbol declaration specifies that the provided identifiers should be
considered C identifiers when encountered in the body of the rule.
Identifiers in the body of the rule that are not declared explicitly are by
default considered symbols, thus symbol declarations are optional.  It is
not required, but it will not cause a parse error, to redeclare a name as a
symbol.  A name declared as a symbol can, however, be redeclared as another
metavariable.  It will be considered to be a metavariable in such rules,
and will revert to being a symbol in subsequent rules.  These conditions
also apply to iterator names and declarer names.

An attribute declaration indicates a name that should be considered to be
an attribute.  It is not possible to match or remove an attribute, only to
add one.

A position metavariable is used by attaching it using \texttt{@} to any
token, including another metavariable.  Its value is the position (file,
line number, etc.) of the code matched by the token.  It is also possible
to attach expression, declaration, type, initialiser, and statement
metavariables in this manner.  In that case, the metavariable is bound to
the closest enclosing expression, declaration, etc.  If such a metavariable
is itself followed by a position metavariable, the position metavariable
applies to the metavariable that it follows, and not to the attached token.
This makes it possible to get eg the starting and ending position of {\tt
  f(...)}, by writing {\tt f(...)@E@p}, for expression metavariable {\tt E}
and position metavariable {\tt p}. This attachment notation for
metavariables of type other than position can also be expressed with a
conjunction, but the @ notation may be more concise.

When used, a format or format list metavariable must be enclosed by a pair
of \texttt{@}s.  A format metavariable matches the format descriptor part,
i.e., \texttt{2x} in \texttt{\%2x}.  A format list metavariable matches a
sequence of format descriptors as well as the text between them.  Any text
around them is matched as well, if it is not matched by the surrounding
text in the semantic patch.  Such text is not partially matched.  If the
length of the format list is specified, that indicates the number of
matched format descriptors.  It is also possible to use \texttt{\ldots} in
a format string, to match a sequence of text fragments and format
descriptors.  This only takes effect if the format string contains format
descriptors.  Note that this makes it impossible to require \texttt{\ldots}
to match exactly in a string, if the semantic patch string contains format
descriptors.  If that is needed, some processing with a scripting language
would be required.  And example for the use of string format metavariables
is found in demos/format.cocci.

Assignment (resp. binary) operator metavariables match any assignment
(resp. binary) operator. The list of operators that can be matched can
be restricted by adding an operator constraint, i.e. a list of accepted
operators.

Other kinds of metavariables can also be attached using \texttt{@} to any
token.  In this case, the metavariable floats up to the enclosing
appropriate expression.  For example, {\tt 3 +@E 4}, where {\tt E} is an
expression metavariable binds {\tt E} to {\tt 3 + 4}.  A particular case is
{\tt Ps@Es}, where {\tt Ps} is a parameter list and {\tt Es} is an
expression list.  This pattern matches a parameter list, and then matches
{\tt Es} to the list of expressions, ie a possible argument list,
represented by the names of the parameters.  Another particular case is
{\tt E@S}, where {\tt E} is any expression and {\tt S} is a statement
metavariable.  {\tt S} matches the closest enclosing statement, which may
be more than what is matches by the semantic match pattern itself.

Matching of various kinds of format strings within strings is supported.
With the {\tt --ibm} option, matching of decimal format declarations is
supported, but the length and precision arguments are not interpreted.
Thus it is not possible to match metavariables in these fields.  Instead,
the entire format is matched as a single string.

\begin{grammar}
  \RULE{\rt{ids}}
  \CASE{\NT{COMMA\_LIST}\mth{(}\NT{pmid}\mth{)}}

  \RULE{\rt{pmid}}
  \CASE{\T{id}}
  \CASE{\NT{mid}}
%   \CASE{list}
%   \CASE{error}
%   \CASE{type}

  \RULE{\rt{mid}}  \CASE{\T{rulename\_id}.\T{id}}

  \RULE{\rt{pmid\_with\_regexp}}
  \CASE{\NT{pmid} =\~{} \NT{regexp}}
  \CASE{\NT{pmid} !\~{} \NT{regexp}}

  \RULE{\rt{pmid\_with\_not\_eq}}
  \CASE{\NT{pmid} \OPT{!= \NT{id\_or\_meta}}}
  \CASE{\NT{pmid}
     \OPT{!= \ttlb~\NT{COMMA\_LIST}\mth{(}\NT{id\_or\_meta}\mth{)} \ttrb}}

  \RULE{\rt{pmid\_with\_virt\_or\_not\_eq}}
  \CASE{virtual.\T{id}}
  \CASE{\NT{pmid\_with\_not\_eq}}

  \RULE{\rt{pmid\_with\_not\_ceq}}
  \CASE{\NT{pmid} \OPT{!= \NT{id\_or\_cst}}}
  \CASE{\NT{pmid} \OPT{!= \ttlb~\NT{COMMA\_LIST}\mth{(}\NT{id\_or\_cst}\mth{)} \ttrb}}

  \RULE{\rt{id\_or\_cst}}
  \CASE{\T{id}}
  \CASE{\T{integer}}

  \RULE{\rt{id\_or\_meta}}
  \CASE{\T{id}}
  \CASE{\T{rulename\_id}.\T{id}}

  \RULE{\rt{pmid\_with\_not\_eq\_mid}}
  \CASE{\NT{pmid} \OPT{\NT{ANDAND\_LIST}\mth{(}\NT{pos\_constraint}\mth{)}}}

  \RULE{\rt{pos\_constraint}}
  \CASE{!= \NT{mid}}
  \CASE{!= \ttlb~\NT{COMMA\_LIST}\mth{(}\NT{mid}\mth{)} \ttrb}
  \CASE{: script:ocaml
        (\NT{COMMA\_LIST}\mth{(} \NT{mid} \mth{)})
        \ttlb \NT{expr} \ttrb}

  \RULE{\rt{ANDAND\_LIST}\mth{(X)}}
  \CASE{\mth{X}}
  \CASE{\mth{X} \&\& \NT{ANDAND\_LIST}\mth{(X)}}
\end{grammar}

Subsequently, we refer to arbitrary metavariables as
\mth{\msf{metaid}^{\mbox{\scriptsize{\it{ty}}}}}, where {\it{ty}}
indicates the {\it metakind} used in the declaration of the variable.
For example, \mth{\msf{metaid}^{\ssf{Type}}} refers to a metavariable
that was declared using \texttt{type} and stands for any type.

{\tt metavariable} declares a metavariable for which the parser tried to
figure out the metavariable type based on the usage context.  Such a
metavariable must be used consistently.  These metavariables cannot be used
in all contexts; specifically, they cannot be used in context that would
make the parsing ambiguous.  Some examples are the leftmost term of an
expression, such as the left-hand side of an assignment, or the type in a
variable declaration.  These restrictions may seems somewhat arbitrary from
the user's point of view.  Thus, it is better to use metavariables with
metavariable types.  If Coccinelle is given the argument {\tt
  -parse\_cocci}, it will print information about the type that is inferred
for each metavariable.

The \NT{ctype} and \NT{ctypes} nonterminals are used by both the grammar of
metavariable declarations and the grammar of transformations, and are
defined on page~\pageref{types}.

An identifier metavariable with {\tt virtual} as its ``rule name'' is given
a value on the command line.  For example, if a semantic patch contains a
rule that declares an identifier metavariable with the name {\tt
  virtual.alloc}, then the command line could contain {\tt -D
  alloc=kmalloc}.  There should not be space around the {\tt =}.  An
example is in {\tt demos/vm.cocci} and {\tt demos/vm.c}.

It is possible to give an identifier metavariable a list of constraints
that it should or should not be equal to.  If the constraint is a list of
(unquoted) strings, then the value of the metavariable should be the same
as one of the strings, in the case of an equality constraint, or different
from all of the strings, in the case of an inequality constraint.  It is
also possible to include inherited identifier metavariables among the
constraints.  In the case of a positive constraint, things work in the same
way, but not with respect to the inherited value of the metavariable.  On
the other hand, an inequality constraint does not work so well, because the
only value available is the one available in the current environment.  If
the proposed value is different from the one in the current environment,
but perhaps the same as the one in some other environment, the match will
still succeed.

Position metavariables can be associated with constraints implemented as
OCaml script code.  The code must have the form of a single C expression,
typically a function call with a tuple of arguments.  This expression must
have type bool.  The script code can be parameterized by any inherited
metavariables.  It is implicitly parameterized by the metavariable being
declared.  In the script, the inherited variable parameters are referred to
by their variable names, without the associated rule name.  The variable
being declared is also referenced by its name.  All parameters, except
position variables, have their string representation.  An example is in
demos/poscon.cocci.

A declaration of a name as a typedef extends through the rest of the
semantic patch.  It is not required, but it will not cause a parse error,
to redeclare a name as a typedef.  A name declared as a typedef can,
however, be redeclared as another metavariable.  It will be considered to
be a metavariable in such rules, and will revert to being a typedef in
subsequent rules.

\paragraph*{Warning:} Each metavariable declaration causes the declared
metavariables to be immediately usable, without any inheritance
indication.  Thus the following are correct:

\begin{quote}
\begin{verbatim}
@@
type r.T;
T x;
@@

[...] // some semantic patch code
\end{verbatim}
\end{quote}

\begin{quote}
\begin{verbatim}
@@
r.T x;
type r.T;
@@

[...] // some semantic patch code
\end{verbatim}
\end{quote}

\noindent
But the following is not correct:

\begin{quote}
\begin{verbatim}
@@
type r.T;
r.T x;
@@

[...] // some semantic patch code
\end{verbatim}
\end{quote}

This applies to position variables, type metavariables, identifier
metavariables that may be used in specifying a structure type, and
metavariables used in the initialization of a fresh identifier.  In the
case of a structure type, any identifier metavariable indeed has to be
declared as an identifier metavariable in advance.  The syntax does not
permit {\tt r.n} as the name of a structure or union type in such a
declaration.

\section{Metavariables for scripts}

Metavariables for scripts can only be inherited from transformation rules.
In the spirit of scripting languages such as Python that use dynamic
typing, metavariables for scripts do not include type declarations.

\begin{grammar}
  \RULE{\rt{script\_metavariables}}
  \CASE{@ script:\NT{language} \OPT{\NT{rulename}} \OPT{depends on \NT{dep}} @
        \any{\NT{script\_metadecl}} @@}
  \CASE{@ initialize:\NT{language} \OPT{depends on \NT{dep}} @
        \any{\NT{script\_virt\_metadecl}} @@}
  \CASE{@ finalize:\NT{language} \OPT{depends on \NT{dep}} @
        \any{\NT{script\_virt\_metadecl}} @@}

  \RULE{\rt{language}} \CASE{python} \CASE{ocaml}

  \RULE{\rt{script\_metadecl}}
  \CASE{\T{id} <{}< \T{rulename\_id}.\T{id} ;}
  \CASE{\T{id} <{}< \T{rulename\_id}.\T{id} = "..." ;}
  \CASE{\T{id} <{}< \T{rulename\_id}.\T{id} = [] ;}
  \CASE{\T{id} ;}

  \RULE{\rt{script\_virt\_metadecl}}
  \CASE{\T{id} <{}< virtual.\T{id} ;}
\end{grammar}

Currently, the only scripting languages that are supported are Python and
OCaml, indicated using {\tt python} and {\tt ocaml}, respectively.  The
set of available scripting languages may be extended at some point.

Script rules declared with \KW{initialize} are run before the treatment of
any file.  Script rules declared with \KW{finalize} are run when the
treatment of all of the files has completed.  There can be at most one of
each per scripting language (thus currently at most one of each).
Initialize and finalize script rules do not have access to SmPL
metavariables.  Nevertheless, a finalize script rule can access any
variables initialized by the other script rules, allowing information to be
transmitted from the matching process to the finalize rule.

Initialize and finalize rules do have access to virtual metavariables,
using the usual syntax.  As for other scripting language rules, the rule
is not run (and essentially does not exist) if some of the required virtual
metavariables are not bound.  In ocaml, a warning is printed in this case.
An example is found in demos/initvirt.cocci.

A script metavariable that does not specify an origin, using \texttt{<<},
is newly declared by the script.  This metavariable should be assigned to a
string and can be inherited by subsequent rules as an identifier.  In
Python, the assignment of such a metavariable $x$ should refer to the
metavariable as {\tt coccinelle.\(x\)}.  Examples are in the files
\texttt{demos/pythontococci.cocci} and \texttt{demos/camltococci.cocci}.

In an OCaml script, the following extended form of \textit{script\_metadecl}
may be used:

\begin{grammar}
  \RULE{\rt{script\_metadecl'}}
  \CASE{(\T{id},\T{id}) <{}< \T{rulename\_id}.\T{id} ;}
  \CASE{\T{id} <{}< \T{rulename\_id}.\T{id} ;}
  \CASE{\T{id} ;}
\end{grammar}

\noindent
In a declaration of the form \texttt{(\T{id},\T{id}) <{}<
  \T{rulename\_id}.\T{id} ;}, the left component of \texttt{(\T{id},\T{id})}
receives a string representation of the value of the inherited metavariable
while the right component receives its abstract syntax tree.  The file
\texttt{parsing\_c/ast\_c.ml} in the Coccinelle implementation gives some
information about the structure of the abstract syntax tree.  Either the
left or right component may be replaced by \verb+_+, indicating that the
string representation or abstract syntax trees representation is not
wanted, respectively.

The abstract syntax tree of a metavariable declared using {\tt
  metavariable} is not available.

Script metavariables can have default values.  This is only allowed if the
abstract syntax tree of the metavariable is not requested.  The default
value of a position metavariable is written as {\tt []}.  The default value
of any other kind of metavariable is a string.  There is no control that
the string actually represents the kind of term represented by the
metavariable.  Normally, a script rule is only applied if all of the
metavariables have values.  If default values are provided, then the script
rule is only applied if all of the metavariables for which there are no
default values have values.  See demos/defaultscript.cocci for examples of
the use of this feature.

\section{Control Flow}

Rules describe a property that Coccinelle must match, and when the
property described is matched the rule is considered successful. One aspect
that is taken into account in determining a match is the program control
flow. A control flow describes a possible run time path taken by a program.

\subsection{Basic dots}
When using Coccinelle, it is possible to express matches of certain code
within certain types of control flows. Ellipses (``...'') can be used to
indicate to Coccinelle that anything can be present between consecutive
statements. For instance the following SmPL patch tells Coccinelle that
rule r0 wishes to remove all calls to function c().

\begin{center}
\begin{tabular}{c}
\begin{lstlisting}[language=Cocci]
@r0@
@@

-c();
\end{lstlisting}\\
\end{tabular}
\end{center}

The context of the rule provides no other guidelines to Coccinelle
about any possible control flow other than this is a statement, and that
c() must be called. We can modify the required control flow required for
this rule by providing additional requirements and using ellipses in between.
For instance, if we only wanted to remove calls to c() that also
had a prior call to foo() we'd use the following SmPL patch:

\begin{center}
\begin{tabular}{c}
\begin{lstlisting}[language=Cocci]
@r1@
@@

foo()
...
-c();
\end{lstlisting}\\
\end{tabular}
\end{center}

\subsection{Dot variants}
There are two possible modifiers to the control flow for ellipses, one
(<... ...>) indicates that matching the pattern in between the ellipses is
optional, and another (<+... ...+>) indicates that the pattern in between
the ellipses must be matched at least once, on some control-flow path.  In
the latter, the \texttt{+} is intended to be reminiscent of the \texttt{+}
used in regular expressions.  For instance, the following SmPL patch tells
Coccinelle to remove all calls to c(). if foo() is present at least
once since the beginning of the function.

\begin{center}
\begin{tabular}{c}
\begin{lstlisting}[language=Cocci]
@r2@
@@

<+...
foo()
...+>
-c();

\end{lstlisting}\\
\end{tabular}
\end{center}

Alternatively, the following inidcates that foo() is allowed but optional.
This case is typically most useful when all occurrences, if any, of foo()
prior to c() should be transformed.

\begin{center}
\begin{tabular}{c}
\begin{lstlisting}[language=Cocci]
@r3@
@@

<...
foo()
...>
-c();

\end{lstlisting}\\
\end{tabular}
\end{center}

\subsection{An example}
Let's consider some sample code to review: flow1.c.

\begin{center}
\begin{tabular}{c}
\begin{lstlisting}[language=C]

int main(void)
{
	int ret, a = 2;

	a = foo(a);
	ret = bar(a);
	c();

	return ret;
}
\end{lstlisting}\\
\end{tabular}
\end{center}

Applying the SmPL rule r0 to flow1.c would remove the c() line as the control
flow provides no specific context requirements. Applying rule r1 would also
succeed as the call to foo() is present. Likewise rules r2 and r3 would also
succeed. If the foo() call is removed from flow1.c only rules r0 and r3 would
succeed, as foo() would not be present and only rules r0 and r3 allow for
foo() to not be present.

One way to describe code control flow is in terms of McCabe cyclomatic
complexity.
The program flow1.1 has a linear control flow, it has no branches. The main
routine has a McCabe cyclomatic complexity of 1. The McCabe cyclomatic
complexity can be computed using
{\tt pmccabe} (https://www.gnu.org/\-software/\-complexity/\-manual/\-html\_node/\-pmccabe-parsing.html).

\begin{center}
\begin{tabular}{c}
\begin{lstlisting}[language=C]
pmccabe /flow1.c
1       1       5       1       10      flow1.c(1): main
\end{lstlisting}\\
\end{tabular}
\end{center}

Since programs can use branches, often times you may also wish to annotate
requirements for control flows in consideration for branches, for when
the McCabe cyclomatic complexity is > 1. The following program, flow2.c,
enables the control flow to diverge on line 7 due to the branch, if (a) --
one control flow possible is if (a) is true, another when if (a) is false.

\begin{center}
\begin{tabular}{c}
\begin{lstlisting}[language=C]
int main(void)
{
	int ret, a = 2;

	a = foo(a);
	ret = bar(a);
	if (a)
		c();

	return ret;
}
\end{lstlisting}\\
\end{tabular}
\end{center}

This program has a McCabe cyclomatic complexity of 2.

\begin{center}
\begin{tabular}{c}
\begin{lstlisting}[language=C]
pmccabe flow2.c
2       2       6       1       11      flow2.c(1): main
\end{lstlisting}\\
\end{tabular}
\end{center}

Using the McCabe cyclomatic complexity is one way to get an idea of
the complexity of the control graph for a function, another way is
to visualize all possible paths. Coccinelle provides a way to visualize
control flows of programs, this however requires {\tt dot}
(http://www.graphviz.org/) and {\tt gv} to be installed (typically provided
by a package called graphviz). To visualize control flow or a program
using Coccinelle you use:

\begin{center}
\begin{tabular}{c}
spatch --control-flow-to-file flow1.c
spatch --control-flow-to-file flow2.c
\end{tabular}
\end{center}

%Below are the two generated control flow graphs for flow1.c and flow2.c
%respectively.

%\begin{figure}
%	\[\includegraphics[width=\linewidth]{flow1.pdf}\]
%	\caption{Linear flow example}
%	\label{devmodel}
%\end{figure}

%\begin{figure}
%	\[\includegraphics[width=\linewidth]{flow2.pdf}\]
%	\caption{Linear flow example}
%	\label{devmodel}
%\end{figure}

Behind the scenes this generates a dot file and uses gv to generate
a PDF file for viewing. To generate and inspect these manually you
can use the following:

\begin{center}
\begin{tabular}{c}
spatch --control-flow-to-file flow2.c
dot -Tpdf flow1:main.dot > flow1.pdf
\end{tabular}
\end{center}

By default properties described in a rule must match all control
flows possible within a code section being inspected by Coccinelle.
So for instance, in the following SmPL patch rule r1 would match all
the control flow possible on flow1.c as its linear, however it would
not match the control possible on flow2.c. The rule r1 would not
be successful in flow2.c

\begin{center}
\begin{tabular}{c}
\begin{lstlisting}[language=Cocci]
@r1@
@@

foo()
...
-c();

\end{lstlisting}\\
\end{tabular}
\end{center}

The default control flow can be modified by using the keyword ``exists''
following the rule name. In the following SmPL patch the rule r2 would
be successful on both flow1.c and flow2.c

\begin{center}
\begin{tabular}{c}
\begin{lstlisting}[language=Cocci]
@r2 exists@
@@

foo()
...
-c();

\end{lstlisting}\\
\end{tabular}
\end{center}

If the rule name is followed by the ``forall'' keyword, then all control flow
paths must match in order for the rule to succeed. By default when a
semantic patch has ``-'' and ``+'', or when it has no annotations at all and
only script code, ellipses (``...'') use the forall semantics.  And when the
semantic patch uses the context annotation (``*''), the ellipses (``...'') uses
the exists semantics.  Using the keyword ``forall'' or ``exists'' in the rule
header affects all ellipses (``...'') uses in the rule. You can also annotate
each ellipses (``...'') with ``when exists'' or ``when forall'' individually.

Rules can also be not be successful if requirements do not match
when a rule name is followed by ``depends on XXX''. When ``depends on'' is used
it means the rule should only apply if rule XXX matched with the current
metavariable environment. Alternatively, ``depends on ever XXX'' can be used
as well, this means this rule should apply if rule XXX was ever matched at
all. A counter to this use is ``depends on never XXX'', which means that this
rule should apply if rule XXX was never matched at all.

\section{Transformation}

Coccinelle semantic patches are able to transform C code.

\subsection{Basic transformations}

The transformation specification essentially has the form of C code, except
that lines to remove are annotated with \verb+-+ in the first column, and
lines to add are annotated with \verb-+-.  A transformation specification
can also use {\em dots}, ``\verb-...-'', describing an arbitrary sequence
of function arguments or instructions within a control-flow path.
Implicitly, ``\verb-...-'' matches the shortest path between something that
matches the pattern before the dots (or the beginning of the function, if
there is nothing before the dots) and something that matches the pattern
after the dots (or the end of the function, if there is nothing after the
dots).  Dots may be modified with a {\tt when} clause, indicating a pattern
that should not occur anywhere within the matched sequence.  {\tt when any}
removes the aforementioned constraint that ``\verb-...-'' matches the
shortest path.  Finally, a transformation can specify a disjunction of
patterns, of the form \mtt{( \mth{\mita{pat}_1} | \mita{\ldots} |
  \mth{\mita{pat}_n} )} where each \texttt{(}, \texttt{|} or \texttt{)} is
in column 0 or preceded by \texttt{\textbackslash}.
Similarly, a transformation can specify a conjunction of
patterns, of the form \mtt{( \mth{\mita{pat}_1} \& \mita{\ldots} \&
  \mth{\mita{pat}_n} )} where each \texttt{(}, \texttt{\&} or \texttt{)} is
in column 0 or preceded by \texttt{\textbackslash}.  All of the patterns
must be matched at the same place in the control-flow graph.

The grammar that we present for the transformation is not actually the
grammar of the SmPL code that can be written by the programmer, but is
instead the grammar of the slice of this consisting of the {\tt -}
annotated and the unannotated code (the context of the transformed lines),
or the {\tt +} annotated code and the unannotated code.  For example, for
parsing purposes, the following transformation
%presented in Section \ref{sec:seq2}
is split into the two variants shown below and each is parsed
separately.

\begin{center}
\begin{tabular}{c}
\begin{lstlisting}[language=Cocci]
  proc_info_func(...) {
    <...
@--    hostno
@++    hostptr->host_no
    ...>
 }
\end{lstlisting}\\
\end{tabular}
\end{center}

{%\sizecodebis
\begin{center}
\begin{tabular}{p{5cm}p{3cm}p{5cm}}
\begin{lstlisting}[language=Cocci]
  proc_info_func(...) {
    <...
@--    hostno
    ...>
 }
\end{lstlisting}
&&
\begin{lstlisting}[language=Cocci]
  proc_info_func(...) {
    <...
@++    hostptr->host_no
    ...>
 }
\end{lstlisting}
\end{tabular}
\end{center}
}

\noindent
Requiring that both slices parse correctly ensures that the rule matches
syntactically valid C code and that it produces syntactically valid C code.
The generated parse trees are then merged for use in the subsequent
matching and transformation process.

The grammar for the minus or plus slice of a transformation is as follows:

\begin{grammar}

  \RULE{\rt{transformation}}
  \CASE{\some{\NT{include}}}
  \CASE{\NT{OPTDOTSEQ}\mth{(}\NT{top}, \NT{when}\mth{)}}

  \RULE{\rt{include}}
  \CASE{\#include \T{include\_string}}

  \RULE{\rt{top}}
  \CASE{\NT{expr}}
  \CASE{\some{\NT{decl\_stmt}}}
  \CASE{\NT{fundecl}}

%  \RULE{\rt{fun\_decl\_stmt}}
%  \CASE{\NT{decl\_stmt}}
%  \CASE{\NT{fundecl}}

%  \CASE{\NT{ctype}}
%  \CASE{\ttlb \NT{initialize\_list} \ttrb}
%  \CASE{\NT{toplevel\_seq\_start\_after\_dots\_init}}
%
%  \RULE{\rt{toplevel\_seq\_start\_after\_dots\_init}}
%  \CASE{\NT{stmt\_dots} \NT{toplevel\_after\_dots}}
%  \CASE{\NT{expr} \opt{\NT{toplevel\_after\_exp}}}
%  \CASE{\NT{decl\_stmt\_expr} \opt{\NT{toplevel\_after\_stmt}}}
%
%  \RULE{\rt{stmt\_dots}}
%  \CASE{... \any{\NT{when}}}
%  \CASE{<... \any{\NT{when}} \NT{nest\_after\_dots} ...>}
%  \CASE{<+... \any{\NT{when}} \NT{nest\_after\_dots} ...+>}

  \RULE{\rt{when}}
  \CASE{when != \NT{when\_code}}
  \CASE{when = \NT{rule\_elem\_stmt}}
  \CASE{when \NT{COMMA\_LIST}\mth{(}\NT{any\_strict}\mth{)}}
  \CASE{when true != \NT{expr}}
  \CASE{when false != \NT{expr}}

  \RULE{\rt{when\_code}}
  \CASE{\NT{OPTDOTSEQ}\mth{(}\some{\NT{decl\_stmt}}, \NT{when}\mth{)}}
  \CASE{\NT{OPTDOTSEQ}\mth{(}\NT{expr}, \NT{when}\mth{)}}

  \RULE{\rt{rule\_elem\_stmt}}
  \CASE{\NT{one\_decl}}
  \CASE{\NT{expr};}
  \CASE{return \opt{\NT{expr}};}
  \CASE{break;}
  \CASE{continue;}
  \CASE{\bs(\NT{rule\_elem\_stmt} \SOME{\bs| \NT{rule\_elem\_stmt}}\bs)}

  \RULE{\rt{any\_strict}}
  \CASE{any}
  \CASE{strict}
  \CASE{forall}
  \CASE{exists}

%  \RULE{\rt{nest\_after\_dots}}
%  \CASE{\NT{decl\_stmt\_exp} \opt{\NT{nest\_after\_stmt}}}
%  \CASE{\opt{\NT{exp}} \opt{\NT{nest\_after\_exp}}}
%
%  \RULE{\rt{nest\_after\_stmt}}
%  \CASE{\NT{stmt\_dots} \NT{nest\_after\_dots}}
%  \CASE{\NT{decl\_stmt} \opt{\NT{nest\_after\_stmt}}}
%
%  \RULE{\rt{nest\_after\_exp}}
%  \CASE{\NT{stmt\_dots} \NT{nest\_after\_dots}}
%
%  \RULE{\rt{toplevel\_after\_dots}}
%  \CASE{\opt{\NT{toplevel\_after\_exp}}}
%  \CASE{\NT{exp} \opt{\NT{toplevel\_after\_exp}}}
%  \CASE{\NT{decl\_stmt\_expr} \NT{toplevel\_after\_stmt}}
%
%  \RULE{\rt{toplevel\_after\_exp}}
%  \CASE{\NT{stmt\_dots} \opt{\NT{toplevel\_after\_dots}}}
%
%  \RULE{\rt{decl\_stmt\_expr}}
%  \CASE{TMetaStmList$^\ddag$}
%  \CASE{\NT{decl\_var}}
%  \CASE{\NT{stmt}}
%  \CASE{(\NT{stmt\_seq} \ANY{| \NT{stmt\_seq}})}
%
%  \RULE{\rt{toplevel\_after\_stmt}}
%  \CASE{\NT{stmt\_dots} \opt{\NT{toplevel\_after\_dots}}}
%  \CASE{\NT{decl\_stmt} \NT{toplevel\_after\_stmt}}

\end{grammar}

\begin{grammar}
  \RULE{\rt{OPTDOTSEQ}\mth{(}\rt{grammar\_ds}, \rt{when\_ds}\mth{)}}
  \CASE{}\multicolumn{3}{r}{\hspace{1cm}
  \KW{\opt{... \ANY{\NT{when\_ds}}} \NT{grammar\_ds}
    \ANY{... \ANY{\NT{when\_ds}} \NT{grammar\_ds}}
    \opt{... \ANY{\NT{when\_ds}}}}
  }

%  \CASE{\opt{... \opt{\NT{when\_ds}}} \NT{grammar}
%    \ANY{... \opt{\NT{when\_ds}} \NT{grammar}}
%    \opt{... \opt{\NT{when\_ds}}}}
%  \CASE{<... \any{\NT{when\_ds}} \NT{grammar} ...>}
%  \CASE{<+... \any{\NT{when\_ds}} \NT{grammar} ...+>}

\end{grammar}

\noindent
Lines may be annotated with an element of the set $\{\mtt{-}, \mtt{+},
\mtt{*}\}$ or the singleton $\mtt{?}$, or one of each set. \mtt{?}
represents at most one match of the given pattern, ie a match of the
pattern is optional. \mtt{*} is used for
semantic match, \emph{i.e.}, a pattern that highlights the fragments
annotated with \mtt{*}, but does not perform any modification of the
matched code. \mtt{*} cannot be mixed with \mtt{-} and \mtt{+}.  There are
some constraints on the use of these annotations:
\begin{itemize}
\item Dots, {\em i.e.} \texttt{...}, cannot occur on a line marked
  \texttt{+}.
\item Nested dots, {\em i.e.}, dots enclosed in {\tt <} and {\tt >}, cannot
  occur on a line marked \texttt{+}.
\end{itemize}

An \#include may be followed by \texttt{"..."}, \texttt{<...>} or simply
\texttt{...}.  With either quotes or angle brackets, it is possible to put
a partial path, ending with ..., such as \texttt{<include/...>}, or to put a
complete path.  A \#include with \texttt{...} matches any include, with
either quotes or angle brackets.  Partial paths or complete are not allowed
in the latter case.  Something that is added before an include will be put
before the last matching include that is not under an ifdef in the file.
Likewise, something that is added after an include will be put after the
last matching include that is not under an ifdef in the file.

Each element of a disjunction must be a proper term like an expression, a
statement, an identifier or a declaration. The constraint on a conjunction
is similiar.  Thus, the rule on the left below is not a syntactically
correct SmPL rule. One may use the rule on the right instead.

\begin{center}
  \begin{tabular}{l@{\hspace{5cm}}r}
\begin{lstlisting}[language=Cocci]
@@
type T;
T b;
@@

(
 writeb(...,
|
 readb(...,
)
@--(T)
 b)
\end{lstlisting}
    &
\begin{lstlisting}[language=Cocci]
@@
type T;
T b;
@@

(
read
|
write
)
 (...,
@-- (T)
  b)
\end{lstlisting}
    \\
  \end{tabular}
\end{center}

Some kinds of terms can only appear in + code.  These include comments,
ifdefs, and attributes (\texttt{\_\_attribute\_\_((...))}).

\subsection{Advanced transformations}

You may run into the situation where a semantic patch needs to add several
disjoint terms at the same place in the code.  Coccinelle does not know in
which order these terms should appear, and thus gives an ``already tagged
token'' error in this situation. If you are sure that order does not matter
you can use the optional double addition token \texttt{++} to indicate to
Coccinelle that it may add things in any order. This may be for instance
safe in situations such as extending a data structure with more members,
based on existing members of the data structure. The following rule helps
to extend a data structure with a respective float for a present int.  If
there is only one int field in the data structure, this semantic patch
works well with the simple \texttt{+}.

\begin{lstlisting}[language=Cocci]
@simpleplus@
identifier x,v;
fresh identifier xx = v ## "_float";
@@

struct x {
+	float xx;
	...
	int v;
	...
}
\end{lstlisting}

This semantic patch works fine, for example, on the following code
(plusplus1.c):

\begin{lstlisting}[language=C]
struct x {
	int z;
	char b;
};
\end{lstlisting}

If however there are multiple int fields tokens that Coccinelle
can transform, order cannot be guaranteed for how Coccinelle
makes additions. If you are sure order does not matter for
the transformation you may use \texttt{++} instead, as follows:

\begin{lstlisting}[language=Cocci]
@plusplus@
identifier x,v;
fresh identifier xx = v ## "_float";
@@

struct x {
++	float xx;
	...
	int v;
	...
}
\end{lstlisting}

This rule would work against a file plusplus2.c that has three
int fields:

\begin{lstlisting}[language=C]
struct x {
	int z;
	int a;
	char b;
	int c;
	int *d;
};
\end{lstlisting}

A possible result is as shown below. The precise order of the float fields
is however not guaranteed with respect to each other:

\begin{lstlisting}[language=C]
struct x {
	float a_float;
	float c_float;
	float z_float;
	int z;
	int a;
	char b;
	int c;
	int *d;
};
\end{lstlisting}

If you used simpleplus rule on plusplus2.c you would end up with
an ``already tagged token'' error due to the ordering considerations
explained in this section.

\section{Types}
\label{types}

\begin{grammar}

  \RULE{\rt{ctypes}}
  \CASE{\NT{COMMA\_LIST}\mth{(}\NT{ctype}\mth{)}}

  \RULE{\rt{ctype}}
  \CASE{\opt{\NT{const\_vol}} \NT{generic\_ctype} \any{*}}
  \CASE{\opt{\NT{const\_vol}} void \some{*}}
  \CASE{(\NT{ctype} \ANY{| \NT{ctype}})}

  \RULE{\rt{const\_vol}}
  \CASE{const}
  \CASE{volatile}

  \RULE{\rt{generic\_ctype}}
  \CASE{\NT{ctype\_qualif}}
  \CASE{\opt{\NT{ctype\_qualif}} char}
  \CASE{\opt{\NT{ctype\_qualif}} short}
  \CASE{\opt{\NT{ctype\_qualif}} short int}
  \CASE{\opt{\NT{ctype\_qualif}} int}
  \CASE{\opt{\NT{ctype\_qualif}} long}
  \CASE{\opt{\NT{ctype\_qualif}} long int}
  \CASE{\opt{\NT{ctype\_qualif}} long long}
  \CASE{\opt{\NT{ctype\_qualif}} long long int}
  \CASE{double}
  \CASE{long double}
  \CASE{float}
  \CASE{size\_t} \CASE{ssize\_t} \CASE{ptrdiff\_t}
  \CASE{enum \NT{id} \{ \NT{PARAMSEQ}\mth{(}\NT{dot\_expr}, \NT{exp\_whencode}\mth{)} \OPT{,} \}}
  \CASE{\OPT{struct\OR union} \T{id} \OPT{\{ \any{\NT{struct\_decl\_list}} \}}}

  \RULE{\rt{ctype\_qualif}}
  \CASE{unsigned}
  \CASE{signed}

  \RULE{\rt{struct\_decl\_list}}
  \CASE{\NT{struct\_decl\_list\_start}}

  \RULE{\rt{struct\_decl\_list\_start}}
  \CASE{\NT{struct\_decl}}
  \CASE{\NT{struct\_decl} \NT{struct\_decl\_list\_start}}
  \CASE{... \opt{when != \NT{struct\_decl}}$^\dag$ \opt{\NT{continue\_struct\_decl\_list}}}

  \RULE{\rt{continue\_struct\_decl\_list}}
  \CASE{\NT{struct\_decl} \NT{struct\_decl\_list\_start}}
  \CASE{\NT{struct\_decl}}

  \RULE{\rt{struct\_decl}}
  \CASE{\NT{ctype} \NT{d\_ident};}
  \CASE{\NT{fn\_ctype} (* \NT{d\_ident}) (\NT{PARAMSEQ}\mth{(}\NT{name\_opt\_decl}, \mth{\varepsilon)});)}
  \CASE{\opt{\NT{const\_vol}} \T{id} \NT{d\_ident};}

  \RULE{\rt{d\_ident}}
  \CASE{\T{id} \any{[\opt{\NT{expr}}]}}

  \RULE{\rt{fn\_ctype}}
  \CASE{\NT{generic\_ctype} \any{*}}
  \CASE{void \any{*}}

  \RULE{\rt{name\_opt\_decl}}
  \CASE{\NT{decl}}
  \CASE{\NT{ctype}}
  \CASE{\NT{fn\_ctype}}
\end{grammar}

$^\dag$ The optional \texttt{when} construct ends at the end of the line.

\section{Function declarations}

\begin{grammar}

  \RULE{\rt{fundecl}}
  \CASE{\opt{\NT{fn\_ctype}} \any{\NT{funinfo}} \NT{funid}
    (\opt{\NT{PARAMSEQ}\mth{(}\NT{param}, \mth{\varepsilon)}})
    \ttlb~\opt{\NT{stmt\_seq}} \ttrb}

  \RULE{\rt{funproto}}
  \CASE{\NT{fn\_ctype} \any{\NT{funinfo}} \NT{funid}
    (\opt{\NT{PARAMSEQ}\mth{(}\NT{param}, \mth{\varepsilon)}});}

  \RULE{\rt{funinfo}}
  \CASE{inline}
  \CASE{\NT{storage}}
%   \CASE{\NT{attr}}

  \RULE{\rt{storage}}
  \CASE{static}
  \CASE{auto}
  \CASE{register}
  \CASE{extern}

  \RULE{\rt{funid}}
  \CASE{\T{id}}
  \CASE{\mth{\T{metaid}^{\ssf{Id}}}}
  \CASE{\NT{OR}\mth{(}\NT{stmt}\mth{)}}
%   \CASE{\mth{\T{metaid}^{\ssf{Func}}}}
%   \CASE{\mth{\T{metaid}^{\ssf{LocalFunc}}}}

  \RULE{\rt{param}}
  \CASE{\NT{type} \T{id}}
  \CASE{\mth{\T{metaid}^{\ssf{Param}}}}
  \CASE{\mth{\T{metaid}^{\ssf{ParamList}}}}
  \CASE{......}

  \RULE{\rt{decl}}
  \CASE{\NT{ctype} \NT{id}}
  \CASE{\NT{fn\_ctype} (* \NT{id}) (\NT{PARAMSEQ}\mth{(}\NT{name\_opt\_decl}, \mth{\varepsilon)})}
  \CASE{void}
  \CASE{\mth{\T{metaid}^{\ssf{Param}}}}
\end{grammar}

\begin{grammar}
  \RULE{\rt{PARAMSEQ}\mth{(}\rt{gram\_p}, \rt{when\_p}\mth{)}}
  \CASE{\NT{COMMA\_LIST}\mth{(}\NT{gram\_p} \OR \ldots \opt{\NT{when\_p}}\mth{)}}
\end{grammar}

To match a function it is not necessary to provide all of the annotations
that appear before the function name.  For example, the following semantic
patch:

\begin{lstlisting}[language=Cocci]
@@
@@

foo() { ... }
\end{lstlisting}

\noindent
matches a function declared as follows:

\begin{lstlisting}[language=C]
static int foo() { return 12; }
\end{lstlisting}

\noindent
This behavior can be turned off by disabling the \KW{optional\_storage}
isomorphism.  If one adds code before a function declaration, then the
effect depends on the kind of code that is added.  If the added code is a
function definition or CPP code, then the new code is placed before
all information associated with the function definition, including any
comments preceding the function definition.  On the other hand, if the new
code is associated with the function, such as the addition of the keyword
{\tt static}, the new code is placed exactly where it appears with respect
to the rest of the function definition in the semantic patch.  For example,

\begin{lstlisting}[language=Cocci]
@@
@@

+ static
foo() { ... }
\end{lstlisting}

\noindent
causes static to be placed just before the function name.  The following
causes it to be placed just before the type

\begin{lstlisting}[language=Cocci]
@@
type T;
@@

+ static
T foo() { ... }
\end{lstlisting}

\noindent
It may be necessary to consider several cases to ensure that the added code
is placed in the right position.  For example, one may need one pattern
that considers that the function is declared {\tt inline} and another that
considers that it is not.

Varargs are written in C using {\tt \ldots}.  Unfortunately, this notation
is already used in the semantic patch language.  A pattern for a varargs
parameter is written as a sequence of 6 dots.

The C parser allows functions that have no return type, and assumes that
the return type is \texttt{int}.  The support for parsing such functions is
limited.  In particular, the parameter list must contain a type for each
parameter, and may not contain varargs.

For a function prototype, unlike a function definition, a specification of
the return type is obligatory.

%\newpage

\section{Declarations}

\begin{grammar}
  \RULE{\rt{decl\_var}}
  \CASE{\NT{common\_decl}}
  \CASE{\opt{\NT{storage}} \NT{ctype} \NT{COMMA\_LIST}\mth{(}\NT{d\_ident}\mth{)} ;}
  \CASE{\opt{\NT{storage}} \opt{\NT{const\_vol}} \T{id} \NT{COMMA\_LIST}\mth{(}\NT{d\_ident}\mth{)} ;}
  \CASE{\opt{\NT{storage}} \NT{fn\_ctype} ( * \NT{d\_ident} ) ( \NT{PARAMSEQ}\mth{(}\NT{name\_opt\_decl}, \mth{\varepsilon)} ) = \NT{initialize} ;}
  \CASE{typedef \NT{ctype} \NT{typedef\_ident} ;}
%  \CASE{\NT{type} \opt{\NT{id} \opt{[\opt{\NT{dot\_expr}}]}
%      \ANY{, \NT{id} \opt{[ \opt{\NT{dot\_expr}}]}}};}

  \RULE{\rt{one\_decl}}
  \CASE{\NT{common\_decl}}
  \CASE{\opt{\NT{storage}} \NT{ctype} \NT{id};}
%  \CASE{\NT{storage} \NT{ctype} \NT{id} \opt{[\opt{\NT{dot\\_expr}}]} = \NT{nest\\_expr};}
  \CASE{\opt{\NT{storage}} \opt{\NT{const\_vol}} \T{id} \NT{d\_ident} ;}

  \RULE{\rt{common\_decl}}
  \CASE{\NT{ctype};}
  \CASE{\NT{funproto}}
  \CASE{\opt{\NT{storage}} \NT{ctype} \NT{d\_ident} = \NT{initialize} ;}
  \CASE{\opt{\NT{storage}} \opt{\NT{const\_vol}} \T{id} \NT{d\_ident} = \NT{initialize} ;}
  \CASE{\opt{\NT{storage}} \NT{fn\_ctype} ( * \NT{d\_ident} ) ( \NT{PARAMSEQ}\mth{(}\NT{name\_opt\_decl}, \mth{\varepsilon)} ) ;}
  \CASE{\NT{decl\_ident} ( \OPT{\NT{COMMA\_LIST}\mth{(}\NT{expr}\mth{)}} ) ;}

  \RULE{\rt{initialize}}
  \CASE{\NT{dot\_expr}}
  \CASE{\mth{\T{metaid}^{\ssf{Initialiser}}}}
  \CASE{\ttlb~\opt{\NT{COMMA\_LIST}\mth{(}\NT{init\_list\_elem}\mth{)}}~\ttrb}

  \RULE{\rt{init\_list\_elem}}
  \CASE{\NT{dot\_expr}}
  \CASE{\NT{designator} = \NT{initialize}}
  \CASE{\mth{\T{metaid}^{\ssf{Initialiser}}}}
  \CASE{\mth{\T{metaid}^{\ssf{InitialiserList}}}}
  \CASE{\NT{id} : \NT{dot\_expr}}

  \RULE{\rt{designator}}
  \CASE{. \NT{id}}
  \CASE{[ \NT{dot\_expr} ]}
  \CASE{[ \NT{dot\_expr} ... \NT{dot\_expr} ]}

  \RULE{\rt{decl\_ident}}
  \CASE{\T{DeclarerId}}
  \CASE{\mth{\T{metaid}^{\ssf{Declarer}}}}
\end{grammar}

An initializer for a structure can be ordered or unordered.  It is
considered to be unordered if there is at least one key-value pair
initializer, e.g., \texttt{.x = e}.

A declaration can have \textit{e.g.} the form \texttt{register x;}.  In
this case, the variable implicitly has type int, and SmPL code
that declares an int variable will match such a declaration.  On the other
hand, the implicit int type has no position.  If the SmPL code tries to
record the position of the type, the match will fail.

\section{Statements}

The first rule {\em statement} describes the various forms of a statement.
The remaining rules implement the constraints that are sensitive to the
context in which the statement occurs: {\em single\_statement} for a
context in which only one statement is allowed, and {\em decl\_statement}
for a context in which a declaration, statement, or sequence thereof is
allowed.

\begin{grammar}
  \RULE{\rt{stmt}}
  \CASE{\NT{directive}}
  \CASE{\mth{\T{metaid}^{\ssf{Stmt}}}}
  \CASE{\NT{expr};}
  \CASE{if (\NT{dot\_expr}) \NT{single\_stmt} \opt{else \NT{single\_stmt}}}
  \CASE{for (\opt{\NT{dot\_expr}}; \opt{\NT{dot\_expr}}; \opt{\NT{dot\_expr}})
    \NT{single\_stmt}}
  \CASE{while (\NT{dot\_expr}) \NT{single\_stmt}}
  \CASE{do \NT{single\_stmt} while (\NT{dot\_expr});}
  \CASE{\NT{iter\_ident} (\any{\NT{dot\_expr}}) \NT{single\_stmt}}
  \CASE{switch (\opt{\NT{dot\_expr}}) \ttlb \any{\NT{case\_line}} \ttrb}
  \CASE{return \opt{\NT{dot\_expr}};}
  \CASE{\ttlb~\opt{\NT{stmt\_seq}} \ttrb}
  \CASE{\NT{NEST}\mth{(}\some{\NT{decl\_stmt}}, \NT{when}\mth{)}}
  \CASE{\NT{NEST}\mth{(}\NT{expr}, \NT{when}\mth{)}}
  \CASE{break;}
  \CASE{continue;}
  \CASE{\NT{id}:}
  \CASE{goto \NT{id};}
  \CASE{\ttlb \NT{stmt\_seq} \ttrb}

  \RULE{\rt{directive}}
  \CASE{\NT{include}}
  \CASE{\#define \NT{id} \opt{\NT{top}}}
  \CASE{\#define \NT{id} (\NT{PARAMSEQ}\mth{(}\NT{id}, \mth{\varepsilon)})
        \opt{\NT{top}}}
  \CASE{\#undef \NT{id}}
  \CASE{\#pragma \NT{id} \some{\NT{id}}}
  \CASE{\#pragma \NT{id} (\NT{PARAMSEQ}\mth{(}\NT{expr}, \mth{\varepsilon)})}
  \CASE{\#pragma \NT{id} ...}

  \RULE{\rt{single\_stmt}}
  \CASE{\NT{stmt}}
  \CASE{\NT{OR}\mth{(}\NT{stmt}\mth{)}}

  \RULE{\rt{decl\_stmt}}
  \CASE{\mth{\T{metaid}^{\ssf{StmtList}}}}
  \CASE{\NT{decl\_var}}
  \CASE{\NT{stmt}}
  \CASE{\NT{OR}\mth{(}\NT{stmt\_seq}\mth{)}}

  \RULE{\rt{stmt\_seq}}
  \CASE{\any{\NT{decl\_stmt}}
    \opt{\NT{DOTSEQ}\mth{(}\some{\NT{decl\_stmt}},
      \NT{when}\mth{)} \any{\NT{decl\_stmt}}}}
  \CASE{\any{\NT{decl\_stmt}}
    \opt{\NT{DOTSEQ}\mth{(}\NT{expr},
      \NT{when}\mth{)} \any{\NT{decl\_stmt}}}}

  \RULE{\rt{case\_line}}
  \CASE{default :~\NT{stmt\_seq}}
  \CASE{case \NT{dot\_expr} :~\NT{stmt\_seq}}

  \RULE{\rt{iter\_ident}}
  \CASE{\T{IteratorId}}
  \CASE{\mth{\T{metaid}^{\ssf{Iterator}}}}
\end{grammar}

\begin{grammar}
  \RULE{\rt{OR}\mth{(}\rt{gram\_o}\mth{)}}
  \CASE{( \NT{gram\_o} \ANY{\ttmid \NT{gram\_o}})}

  \RULE{\rt{DOTSEQ}\mth{(}\rt{gram\_d}, \rt{when\_d}\mth{)}}
  \CASE{\ldots \opt{\NT{when\_d}} \ANY{\NT{gram\_d} \ldots \opt{\NT{when\_d}}}}

  \RULE{\rt{NEST}\mth{(}\rt{gram\_n}, \rt{when\_n}\mth{)}}
  \CASE{<\ldots \opt{\NT{when\_n}} \NT{gram\_n} \ANY{\ldots \opt{\NT{when\_n}} \NT{gram\_n}} \ldots>}
  \CASE{<+\ldots \opt{\NT{when\_n}} \NT{gram\_n} \ANY{\ldots \opt{\NT{when\_n}} \NT{gram\_n}} \ldots+>}
\end{grammar}

\noindent
OR is a macro that generates a disjunction of patterns.  The three
tokens \T{(}, \T{\ttmid}, and \T{)} must appear in the leftmost
column, to differentiate them from the parentheses and bit-or tokens
that can appear within expressions (and cannot appear in the leftmost
column). These token may also be preceded by \texttt{\bs}
when they are used in an other column.  These tokens are furthermore
different from (, \(\mid\), and ), which are part of the grammar
metalanguage.

\section{Expressions}

A nest or a single ellipsis is allowed in some expression contexts, and
causes ambiguity in others.  For example, in a sequence \mtt{\ldots
\mita{expr} \ldots}, the nonterminal \mita{expr} must be instantiated as an
explicit C-language expression, while in an array reference,
\mtt{\mth{\mita{expr}_1} \mtt{[} \mth{\mita{expr}_2} \mtt{]}}, the
nonterminal \mth{\mita{expr}_2}, because it is delimited by brackets, can
be also instantiated as \mtt{\ldots}, representing an arbitrary expression.  To
distinguish between the various possibilities, we define three nonterminals
for expressions: {\em expr} does not allow either top-level nests or
ellipses, {\em nest\_expr} allows a nest but not an ellipsis, and {\em
dot\_expr} allows both.  The EXPR macro is used to express these variants
in a concise way.

\begin{grammar}
  \RULE{\rt{expr}}
  \CASE{\NT{EXPR}\mth{(}\NT{expr}\mth{)}}

  \RULE{\rt{nest\_expr}}
  \CASE{\NT{EXPR}\mth{(}\NT{nest\_expr}\mth{)}}
  \CASE{\NT{NEST}\mth{(}\NT{nest\_expr}, \NT{exp\_whencode}\mth{)}}

  \RULE{\rt{dot\_expr}}
  \CASE{\NT{EXPR}\mth{(}\NT{dot\_expr}\mth{)}}
  \CASE{\NT{NEST}\mth{(}\NT{dot\_expr}, \NT{exp\_whencode}\mth{)}}
  \CASE{...~\opt{\NT{exp\_whencode}}}

  \RULE{\rt{EXPR}\mth{(}\rt{exp}\mth{)}}
  \CASE{\NT{exp} \NT{assign\_op} \NT{exp}}
  \CASE{\NT{exp} \mth{\T{metaid}^{\ssf{AssignOp}}} \NT{exp}}
  \CASE{\NT{exp}++}
  \CASE{\NT{exp}--}
  \CASE{\NT{unary\_op} \NT{exp}}
  \CASE{\NT{exp} \NT{bin\_op} \NT{exp}}
  \CASE{\NT{exp} \mth{\T{metaid}^{\ssf{BinOp}}} \NT{exp}}
  \CASE{\NT{exp} ?~\NT{dot\_expr} :~\NT{exp}}
  \CASE{(\NT{type}) \NT{exp}}
  \CASE{\NT{exp} [\NT{dot\_expr}]}
  \CASE{\NT{exp} .~\NT{id}}
  \CASE{\NT{exp} -> \NT{id}}
  \CASE{\NT{exp}(\opt{\NT{PARAMSEQ}\mth{(}\NT{arg}, \NT{exp\_whencode}\mth{)}})}
  \CASE{\NT{id}}
  \CASE{(\NT{type}) \ttlb~{\NT{COMMA\_LIST}\mth{(}\NT{init\_list\_elem}\mth{)}}~\ttrb}
%   \CASE{\mth{\T{metaid}^{\ssf{Func}}}}
%   \CASE{\mth{\T{metaid}^{\ssf{LocalFunc}}}}
  \CASE{\mth{\T{metaid}^{\ssf{Exp}}}}
%   \CASE{\mth{\T{metaid}^{\ssf{Err}}}}
  \CASE{\mth{\T{metaid}^{\ssf{Const}}}}
  \CASE{\NT{const}}
  \CASE{(\NT{dot\_expr})}
  \CASE{\NT{OR}\mth{(}\NT{exp}\mth{)}}

  \RULE{\rt{arg}}
  \CASE{\NT{nest\_expr}}
  \CASE{\mth{\T{metaid}^{\ssf{ExpList}}}}

  \RULE{\rt{exp\_whencode}}
  \CASE{when != \NT{expr}}

  \RULE{\rt{assign\_op}}
  \CASE{= \OR -= \OR += \OR *= \OR /= \OR \%=}
  \CASE{\&= \OR |= \OR \caret= \OR \lt\lt= \OR \gt\gt=}

  \RULE{\rt{bin\_op}}
  \CASE{* \OR / \OR \% \OR + \OR -}
  \CASE{\lt\lt \OR \gt\gt \OR \caret\xspace \OR \& \OR \ttmid}
  \CASE{< \OR > \OR <= \OR >= \OR == \OR != \OR \&\& \OR \ttmid\ttmid}

  \RULE{\rt{unary\_op}}
  \CASE{++ \OR -- \OR \& \OR * \OR + \OR - \OR !}

\end{grammar}

\section{Constants, Identifiers and Types for Transformations}

\begin{grammar}
  \RULE{\rt{const}}
  \CASE{\NT{string}}
  \CASE{[0-9]+}
  \CASE{\mth{\cdots}}

  \RULE{\rt{string}}
  \CASE{"\any{[\^{}"]}"}

  \RULE{\rt{id}}
  \CASE{\T{id} \OR \mth{\T{metaid}^{\ssf{Id}}}
        \OR {\NT{OR}\mth{(}\NT{stmt}\mth{)}}}

  \RULE{\rt{typedef\_ident}}
  \CASE{\T{id} \OR \mth{\T{metaid}^{\ssf{Type}}}}

  \RULE{\rt{type}}
  \CASE{\NT{ctype} \OR \mth{\T{metaid}^{\ssf{Type}}}}

  \RULE{\rt{pathToIsoFile}}
  \CASE{<.*>}

  \RULE{\rt{regexp}}
  \CASE{"\any{[\^{}"]}"}
\end{grammar}

\section{Comments and preprocessor directives}

A \verb+//+ or \verb+/* */+ comment that is annotated with + in the
leftmost column is considered to be added code.  A \verb+//+ or
\verb+/* */+ comment without such an annotation is considered to be a
comment about the SmPL code, and thus is not matched in the C code.

The following preprocessor directives can likewise be added.  They cannot
be matched against.  The entire line is added, but it is not parsed.

\begin{itemize}
\item \verb+if+
\item \verb+ifdef+
\item \verb+ifndef+
\item \verb+else+
\item \verb+elif+
\item \verb+endif+
\item \verb+error+
%\item \verb+pragma+
\item \verb+line+
\end{itemize}

\section{Command-line semantic match}

It is possible to specify a semantic match on the spatch command line,
using the argument {\tt -sp}.  In such a semantic match, any token
beginning with a capital letter is assumed to be a metavariable of type
{\tt metavariable}.  In this case, the parser must be able to figure out what
kind of metavariable it is.  It is also possible to specify the type of a
metavariable by enclosing the type in :'s, concatenated directly to the
metavariable name.

Some examples of semantic matches that can be given as an argument to {\tt
  -sp} are as follows:

\begin{itemize}
\item \texttt{f(e)}: This only matches the expression \texttt{f(e)}.
\item \texttt{f(E)}: This matches a call to f with any argument.
\item \texttt{F(E)}: This gives a parse error; the semantic patch parser
  cannot figure out what kind of metavariable \texttt{F} is.
\item \texttt{F:identifier:(E)}: This matches any one argument function
  call.
\item \texttt{f:identifier:(e:struct foo *:)}: This matches any one
  argument function call where the argument has type \texttt{struct foo
    *}.  Since the types of the metavariables are specified, it is not
  necessary for the metavariable names to begin with a capital letter.
\item \texttt{F:identifier:(F)}: This matches any one argument function call
  where the argument is the name of the function itself.  This example
  shows that it is not necessary to repeat the metavariable type name.
\item \texttt{F:identifier:(F:identifier:)}: This matches any one argument
  function call
  where the argument is the name of the function itself.  This example
  shows that it is possible to repeat the metavariable type name.
\end{itemize}

\texttt{When} constraints, \textit{e.g.} \texttt{when != e}, are allowed
but the expression \texttt{e} must be represented as a single token.

The generated semantic match behaves as though there were a \texttt{*} in front
of every token.

\section{Iteration}

It is possible to iterate Coccinelle, giving the subsequent iterations a
different set of virtual rules or virtual identifier bindings.  Coccinelle
currently supports iteration with both OCaml and Python scripting. An
example with OCaml is fond in {\tt demos/iteration.cocci}, a Python
example is found in {\tt demos/python\_iteration.cocci}.

The OCaml scripting iteration example starts as follows.

\begin{quote}
\begin{verbatim}
virtual after_start

@initialize:ocaml@

let tbl = Hashtbl.create(100)

let add_if_not_present from f file =
try let _ = Hashtbl.find tbl (f,file) in ()
with Not_found ->
   Hashtbl.add tbl (f,file) file;
   let it = new iteration() in
   (match file with
     Some fl -> it#set_files [fl]
   | None -> ());
   it#add_virtual_rule After_start;
   it#add_virtual_identifier Err_ptr_function f;
   it#register()
\end{verbatim}
\end{quote}

The respective Python scripting iteration example starts as follows:

\begin{quote}
\begin{verbatim}
virtual after_start

@initialize:python@
@@

seen = set()

def add_if_not_present (source, f, file):
    if (f, file) not in seen:
        seen.add((f, file))
        it = Iteration()
        if file != None:
            it.set_files([file])
        it.add_virtual_rule(after_start)
        it.add_virtual_identifier(err_ptr_function, f)
        it.register()
\end{verbatim}
\end{quote}

The virtual rule {\tt after\_start} is used to distinguish between the
first iteration (in which it is not considered to have matched) and all
others.  This is done by not mentioning {\tt after\_start} in the command
line, but adding it on each iteration.

The main code for performing the iteration is found in the function {\tt
  add\_if\_not\_present}, between the lines calling {\tt new iteration} and
{\tt register}.  {\tt New iteration} creates a structure representing the
new iteration.  {\tt set\_files} sets the list of files to be considered on
the new iteration.  If this function is not called, the new iteration
treats the same files as the current iteration.  {\tt add\_virtual\_rule a}
has the same effect as putting {\tt -D a} on the command line.  If
using OCaml scripting instead of Python scripting the first letter of the rule
name is capitalized, although this is not done elsewhere.
{\tt add\_virtual\_identifier x v} has the same effect as putting {\tt -D x=v}
on the command line.  Again, when using OCaml scripting there is a case change.
{\tt extend\_virtual\_identifiers()} (not shown) preserves all virtual
identifiers of the current iteration that are not overridden by calls to
{\tt add\_virtual\_identifier}.  Finally, the call to {\tt register} queues
the collected information to trigger a new iteration at some time in the
future.

Modification is not allowed when using iteration.  Thus, it is required to
use the {\tt --no-show-diff}, unless the semantic patch contains {\tt *}s
(a semantic match rather than a semantic patch).

When using Python scripting a tuple is used throughout the rest of the
code to ensure that the same information is not enqueued more than once.
When using OCaml scripting a hash table is used for the same purposes.
Coccinelle itself provides no support for this and as such addressing
this with scripting is necessary.

%%% Local Variables:
%%% mode: LaTeX
%%% TeX-master: "main_grammar"
%%% coding: utf-8
%%% TeX-PDF-mode: t
%%% ispell-local-dictionary: "american"
%%% End:
