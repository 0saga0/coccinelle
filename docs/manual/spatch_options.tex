\section{Introduction}

This document describes the options provided by Coccinelle.  The options
have an impact on various phases of the semantic patch application
process.  These are:

\begin{enumerate}
\item Selecting and parsing the semantic patch.
\item Selecting and parsing the C code.
\item Application of the semantic patch to the C code.
\item Transformation.
\item Generation of the result.
\end{enumerate}

\noindent
One can either initiate the complete process from step 1, or
to perform step 1 or step 2 individually.

Coccinelle has quite a lot of options.  The most common usages are as
follows, for a semantic match {\tt foo.cocci}, a C file {\tt foo.c}, and a
directory {\tt foodir}:

\begin{itemize}
\item {\tt spatch --parse-cocci foo.cocci}: Check that the semantic patch
  is syntactically correct.
\item {\tt spatch --parse-c foo.c}: Check that the C file
  is syntactically correct.  The Coccinelle C parser tries to recover
  during the parsing process, so if one function does not parse, it will
  start up again with the next one.  Thus, a parse error is often not a
  cause for concern, unless it occurs in a function that is relevant to the
  semantic patch.
\item {\tt spatch --sp-file foo.cocci foo.c}: Apply the semantic patch {\tt
    foo.cocci} to the file {\tt foo.c} and print out any transformations as
  the changes between the original and transformed code, using the program
  {\tt diff}.  {\tt --sp-file} is optional in this and the following cases.
\item {\tt spatch --sp-file foo.cocci foo.c --debug}:  The same as the
  previous case, but print out some information about the matching process.
\item {\tt spatch --sp-file foo.cocci --dir foodir}:  Apply the semantic
  patch {\tt foo.cocci} to all of the C files in the directory {\tt foodir}.
\item {\tt spatch --sp-file foo.cocci --dir foodir --include-headers}:  Apply
  the semantic patch {\tt foo.cocci} to all of the C files and header files
  in the directory {\tt foodir}. 
\end{itemize}

In the rest of this document, the options are annotated as follows:
\begin{itemize}
\item \FilledBigDiamondshape: a basic option, that is most likely of
  interest to all users.
\item \BigLowerDiamond: an option that is frequently used, often for better
understanding the effect of a semantic patch.
\item \BigDiamondshape: an option that is likely to be rarely used, but
  whose effect is still comprehensible to a user.
\item An option with no annotation is likely of interest only to
  developers.
\end{itemize}

\section{Selecting and parsing the semantic patch}

\subsection{Standalone options}

\normal{--parse-cocci $\langle$file$\rangle$}{ Parse a semantic
patch file and print out some information about it.}

\subsection{The semantic patch}

\minimum{--sp-file $\langle$file$\rangle$, -c $\langle$file$\rangle$, 
-cocci-file $\langle$file$\rangle$}{ Specify the name of the file
  containing the semantic patch.  The file name should end in {\tt .cocci}.
All three options do the same thing.  These options are optional.  If they
are not used, the single file whose name ends in \texttt{.cocci} is
assoumed to be the name of the file containing the semantic patch.}

\rare{--sp ``semantic patch string''}{Specify a semantic match as a
  command-line argument.  See the section ``Command-line semantic match''
  in the manual.}

\subsection{Isomorphisms}

\rare{--iso, --iso-file}{ Specify a file containing isomorphisms to be used in
place of the standard one.  Normally one should use the {\tt using}
construct within a semantic patch to specify isomorphisms to be used {\em
  in addition to} the standard ones.}

\rare{--iso-limit $\langle$int$\rangle$} Limit the depth of application of
isomorphisms to the specified integer.

\rare{--no-iso-limit} Put no limit on the number of times that
isomorphisms can be applied. This is the default.

\rare{--disable-iso}{Disable a specific isomorphism from the command line.
  This option can be specified multiple times.}

\developer{--track-iso}{ Gather information about isomorphism usage.}

\developer{--profile-iso}{ Gather information about the time required for
isomorphism expansion.}

\subsection{Display options}

\rare{--show-cocci}{Show the semantic patch that is being processed before
  expanding isomorphisms.}

\rare{--show-SP}{Show the semantic patch that is being processed after
  expanding isomorphisms.}

\rare{--show-ctl-text}{ Show the representation
of the semantic patch in CTL.}

\rare{--ctl-inline-let}{ Sometimes {\tt let} is used to name
intermediate terms CTL representation.  This option causes the let-bound
terms to be inlined at the point of their reference.
This option implicitly sets {\bf --show-ctl-text}.}

\rare{--ctl-show-mcodekind}{ Show
transformation information within the CTL representation
of the semantic patch. This option implicitly sets {\bf --show-ctl-text}.}

\rare{--show-ctl-tex}{ Create a LaTeX files showing the representation
of the semantic patch in CTL.}

\section{Selecting and parsing the C files}

\subsection{Standalone options}

\normal{--parse-c $\langle$file/dir$\rangle$}{ Parse a {\tt .c} file or all
  of the {\tt .c} files in a directory.  This generates information about
  any parse errors encountered.}

\normal{--parse-h $\langle$file/dir$\rangle$}{ Parse a {\tt .h} file or all
  of the {\tt .h} files in a directory.  This generates information about
  any parse errors encountered.}

\normal{--parse-ch $\langle$file/dir$\rangle$}{ Parse a {\tt .c} or {\tt
    .h} file or all of the {\tt .c} or {\tt .h} files in a directory.  This
  generates information about any parse errors encountered.}

\normal{--control-flow $\langle$file$\rangle$, --control-flow
$\langle$file$\rangle$:$\langle$function$\rangle$}{ Print a control-flow
graph for all of the functions in a file or for a specific function in a
file.  This requires {\tt dot} (http://www.graphviz.org/) and {\tt gv}.}

\rare{--control-flow-to-file $\langle$file$\rangle$,
  --control-flow-to-file
  $\langle$file$\rangle$:$\langle$function$\rangle$}{ Like --control-flow
  but just puts the dot output in a file in the {\em current} directory.
  For PATH/file.c, this produces file:xxx.dot for each (selected) function
  xxx in PATH/file.c.}

\rare{--type-c $\langle$file$\rangle$}{ Parse a C file and pretty-print a
version including type information.}

\developer{--tokens-c $\langle$file$\rangle$}{Prints the tokens in a C
  file.}

\developer{--parse-unparse $\langle$file$\rangle$}{Parse and then reconstruct
  a C file.}

\developer{--compare-c $\langle$file$\rangle$ $\langle$file$\rangle$,
  --compare-c-hardcoded}{Compares one C file to another, or compare the
file tests/compare1.c to the file tests/compare2.c.}

\developer{--test-cfg-ifdef $\langle$file$\rangle$}{Do some special
processing of \#ifdef and display the resulting control-flow graph.  This
requires {\tt dot} and {\tt gv}.}

\developer{--test-attributes $\langle$file$\rangle$,
           --test-cpp $\langle$file$\rangle$}{
Test the parsing of cpp code and attributes, respectively.}

\subsection{Selecting C files}

An argument that ends in {\tt .c} is assumed to be a C file to process.
Normally, only one C file or one directory is specified.  If multiple C
files are specified, they are treated in parallel, {\em i.e.}, the first
semantic patch rule is applied to all functions in all files, then the
second semantic patch rule is applied to all functions in all files, etc.
If a directory is specified then no files may be specified and only the
rightmost directory specified is used.

\normal{--include-headers}{ This option causes header files to be processed
independently.  This option only makes sense if a directory is specified
using {\bf --dir}.}

\normal{--use-glimpse}{ Use a glimpse index to select the files to which
a semantic patch may be relevant.  This option requires that a directory is
specified.  The index may be created using the script {\tt
  coccinelle/scripts/ glimpseindex-cocci.sh}.  Glimpse is available at
http://webglimpse.net/.  In conjunction with the option {\bf --patch-cocci}
this option prints the regular expression that will be passed to glimpse.}

\normal{--use-idutils}{ Use an id-utils index created using lid to select
  the files to which a semantic patch may be relevant.  This option
  requires that a directory is specified.  The index may be created using
  the script {\tt coccinelle/scripts/ idindex-cocci.sh}.  In conjunction
  with the option {\bf --patch-cocci} this option prints the regular
  expression that will be passed to glimpse.}

\rare{--dir}{ Specify a directory containing C files to process.  A trailing
  {\tt /} is permitted on the directory name and has no impact on the
  result.  By default, the include path will be set to the ``include''
  subdirectory of this directory.  A different include path can be
  specified using the option {\bf -I}.  {\bf --dir} only considers the
  rightmost directory in the argument list.  This behavior is convenient
  for creating a script that always works on a single directory, but allows
  the user of the script to override the provided directory with another
  one.  Spatch collects the files in the directory using {\tt find} and
  does not follow symbolic links.}

\developer{--kbuild-info $\langle$file$\rangle$}{ The specified file
  contains information about which sets of files should be considered in
  parallel.}

\developer{--disable-worth-trying-opt}{Normally, a C file is only
  processed if it contains some keywords that have been determined to be
  essential for the semantic patch to match somewhere in the file.  This
  option disables this optimization and tries the semantic patch on all files.}

\developer{--test $\langle$file$\rangle$}{ A shortcut for running Coccinelle
on the semantic patch ``file{\tt{.cocci}}'' and the C file
``file{\tt{.c}}''.  The result is put in the file {\tt
  /tmp/file{\tt{.res}}}.  If writing a file in /tmp with a non-fresh name
is a concern, then do not use this option.
}

\developer{--testall}{A shortcut for running Coccinelle on all files in a
  subdirectory {\tt tests} such that there are all of a {\tt .cocci} file, a {\tt
    .c} file, and a {\tt .res} file, where the {\tt .res} contains the
  expected result.}

\developer{--test-okfailed, --test-regression-okfailed} Other options for
keeping track of tests that have succeeded and failed.

\developer{--compare-with-expected}{Compare the result of applying
  Coccinelle to file{\tt{.c}} to the file file{\tt{.res}} representing the
  expected result.}

\developer{--expected-score-file $\langle$file$\rangle$}{
which score file to compare with in the testall run}

\subsection{Parsing C files}

\rare{--show-c}{Show the C code that is being processed.}

\rare{--parse-error-msg}{Show parsing errors in the C file.}

\rare{--verbose-parsing}{Show parsing errors in the C file, as well as
  information about attempts to accomodate such errors.  This implicitly
  sets --parse-error-msg.}

\rare{--type-error-msg}{Show information about where the C type checker
  was not able to determine the type of an expression.}

\rare{--int-bits $\langle$n$\rangle$, --long-bits
$\langle$n$\rangle$}{Provide integer size information. n is the number of
bits in an unsigned integer or unsigned long, respectively.  If only the
option {\bf --int-bits} is used, unsigned longs will be assumed to have
twice as many bits as unsigned integers.  If only the option {\bf
-long-bits} is used, unsigned ints will be assumed to have half as many
bits as unsigned integers.  This information is only used in determining
the types of integer constants, according to the ANSI C standard (C89).  If
neither is provided, the type of an integer constant is determined by the
sequence of ``u'' and ``l'' annotations following the constant.  If there
is none, the constant is assumed to be a signed integer.  If there is only
``u'', the constant is assumed to be an unsigned integer, etc.}

\rare{--no-loops}{Drop back edges for loops.  This may make a semantic
  patch/match run faster, at the cost of not finding matches that wrap
  around loops.}

\developer{--use-cache}{Use preparsed versions of the C files that are
stored in a cache.}

\developer{--cache-prefix}{Specify the directory in which to store 
preparsed versions of the C files.  This sets {--use-cache}}

\developer{--cache-limit}{Specify the maximum number of
preparsed C files to store.  The cache is cleared of all files with names
ending in .ast-raw and .depend-raw on reaching this limit.  Only
effective if --cache-prefix is used as well.  This is most useful when
iteration is used to process a file multiple times within a single run of
Coccinelle.}

\developer{--debug-cpp, --debug-lexer, --debug-etdt,
  --debug-typedef}{Various options for debugging the C parser.}

\developer{--filter-msg, --filter-define-error,
  --filter-passed-level}{Various options for debugging the C parser.}

\developer{--only-return-is-error-exit}{In matching ``{\tt{\ldots}}'' in
  a semantic patch or when forall is specified, a rule must match all
  control-flow paths starting from a node matching the beginning of the
  rule.  This is relaxed, however, for error handling code.  Normally, error
  handling code is considered to be a conditional with only a then branch
  that ends in goto, break, continue, or return.  If this option is set,
  then only a then branch ending in a return is considered to be error
  handling code.  Usually a better strategy is to use {\tt when strict} in
  the semantic patch, and then match explicitly the case where there is a
  conditional whose then branch ends in a return.}

\subsubsection*{Macros and other preprocessor code}

\normal{--macro-file $\langle$file$\rangle$}{
  Extra macro definitions to be taken into account when parsing the C
  files.  This uses the provided macro definitions in addition to those in
  the default macro file.}

\normal{--macro-file-builtins $\langle$file$\rangle$}{
  Builtin macro definitions to be taken into account when parsing the C
  files.  This causes the macro definitions provided in the default macro
  file to be ignored and the ones in the specified file to be used instead.}

\rare{--ifdef-to-if,-no-ifdef-to-if}{
The option {\bf --ifdef-to-if}
represents an {\tt \#ifdef} in the source code as a conditional in the
control-flow graph when doing so represents valid code.  {\bf
-no-ifdef-to-if} disables this feature.  {\bf --ifdef-to-if} is the
default.
}

\rare{--noif0-passing}{ Normally code under \#if 0 is ignored.  If this
option is set then the code is considered, just like the code under any
other \#ifdef.}

\rare{--defined $s$}{The string $s$ is a comma-separated list of constants
  that should be considered to be defined, with respect to uses of {\tt
    \#ifdef} and {\tt \#ifndef} in C code.  No spaces should appear in $s$.
  Multiple {\bf --defined} arguments can be provided and the list of
  strings accumulates.  For the provided strings any {\tt \#else}s of {\tt
    \#ifdef}s are ignored and any {\tt \#ifndef}s are ignored, unless the
  argument {\bf --noif0-passing} is also given, in which case {\bf
    --defined} has no effect.  Note that occurrences of {\tt \#define} in
  the C code have no effect on the list of defined constants.}

\rare{--undefined $s$}{Analogous to {\bf --defined} except that the strings
  represent constants that should be considered to be undefined.}

\developer{--noadd-typedef-root}{This seems to reduce the scope of a
  typedef declaration found in the C code.}

\subsubsection*{Include files}

\normal{--recursive-includes, --all-includes, --local-includes,
  --no-includes}{ These options control which include files mentioned in a
  C file are taken into account.  {\bf --recursive-includes} indicates
  that all included files mentioned in the .c file(s) or any included files
  will be processed.  {\bf --all-includes} indicates that all included
  files mentioned in the .c file(s) will be processed.  {\bf
    --local-includes} indicates that only included files in the current
  directory will be processed.  {\bf --no-includes} indicates that no
  included files will be processed.  If the semantic patch contains type
  specifications on expression metavariables, then the default is {\bf
    --local-includes}.  Otherwise the default is {\bf --no-includes}.  At
  most one of these options can be specified.}

\normal{-I $\langle$path$\rangle$}{ This option specifies a directory
  in which to find non-local include files.  This option can be used
  several times.}

\rare{--relax-include-path}{This option when combined with --all-includes
  causes the search for local
  include files to consider the current directory, even if the include
  patch specifies a subdirectory.  This is really only useful for testing,
  eg with the option {\bf --testall}}

\rare{--c++}{Make an extremely minimal effort to parse C++ code.  Currently,
  this is limited to allowing identifiers to contain ``::'', tilde, and
  template invocations.  Consider testing your code first with spatch
  --type-c to see if there are any type annotations in the code you are
  interested in processing.  If not, then it was probably not parsed.}

\section{Application of the semantic patch to the C code}

\subsection{Feedback at the rule level during the application of the
  semantic patch}

\normal{--show-bindings}{
Show the environments with respect to which each rule is applied and the
bindings that result from each such application.}

\normal{--show-dependencies}{ Show the status (matched or unmatched) of the
rules on which a given rule depends.  {\bf --show-dependencies} implicitly
sets {\bf --show-bindings}, as the values of the dependencies are
environment-specific.}

\normal{--show-trying}{
Show the name of each program element to which each rule is applied.}

\normal{--show-transinfo}{
Show information about each transformation that is performed.
The node numbers that are referenced are the number of the nodes in the
control-flow graph, which can be seen using the option {\bf --control-flow}
(the initial control-flow graph only) or the option {\bf --show-flow} (the
control-flow graph before and after each rule application).}

\normal{--show-misc}{Show some miscellaneous information.}

\rare{--show-flow $\langle$file$\rangle$, --show-flow
  $\langle$file$\rangle$:$\langle$function$\rangle$} Show the control-flow
graph before and after the application of each rule.

\developer{--show-before-fixed-flow}{This is similar to {\bf
    --show-flow}, but shows a preliminary version of the control-flow graph.}

\subsection{Feedback at the CTL level during the application of the
  semantic patch}

\normal{--verbose-engine}{Show a trace of the matching of atomic terms to C
  code.}

\rare{--verbose-ctl-engine}{Show a trace of the CTL matching process.
  This is unfortunately rather voluminous and not so helpful for someone
  who is not familiar with CTL in general and the translation of SmPL into
  CTL specifically.  This option implicitly sets the option {\bf
    --show-ctl-text}.}

\rare{--graphical-trace}{Create a pdf file containing the control flow
  graph annotated with the various nodes matched during the CTL matching
  process.  Unfortunately, except for the most simple examples, the output
  is voluminous, and so the option is not really practical for most
  examples.  This requires {\tt dot} (http://www.graphviz.org/) and {\tt
  pdftk}.}

\rare{--gt-without-label}{The same as {\bf --graphical-trace}, but the PDF
  file does not contain the CTL code.}

\rare{--partial-match}{
Report partial matches of the semantic patch on the C file.  This can
  be substantially slower than normal matching.}

\rare{--verbose-match}{
Report on when CTL matching is not applied to a function or other program
unit because it does not contain some required atomic pattern.
This can be viewed as a simpler, more efficient, but less informative
version of {\bf --partial-match}.}

\subsection{Actions during the application of the semantic patch}

\normal{-D rulename}{Run the patch considering that the virtual rule
  ``rulename'' is satisfied.  Virtual rules should be declared at the
  beginning of the semantic patch in a comma separated list following the
  keyword virtual.  Other rules can depend on the satisfaction or non
  satifaction of these rules using the keyword {\tt depends on} in the
  usual way.}

\normal{-D variable=value}{Run the patch considering that the virtual
  identifier metavariable ``variable'' is bound to ``value''.  Any
  identifier metavariable can be designated as being virtual by giving it
  the rule name {\tt virtual}.  An example is in demos/vm.coci}

\rare{--allow-inconsistent-paths}{Normally, a term that is transformed
  should only be accessible from other terms that are matched by the
  semantic patch.  This option removes this constraint.  Doing so, is
  unsafe, however, because the properties that hold along the matched path
  might not hold at all along the unmatched path.}

\rare{--disallow-nested-exps}{In an expression that contains repeated
  nested subterms, {\em e.g.} of the form {\tt f(f(x))}, a pattern can
  match a single expression in multiple ways, some nested inside others.
  This option causes the matching process to stop immediately at the
  outermost match.  Thus, in the example {\tt f(f(x))}, the possibility
  that the pattern {\tt f(E)}, with metavariable {\tt E}, matches with {\tt
    E} as {\tt x} will not be considered.}

\rare{--no-safe-expressions}{normally, we check that an expression does
       not match something earlier in the disjunction.  But for large
       disjunctions, this can result in a very big CTL formula.  So this
       option give the user the option to say he doesn't want this feature,
       if that is the case.}

\rare{--pyoutput coccilib.output.Gtk, --pyoutput coccilib.output.Console}{
This controls whether Python output is sent to Gtk or to the console.  {\bf
  --pyoutput coccilib.output.Console} is the default.  The Gtk option is
currently not well supported.}

\developer{--loop}{When there is ``{\tt{\ldots}}'' in the semantic patch,
  the CTL operator {\sf AU} is used if the current function does not
  contain a loop, and {\sf AW} may be used if it does.  This option causes
  {\sf AW} always to be used.}

\rare{--ocaml-regexps}{Use the regular expressions provided by the OCaml
  \texttt{Str} library.  This is the default if the PCRE library is not
  available.  Otherwise PCRE regular expressions are used by default.}

\developer{--steps $\langle$int$\rangle$}{
This limits the number of steps performed by the CTL engine to the
specified number.  This option is unsafe as it might cause a rule to fail
due to running out of steps rather than due to not matching.}

\developer{--bench $\langle$int$\rangle$}{This collects various information
  about the operations performed during the CTL matching process.}

% \developer{--popl, --popl-mark-all, --popl-keep-all-wits}{
% These options use a simplified version of the SmPL language.  {\bf
%   --popl-mark-all} and {\bf --popl-keep-all-wits} implicitly set {\bf
%   --popl}.}

\rare{--reverse}{Inverts the semantic patch before applying it.
A potential use case is backporting changes to previous versions. If a
semantic patch represents an API change, then the reverse undoes the
API change. Note that inverting a semantic patch is not always possible.
In particular, the composition of a
semantic patch with its inverse is not guaranteed to be an empty patch.}

\section{Generation of the result}

Normally, the only output is the differences between the original code and
the transformed code obtained using the program {\tt diff} with the unified
format option.  If
stars are used in column 0 rather than {\tt -} and {\tt +}, then the {\tt
  -} lines in the output are the lines that matched the stars.

\normal{--keep-comments}{Don't remove comments adjacent to removed code.}

\normal{--linux-spacing, --smpl-spacing}{Control the spacing within the code
  added by the semantic patch.  The option {\bf --linux-spacing} causes
  spatch to follow the conventions of Linux, regardless of the spacing in
  the semantic patch.  This is the default.  The option {\bf
  --smpl-spacing} causes spatch to follow the spacing given in the semantic
  patch, within individual lines.}

\rare{-o $\langle$file$\rangle$}{ This causes the transformed code to be
  placed in the file {\tt file}.  The difference between the original code
  and the transformed code is still printed to the standard output using
  {\tt diff} with the unified format option.  This option only makes sense
  when {\tt -} and {\tt +} are used.}

\rare{--in-place}{ Modify the input file to contain the transformed code.
  The difference between the original code and the transformed code is
  still printed to the standard output using {\tt diff} with the unified
  format option.  By default, the input file is overwritten when using this
  option, with no backup.  This option only makes sense when {\tt -} and
  {\tt +} are used.}

\rare{--backup-suffix $s$}{The suffix $s$ of the file to use in making a
  backup of the original file(s).  This suffix should include the leading
  ``.'', if one is desired.  This option only has an effect when the option
  {\tt --in-place} is also used.}

\rare{--out-place}{ Store the result of modifying the code in a .cocci-res
  file.  The difference between the original code
  and the transformed code is still printed to the standard output using
  {\tt diff} with the unified format option.  This option only
  makes sense when {\tt -} and {\tt +} are used.}

\rare{--no-show-diff}{ Normally, the difference between the original and
  transformed code is printed on the standard output.  This option causes
  this not to be done.}

\rare{-U}{ Set number of context lines to be provided by {\tt diff}.}

\rare{--patch $\langle$path$\rangle$}{The prefix of the pathname of the
  directory or file name that should dropped from the {\tt diff} line in the
  generated patch.  This is useful if you want to apply a patch only to a
  subdirectory of a source code tree but want to create a patch that can be
  applied at the root of the source code tree.  An example could be {\tt
  spatch --sp-file foo.cocci --dir /var/linuxes/linux-next/drivers --patch
  /var/linuxes/linux-next}.  A trailing {\tt /} is permitted on the
  directory name and has no impact on the result.}

\rare{--save-tmp-files}{Coccinelle creates some temporary
  files in {\tt /tmp} that it deletes after use.  This option causes these
  files to be saved.}

\developer{--debug-unparsing}{Show some debugging information about the
  generation of the transformed code.  This has the side-effect of
  deleting the transformed code.}


\section{Other options}

\subsection{Version information}

\normal{--version}{ The version of Coccinelle.  No other options are
allowed.}

\normal{--date}{ The date of the current version of Coccinelle. No other
options are allowed.}

\subsection{Help}

\minimum{--h, --shorthelp}{ The most useful commands.}

\minimum{--help, --help, --longhelp}{ A complete listing of the available
commands.}

\subsection{Controlling the execution of Coccinelle}

\normal{--timeout $\langle$int$\rangle$}{ The maximum time in seconds for
  processing a single file.}

\rare{--max $\langle$int$\rangle$}{This option informs Coccinelle of the
  number of instances of Coccinelle that will be run concurrently.  This
  option requires {\bf --index}.  It is usually used with {\bf --dir}.}

\rare{--index $\langle$int$\rangle$}{This option informs Coccinelle of
  which of the concurrent instances is the current one.  This option
  requires {\bf --max}.}

\rare{--mod-distrib}{When multiple instances of Coccinelle are run in
  parallel, normally the first instance processes the first $n$ files, the
  second instance the second $n$ files, etc.  With this option, the files
  are distributed among the instances in a round-robin fashion.}

\developer{--debugger}{Option for running Coccinelle from within the OCaml
  debugger.}

\developer{--profile}{ Gather timing information about the main Coccinelle
functions.}

\developer{--disable-once}{Print various warning messages every time some
condition occurs, rather than only once.}

\subsection{Miscellaneous}

\rare{--quiet}{Suppress most output.  This is the default.}

%\developer{--pad, -hrule $\langle$dir$\rangle$, -xxx, -l1}{}
\developer{--pad, --xxx, --l1}{}


%%% Local Variables:
%%% mode: LaTeX
%%% TeX-master: "main_options"
%%% coding: utf-8
%%% TeX-PDF-mode: t
%%% ispell-local-dictionary: "american"
%%% End:
