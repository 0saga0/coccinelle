\documentclass{report}

%******************************************************************************
% Prelude
%******************************************************************************
%------------------------------------------------------------------------------
% Packages
%------------------------------------------------------------------------------

\usepackage{ifthen}
\usepackage{hevea}

\usepackage{times}
\usepackage{fullpage}

\usepackage[T1]{fontenc}

\usepackage{amsmath}
\usepackage{amssymb}

% fancy symbol, but require latex-extra-fonts (huge) package
\usepackage[geometry]{ifsym}



\usepackage{graphics}
\usepackage[pdftex]{graphicx}


\usepackage{epsfig}
\usepackage{subfigure}
\usepackage{wrapfig}

\usepackage[all]{xy}


\usepackage{fancyvrb}
\usepackage{moreverb}
\usepackage{alltt}

\usepackage{boxedminipage}

\usepackage{xspace}
\usepackage{endnotes}

\usepackage{multirow}
%\usepackage{colortbl} % conflict with color package below

\usepackage{listings}
%\usepackage{code/lgrind}

\usepackage[pdfborder={0 0 0}]{hyperref}
%\usepackage{url}

\usepackage[usenames,dvipsnames]{color}

%------------------------------------------------------------------------------
% Shortcuts
%------------------------------------------------------------------------------


% Very convenient to add comments on the paper. Just set the boolean
% to false before sending the paper:
\newboolean{showcomments}
\setboolean{showcomments}{true}
\ifthenelse{\boolean{showcomments}}
{ \newcommand{\mynote}[2]{
    \fbox{\bfseries\sffamily\scriptsize#1}
    {\small$\blacktriangleright$\textsf{\emph{#2}}$\blacktriangleleft$}}}
{ \newcommand{\mynote}[2]{}}

\newcommand\jl[1]{\mynote{Julia}{#1}}



\newcommand{\sizecodebis}[0]{\scriptsize}

\newcommand{\mita}[1]{\mbox{\it{{#1}}}}
\newcommand{\mtt}[1]{\mbox{\tt{{#1}}}}
\newcommand{\msf}[1]{\mbox{\sf{{#1}}}}
\newcommand{\stt}[1]{\mbox{\scriptsize\tt{{#1}}}}
\newcommand{\ssf}[1]{\mbox{\scriptsize\sf{{#1}}}}
\newcommand{\sita}[1]{\mbox{\scriptsize\it{{#1}}}}
\newcommand{\mrm}[1]{\mbox{\rm{{#1}}}}
\newcommand{\mth}[1]{\({#1}\)}
\newcommand{\entails}[2]{\begin{array}{@{}c@{}}{#1}\\\hline{#2}\end{array}}
\newcommand{\ttlb}{\mbox{\tt \char'173}}
\newcommand{\ttrb}{\mbox{\tt \char'175}}
\newcommand{\ttmid}{\mbox{\tt \char'174}}
\newcommand{\tttld}{\mbox{\tt \char'176}}

\newcommand{\fixme}[1]{{\color{red} #1}}


%------------------------------------------------------------------------------
%%%Squeeze space in bibliographybg
%  \let\oldthebibliography=\thebibliography
%  \let\endoldthebibliography=\endthebibliography
%  \renewenvironment{thebibliography}[1]{%
%    \begin{oldthebibliography}{#1}%
%      \setlength{\parskip}{0ex}%
%      \setlength{\itemsep}{0ex}%
%  }%
%  {%
%    \end{oldthebibliography}%
%  }


%------------------------------------------------------------------------------

%christian lindig tricks: 
% http://www.st.cs.uni-saarland.de/~lindig/tex.html

%\newif\ifdraft\drafttrue

%see overfull 
%\ifdraft
%\overfullrule3pt
%\fi



%\def\<#1>{\texttt{#1}}
%Now I can write \<some code here>, which feels more natural 
% in the editor than \texttt{some code here}.

%e.g:
%  \newcommand{\mita}[1]{\mbox{\it{{#1}}}}


\newcommand{\minimum}[2]{\paragraph*{\makebox[0in][r]{\FilledBigDiamondshape\,\,} {{#1}}} {#2}}
\newcommand{\normal}[2]{\paragraph*{\makebox[0in][r]{\BigLowerDiamond\,\,} {{#1}}} {#2}}
\newcommand{\rare}[2]{\paragraph*{\makebox[0in][r]{\BigDiamondshape\,\,} {{#1}}} {#2}}
\newcommand{\developer}[2]{\paragraph*{{#1}} {#2}}


\lstset{basicstyle=\ttfamily,numbers=left, numberstyle=\tiny, stepnumber=1, numbersep=5pt,language=C,commentstyle=\color{OliveGreen},keywordstyle=\color{blue},stringstyle=\color{BrickRed}}

%
% You must prefix the +/- lines of
% cocci files with @+/@- respectively.
% This will enable the automatic coloration.
%
% Note: You need at least the following version of hevea
% http://hevea.inria.fr/distri/unstable/hevea-2008-12-17.tar.gz
%
\ifhevea % For HTML generation
\lstdefinelanguage{Cocci}{
morekeywords={idexpression,expression,statement,identifier,type,
parameter,list,when,strict,any,forall,local,position,typedef},
keywordstyle=\color{OliveGreen}\bfseries,
sensitive=false,
moredelim=[is][\color{blue}]{@M}{@M},
moredelim=[il][\color{OliveGreen}]{@+},
moredelim=[il][\color{BrickRed}]{@-}}

\lstdefinelanguage{PatchC}[ANSI]{C}{
stringstyle=\color{black},
moredelim=[il][\color{OliveGreen}]{@+},
moredelim=[il][\color{BrickRed}]{@-},
moredelim=[il][\color{Plum}]{@M}}

\else % For DVI/PS/PDF generation
\lstdefinelanguage{Cocci}{
morekeywords={idexpression,expression,statement,identifier,type,
parameter,list,when,strict,any,forall,local,position,typedef},
keywordstyle=\color{OliveGreen}\bfseries,
sensitive=false,
moredelim=*[is][\color{blue}]{@M}{@M},
moredelim=[il][\color{OliveGreen}]{@+},
moredelim=[il][\color{BrickRed}]{@-}}

\lstdefinelanguage{PatchC}[ANSI]{C}{
stringstyle=\color{black},
moredelim=[il][\color{OliveGreen}]{@+},
moredelim=[il][\color{BrickRed}]{@-},
moredelim=[il][\color{Plum}]{@M}}
\fi

\newif\iflanguagestyle
\languagestylefalse
 % order is important
% Definition of a grammar (BNF style) package for Latex and Hevea


\ifhevea
% Definition for Hevea (HTML generation)
\def\T#1{{\sf{#1}}}
\def\NTS#1{{\maroon #1\/}}
\def\KW#1{{\blue #1}}
\def\gramor{{\black $|$}}
\def\grameq{{\black \quad::=\quad}}
\def\lparen{{\black (}}
\def\rparen{{\black )}}
\def\lbracket{{\black [}}
\def\rbracket{{\black ]}}
\def\plus{{\black +}}
\def\questionmark{{\black ?}}
\def\etoile{{\black *}}
\else
% Definition for LaTeX
\def\T#1{{\textsf{\small{#1}}}}
\def\NTS#1{{\it #1\/}}
\def\KW#1{{\mtt{#1}}}
%\def\gramor{$\vert$}
\def\gramor{$\mid$}
\def\grameq{\,\,\,::=\,\,\,\,\,}
\def\lparen{(}
\def\rparen{)}
\def\lbracket{$[$}
\def\rbracket{$]$}
\def\plus{+}
\def\questionmark{?}
\def\etoile{*}
\fi

\def\NT#1{\hyperlink{#1}{\NTS{#1}}}
\def\group#1{{\rm\lparen}#1{\rm\rparen}}
\def\range#1#2{#1{..}#2}
\def\any#1{#1$^{\etoile}$}
\def\some#1{#1$^{\plus}$}
\def\ANY#1{\any{{\rm\lparen}#1{\rm\rparen}}}
\def\SOME#1{\some{{\rm\lparen}#1{\rm\rparen}}}
\def\OR{\gramor\ }

\iflanguagestyle
% Option notation : [ xxx ] versus (xxx)^?
\def\opt#1{#1$^{\questionmark}$}
\def\OPT#1{\opt{{\rm\lparen}#1{\rm\rparen}}}
\else
\def\opt#1{{\lbracket}#1{\rbracket}}
\def\OPT#1{\opt{#1}}
\fi

\newenvironment{grammar}{\begin{center}\begin{tabular}{l@{}c@{}l}}{\end{tabular}\end{center}}
\def\RULE#1\CASE#2{\NTS{#1} & \grameq & \KW{#2} \\}
\def\CASE#1{& \gramor & \KW{#1} \\}

\newcommand{\rt}[1]{\hypertarget{#1}{#1}}
\newcommand{\bs}{\textbackslash}

\def\lb{\char123}
\def\rb{\char125}
\def\lt{\tt\char60}
\def\gt{\tt\char62}
\def\caret{\tt\^{}}



%------------------------------------------------------------------------------
% Globals
%------------------------------------------------------------------------------
\newcommand{\spatch}{\texttt{spatch}\xspace}
\newcommand{\sgrep}{\texttt{sgrep}\xspace}

\newcommand{\cpp}{\texttt{cpp}\xspace}


\newcommand{\cocciversion}{0.2.2\xspace}


%e.g:
% \newcommand{\avgcorrect}{96\%\xspace} % if consider all files
% \newcommand{\MyTool}{aComment\xspace}
% \newcommand{\bugsfound}{XXX\xspace}


%******************************************************************************
% Title
%******************************************************************************
\begin{document}
%don't want date printed
\date{}

%------------------------------------------------------------------------------
\title{
{\Huge \bf Coccinelle}\\
{User's manual}\\
{release \cocciversion}\\
}
%\title{\spatch and \sgrep manual}

\author{
Julia Lawall and Yoann Padioleau \\
{(with contributions from
Rene Rydhof Hansen,
Nicolas Palix,
Henrik Stuart)
}
}
%src: Xavier Leroy manual

\date{\today}

\maketitle
\tableofcontents

%******************************************************************************
% Body
%******************************************************************************

\chapter*{Foreword}

This manual documents the release \cocciversion of Coccinelle.
It is organized as follows:
\begin{itemize}
  \item Part~\ref{part:usermanual} is an introduction to Coccinelle
  \item Part~\ref{part:refmanual} is the reference description
    of Coccinelle, its language and command line tool.
\end{itemize}

\section*{Conventions}

\section*{Copyright}

%coupling: copyright.txt
Coccinelle is copyright \copyright 2005, 2006, 2007, 2008, 2009
University of Copenhagen DIKU and Ecole des Mines de Nantes.

Coccinelle is open source and can be freely redistributed under the
terms of the GNU General Public License version 2. See the file
\verb+license.txt+ in the distribution for licensing information.

The present documentation is copyright 2008, 2009 Yoann Padioleau
and Julia Lawall and distributed under the terms of the
GNU Free Documentation License version 1.3.

\section*{Availability}

Coccinelle can be freely downloaded
from \verb+http://coccinelle.lip6.fr/+.
This website contains also additional information.


\part{User Manual}
\label{part:usermanual}

\chapter{Introduction}
%src: cocci website, LWN article

Coccinelle is a tool to help automate repetitive 
source-to-source style-preserving program transformations
on C source code, like for instance to perform some refactorings.
%coupling: readme.txt
Coccinelle is presented as a command line tool called \spatch that takes
as input the name of a file containing the specification of a program
transformation, called a {\em semantic patch}, and a set of C files,
and then performs the transformation on all those C files.
%synopsis ?

To make it easy to express those transformations,
Coccinelle proposes a WYSISWYG approach where the C programmer 
can leverage the things he already knows: the C syntax
and the patch syntax. Indeed, with Coccinelle transformations
are written in specific language called SmPL, for 
Semantic Patch Language, which as the name suggests is very
close to the syntax of a patch, but which does not 
work at a line level, than traditional patches do.
but a more high level, or semantic level.

Here is an example of a simple program transformation.
To replace every calls to \verb+foo+ of any expression $x$ 
to a call to \verb+bar+, create a semantic patch file \verb+ex1.cocci+
(semantic patches usually ends with the \verb+.cocci+  filename extension)
containing:
\begin{verbatim}
@@ expression x; @@

- foo(x)
+ bar(x)

\end{verbatim}

Then to ``apply'' the specified program transformation to a set of C files,
simply do:
\begin{verbatim}
$ spatch -sp_file ex1.cocci *.c
\end{verbatim}


Coccinelle primarily targets ANSI C, and supports some GCC extensions.  It
has only partial support for K\&R C.  K\&R function declarations are only
recognized if the parameter declarations are indented.  Furthermore, the
parameter names are subsequently considered to be type names, due to
confusion with function prototypes, in which a name by itself is indeed the
name of a type.


%command line: 

%can do inplace, or with git, cf chapter on developing ...

%Other approaches 
%instead of 
%expressing the transformation on the internal representation
%of a C frontend, for instance the abstract syntax tree 
%used internally by gcc, which would require for the user
%to learn how to use this internal data structure, 

%if can find and transform, can also find, so semantic grep.

%vs regexp
%vs ast

%features:
%src: darcs manual

%%% Local Variables:
%%% mode: LaTeX
%%% coding: utf-8
%%% TeX-PDF-mode: t
%%% ispell-local-dictionary: "american"
%%% End:


%##############################################################################
\chapter{Installing Coccinelle}
%##############################################################################
%\chapter{Building \spatch and \sgrep}

\section{Requirements}
%coupling: with install.txt

\section{Getting Coccinelle}

\section{Compiling Coccinelle}
%coupling: with install.txt

\section{Running Coccinelle}
%coupling: with readme.txt


%%% Local Variables:
%%% mode: LaTeX
%%% coding: utf-8
%%% TeX-PDF-mode: t
%%% ispell-local-dictionary: "american"
%%% End:


\chapter{Tutorial}



\chapter{Examples}

\section{Examples}
%\label{sec:examples}

This section presents a range of examples.  Each
example is presented along with some C code to which it is
applied. The description explains the rules and the matching process.

\subsection{Function renaming}

One of the primary goals of Coccinelle is to perform software
evolution.  For instance, Coccinelle could be used to perform function
renaming. In the following example, every occurrence of a call to the
function \texttt{foo} is replaced by a call to the
function \texttt{bar}.\\

\begin{tabular}{ccc}
Before & Semantic patch & After \\
\begin{minipage}[t]{.3\linewidth}
\begin{lstlisting}
#DEFINE TEST "foo"

printf("foo");

int main(int i) {
//Test
  int k = foo();

  if(1) {
    foo();
  } else {
    foo();
  }

  foo();
}
\end{lstlisting}
\end{minipage}
&
\begin{minipage}[t]{.3\linewidth}
\begin{lstlisting}[language=Cocci]
@M@@

@@@M


@-- foo()
@++ bar()
\end{lstlisting}
\end{minipage}
&
\begin{minipage}[t]{.3\linewidth}
\begin{lstlisting}
#DEFINE TEST "foo"

printf("foo");

int main(int i) {
//Test
  int k = bar();

  if(1) {
    bar();
  } else {
    bar();
  }

  bar();
}
\end{lstlisting}
\end{minipage}\\
\end{tabular}

\newpage
\subsection{Removing a function argument}

Another important kind of evolution is the introduction or deletion of a
function argument. In the following example, the rule \texttt{rule1} looks
for definitions of functions having return type \texttt{irqreturn\_t} and
two parameters. A second \emph{anonymous} rule then looks for calls to the
previously matched functions that have three arguments. The third argument
is then removed to correspond to the new function prototype.\\

\begin{tabular}{c}
\begin{lstlisting}[language=Cocci,name=arg]
@M@ rule1 @
identifier fn;
identifier irq, dev_id;
typedef irqreturn_t;
@@@M

static irqreturn_t fn (int irq, void *dev_id)
{
   ...
}

@M@@
identifier rule1.fn;
expression E1, E2, E3;
@@@M

 fn(E1, E2
@--  ,E3
   )
\end{lstlisting}\\
\end{tabular}

\vspace{1cm}

\begin{tabular}{c}
  \texttt{drivers/atm/firestream.c} at line 1653 before transformation\\
\begin{lstlisting}[language=PatchC]
static void fs_poll (unsigned long data)
{
        struct fs_dev *dev = (struct fs_dev *) data;

@-        fs_irq (0, dev, NULL);
        dev->timer.expires = jiffies + FS_POLL_FREQ;
        add_timer (&dev->timer);
}
\end{lstlisting}\\
\vspace{1cm}
\\


  \texttt{drivers/atm/firestream.c} at line 1653 after transformation\\
\begin{lstlisting}[language=PatchC]
static void fs_poll (unsigned long data)
{
        struct fs_dev *dev = (struct fs_dev *) data;

@+        fs_irq (0, dev);
        dev->timer.expires = jiffies + FS_POLL_FREQ;
        add_timer (&dev->timer);
}
\end{lstlisting}\\
\end{tabular}

\newpage
\subsection{Introduction of a macro}

To avoid code duplication or error prone code, the kernel provides
macros such as \texttt{BUG\_ON}, \texttt{DIV\_ROUND\_UP} and
\texttt{FIELD\_SIZE}. In these cases, the semantic patches look for
the old code pattern and replace it by the new code.\\

A semantic patch to introduce uses of the \texttt{DIV\_ROUND\_UP} macro
looks for the corresponding expression, \emph{i.e.}, $(n + d - 1) /
d$. When some code is matched, the metavariables \texttt{n} and \texttt{d}
are bound to their corresponding expressions. Finally, Coccinelle rewrites
the code with the \texttt{DIV\_ROUND\_UP} macro using the values bound to
\texttt{n} and \texttt{d}, as illustrated in the patch that follows.\\

\begin{tabular}{c}
Semantic patch to introduce uses of the \texttt{DIV\_ROUND\_UP} macro\\
\begin{lstlisting}[language=Cocci,name=divround]
@M@ haskernel @
@@@M

#include <linux/kernel.h>

@M@ depends on haskernel @
expression n,d;
@@@M

(
@-- (((n) + (d)) - 1) / (d))
@++ DIV_ROUND_UP(n,d)
|
@-- (((n) + ((d) - 1)) / (d))
@++ DIV_ROUND_UP(n,d)
)
\end{lstlisting}
\end{tabular}\\

\vspace{1cm}

\begin{tabular}{c}
Example of a generated patch hunk\\
\begin{lstlisting}[language=PatchC]
@---- a/drivers/atm/horizon.c
@++++ b/drivers/atm/horizon.c
@M@@ -698,7 +698,7 @@ got_it:
                if (bits)
                        *bits = (div<<CLOCK_SELECT_SHIFT) | (pre-1);
                if (actual) {
@--                       *actual = (br + (pre<<div) - 1) / (pre<<div);
@++                       *actual = DIV_ROUND_UP(br, pre<<div);
                        PRINTD (DBG_QOS, "actual rate: %u", *actual);
                }
                return 0;
\end{lstlisting}
\end{tabular}\\

\newpage

The \texttt{BUG\_ON} macro makes a assertion about the value of an
expression. However, because some parts of the kernel define
\texttt{BUG\_ON} to be the empty statement when debugging is not wanted,
care must be taken when the asserted expression may have some side-effects,
as is the case of a function call. Thus, we create a rule introducing
\texttt{BUG\_ON} only in the case when the asserted expression does not
perform a function call.

On particular piece of code that has the form of a function call is a use
of \texttt{unlikely}, which informs the compiler that a particular
expression is unlikely to be true.  In this case, because \texttt{unlikely}
does not perform any side effects, it is safe to use \texttt{BUG\_ON}.  The
second rule takes care of this case.  It furthermore disables the
isomorphism that allows a call to \texttt{unlikely} be replaced with its
argument, as then the second rule would be the same as the first one.\\

\begin{tabular}{c}
\begin{lstlisting}[language=Cocci,name=bugon]
@M@@
expression E,f;
@@@M

(
  if (<+... f(...) ...+>) { BUG(); }
|
@-- if (E) { BUG(); }
@++ BUG_ON(E);
)

@M@ disable unlikely @
expression E,f;
@@@M

(
  if (<+... f(...) ...+>) { BUG(); }
|
@-- if (unlikely(E)) { BUG(); }
@++ BUG_ON(E);
)
\end{lstlisting}\\
\end{tabular}\\

For instance, using the semantic patch above, Coccinelle generates
patches like the following one.

\begin{tabular}{c}
\begin{lstlisting}[language=PatchC]
@---- a/fs/ext3/balloc.c
@++++ b/fs/ext3/balloc.c
@M@@ -232,8 +232,7 @@ restart:
                prev = rsv;
        }
        printk("Window map complete.\n");
@--       if (bad)
@--               BUG();
@++       BUG_ON(bad);
 }
 #define rsv_window_dump(root, verbose) \
        __rsv_window_dump((root), (verbose), __FUNCTION__)
\end{lstlisting}
\end{tabular}

\newpage
\subsection{Look for \texttt{NULL} dereference}

This SmPL match looks for \texttt{NULL} dereferences. Once an
expression has been compared to \texttt{NULL}, a dereference to this
expression is prohibited unless the pointer variable is reassigned.\\

\begin{tabular}{c}
    Original \\

\begin{lstlisting}
foo = kmalloc(1024);
if (!foo) {
  printk ("Error %s", foo->here);
  return;
}
foo->ok = 1;
return;
\end{lstlisting}\\
  \end{tabular}

\vspace{1cm}

\begin{tabular}{c}
  Semantic match\\

\begin{lstlisting}[language=Cocci]
@M@@
expression E, E1;
identifier f;
statement S1,S2,S3;
@@@M

@+* if (E == NULL)
{
  ... when != if (E == NULL) S1 else S2
      when != E = E1
@+* E->f
  ... when any
  return ...;
}
else S3
\end{lstlisting}\\
\end{tabular}

\vspace{1cm}

\begin{tabular}{c}
  Matched lines\\

\begin{lstlisting}[language=PatchC]
foo = kmalloc(1024);
@-if (!foo) {
@-  printk ("Error %s", foo->here);
  return;
}
foo->ok = 1;
return;
\end{lstlisting}\\
\end{tabular}

\newpage
\subsection{Reference counter: the of\_xxx API}

Coccinelle can embed Python code. Python code is used inside special
SmPL rule annotated with \texttt{script:python}.  Python rules inherit
metavariables, such as identifier or token positions, from other SmPL
rules. The inherited metavariables can then be manipulated by Python
code.

The following semantic match looks for a call to the
\texttt{of\_find\_node\_by\_name} function. This call increments a
counter which must be decremented to release the resource. Then, when
there is no call to \texttt{of\_node\_put}, no new assignment to the
\texttt{device\_node} variable \texttt{n} and a \texttt{return}
statement is reached, a bug is detected and the position \texttt{p1}
and \texttt{p2} are initialized. As the Python only depends on the
positions \texttt{p1} and \texttt{p2}, it is evaluated. In the
following case, some emacs Org mode data are produced.  This example
illustrates the various fields that can be accessed in the Python code from
a position variable.

\begin{tabular}{c}
\begin{lstlisting}[language=Cocci,breaklines=true]
@M@ r exists @
local idexpression struct device_node *n;
position p1, p2;
statement S1,S2;
expression E,E1;
@@@M

(
if (!(n@p1 = of_find_node_by_name(...))) S1
|
n@p1 = of_find_node_by_name(...)
)
<... when != of_node_put(n)
    when != if (...) { <+... of_node_put(n) ...+> }
    when != true !n  || ...
    when != n = E
    when != E = n
if (!n || ...) S2
...>
(
  return <+...n...+>;
|
return@p2 ...;
|
n = E1
|
E1 = n
)

@M@ script:python @
p1 << r.p1;
p2 << r.p2;
@@@M

print "* TODO [[view:%s::face=ovl-face1::linb=%s::colb=%s::cole=%s][inc. counter:%s::%s]]" % (p1[0].file,p1[0].line,p1[0].column,p1[0].column_end,p1[0].file,p1[0].line)
print "[[view:%s::face=ovl-face2::linb=%s::colb=%s::cole=%s][return]]" % (p2[0].file,p2[0].line,p2[0].column,p2[0].column_end)
\end{lstlisting}
\end{tabular}


\newpage

Lines 13 to 17 list a variety of constructs that should not appear
between a call to \texttt{of\_find\_node\_by\_name} and a buggy return
site. Examples are a call to \texttt{of\_node\_put} (line 13) and a
transition into the then branch of a conditional testing whether
\texttt{n} is \texttt{NULL} (line 15). Any number of conditionals
testing whether \texttt{n} is \texttt{NULL} are allowed as indicated
by the use of a nest \texttt{<...~~...>} to describe the path between
the call to \texttt{of\_find\_node\_by\_name}, the return and the
conditional in the pattern on line 18.\\

The previously semantic match has been used to generate the following
lines. They may be edited using the emacs Org mode to navigate in the code
from a site to another.

\begin{lstlisting}[language=,breaklines=true]
* TODO [[view:/linux-next/arch/powerpc/platforms/pseries/setup.c::face=ovl-face1::linb=236::colb=18::cole=20][inc. counter:/linux-next/arch/powerpc/platforms/pseries/setup.c::236]]
[[view:/linux-next/arch/powerpc/platforms/pseries/setup.c::face=ovl-face2::linb=250::colb=3::cole=9][return]]
* TODO [[view:/linux-next/arch/powerpc/platforms/pseries/setup.c::face=ovl-face1::linb=236::colb=18::cole=20][inc. counter:/linux-next/arch/powerpc/platforms/pseries/setup.c::236]]
[[view:/linux-next/arch/powerpc/platforms/pseries/setup.c::face=ovl-face2::linb=245::colb=3::cole=9][return]]
\end{lstlisting}

Note~: Coccinelle provides some predefined Python functions,
\emph{i.e.}, \texttt{cocci.print\_main}, \texttt{cocci.print\_sec} and
\texttt{cocci.print\_secs}. One could alternatively write the following
SmPL rule instead of the previously presented one.

\begin{tabular}{c}
\begin{lstlisting}[language=Cocci]
@M@ script:python @
p1 << r.p1;
p2 << r.p2;
@@@M

cocci.print_main(p1)
cocci.print_sec("return",p2)
\end{lstlisting}
\end{tabular}\\

The function \texttt{cocci.print\_secs} is used when there are several
positions which are matched by a single position variable and that
every matched position should be printed.

Any metavariable could be inherited in the Python code. However,
accessible fields are not currently equally supported among them.

% \begin{tabular}{ccc}
% Before & Semantic patch & After \\
% \begin{minipage}[t]{.3\linewidth}
% \begin{lstlisting}
% \end{lstlisting}
% \end{minipage}
% &
% \begin{minipage}[t]{.3\linewidth}
% \begin{lstlisting}[language=Cocci]
% \end{lstlisting}
% \end{minipage}
% &
% \begin{minipage}[t]{.3\linewidth}
% \begin{lstlisting}
% \end{lstlisting}
% \end{minipage}\\
% \end{tabular}

%%% Local Variables:
%%% mode: LaTeX
%%% TeX-master: "cocci_syntax"
%%% coding: latin-9
%%% TeX-PDF-mode: t
%%% ispell-local-dictionary: "american"
%%% End:


\section{Tips and Tricks}

This section presents some tips and tricks for using Coccinelle.

\subsection{How to remove useless parentheses?}

If you want to rewrite any access to a pointer value by a function
call, you may use the following semantic patch.

\begin{lstlisting}[language=Cocci]
@-- a = *b
@++ a = readb(b)
\end{lstlisting}

However, if for some reason your code looks like \verb|bar = *(foo)|,
you will end up with \verb|bar = readb((foo))| as the extra
parentheses around \texttt{foo} are capture by the metavariable
\texttt{b}.

In order to generate better output code, you can use the following
semantic patch instead.
\begin{lstlisting}[language=Cocci]
@-- a = *(b)
@++ a = readb(b)
\end{lstlisting}

\noindent
And rely on your standard.iso isomorphism file which should contain:
\begin{lstlisting}[language=Cocci]
Expression
@ paren @
expression E;
@@

 (E) => E
\end{lstlisting}

Coccinelle will then consider \verb|bar = *(foo)| as equivalent to
\verb|bar = *foo| (but not the other way around) and capture both.
Finally, it will generate \verb|bar = readb(foo)| as expected.

%%% Local Variables:
%%% mode: LaTeX
%%% TeX-master: "main_grammar"
%%% coding: utf-8
%%% TeX-PDF-mode: t
%%% ispell-local-dictionary: "american"
%%% End:



\chapter{Isomorphisms and \texttt{standard.iso}}


%%% Local Variables:
%%% mode: LaTeX
%%% coding: utf-8
%%% TeX-PDF-mode: t
%%% ispell-local-dictionary: "american"
%%% End:


\chapter{Parsing C, \cpp, and \texttt{standard.h}}


\chapter{Developing a semantic patch}

%editing semantic patch, emacs mode



\chapter{Advanced features}





\part{Reference Manual}
\label{part:refmanual}

\chapter{SmPL grammar}

\documentclass{article}
\usepackage{hevea}
\usepackage{fullpage}
\usepackage{alltt}
\usepackage{xspace}
\usepackage[pdfborder={0 0 0}]{hyperref}
\usepackage{listings}
\usepackage[usenames,dvipsnames]{color}
\usepackage[T1]{fontenc}

\lstset{basicstyle=\ttfamily,numbers=left, numberstyle=\tiny, stepnumber=1, numbersep=5pt,language=C,commentstyle=\color{OliveGreen},keywordstyle=\color{blue},stringstyle=\color{BrickRed}}

% You must prefix the +/- lines of
% cocci files with @+/@- respectively.
% This will enable the automatic coloration.
\ifhevea
\lstdefinelanguage{Cocci}{
morekeywords={idexpression,expression,statement,identifier,
parameter,list,when,strict,any,forall,exists},
keywordstyle=\color{Bittersweet}\bfseries,
sensitive=false
}
\else
\lstdefinelanguage{Cocci}{
morekeywords={idexpression,expression,statement,identifier,
parameter,list,when,strict,any,forall,exists},
keywordstyle=\color{Bittersweet}\bfseries,
sensitive=false,
morecomment=*[s][\color{OliveGreen}]{@}{@@},
morecomment=*[s][\color{OliveGreen}]{@@}{@@},
moredelim=[il][\color{blue}]{@+},
moredelim=[il][\color{BrickRed}]{@-}
}
\fi

\newif\iflanguagestyle
\languagestylefalse
% Definition of a grammar (BNF style) package for Latex and Hevea


\ifhevea
% Definition for Hevea (HTML generation)
\def\T#1{{\sf #1}}
\def\NTS#1{{\maroon #1\/}}
\def\KW#1{{\blue #1}}
\def\gramor{{\black $|$}}   
\def\grameq{{\black \quad::=\quad}}
\def\lparen{{\black (}}
\def\rparen{{\black )}}
\def\lbracket{{\black [}}
\def\rbracket{{\black ]}}
\def\plus{{\black +}}
\def\questionmark{{\black ?}}
\def\etoile{{\black *}}
\else
% Definition for LaTeX
\def\T#1{{\sf #1}}
\def\NTS#1{{\it #1\/}}
\def\KW#1{{\mtt{#1}}}
%\def\gramor{$\vert$}
\def\gramor{$\mid$}
\def\grameq{\,\,\,::=\,\,\,\,\,}
\def\lparen{(}
\def\rparen{)}
\def\lbracket{[}
\def\rbracket{]}
\def\plus{+}
\def\questionmark{?}
\def\etoile{*}
\fi

\def\NT#1{\hyperlink{#1}{\NTS{#1}}}
\def\group#1{{\rm\lparen}#1{\rm\rparen}}
\def\range#1#2{#1{..}#2}
\def\any#1{#1$^{\etoile}$}
\def\some#1{#1$^{\plus}$}
\def\ANY#1{\any{{\rm\lparen}#1{\rm\rparen}}}
\def\SOME#1{\some{{\rm\lparen}#1{\rm\rparen}}}
\def\OR{\gramor\ }

\iflanguagestyle
% Option notation : [ xxx ] versus (xxx)^?
\def\opt#1{#1$^{\questionmark}$}
\def\OPT#1{\opt{{\rm\lparen}#1{\rm\rparen}}}
\else
\def\opt#1{{\lbracket}#1{\rbracket}}
\def\OPT#1{\opt{#1}}
\fi

\newenvironment{grammar}{\begin{center}\begin{tabular}{l@{}c@{}l}}{\end{tabular}\end{center}}
\def\RULE#1\CASE#2{\hypertarget{#1}{\NTS{#1}} & \grameq & \KW{#2} \\}
\def\CASE#1{& \gramor & \KW{#1} \\}


\def\lb{\char123}
\def\rb{\char125}


\newcommand{\sizecodebis}[0]{\scriptsize}

\newcommand{\mita}[1]{\mbox{\it{{#1}}}}
\newcommand{\mtt}[1]{\mbox{\tt{{#1}}}}
\newcommand{\msf}[1]{\mbox{\sf{{#1}}}}
\newcommand{\stt}[1]{\mbox{\scriptsize\tt{{#1}}}}
\newcommand{\ssf}[1]{\mbox{\scriptsize\sf{{#1}}}}
\newcommand{\sita}[1]{\mbox{\scriptsize\it{{#1}}}}
\newcommand{\mrm}[1]{\mbox{\rm{{#1}}}}
\newcommand{\mth}[1]{\({#1}\)}
\newcommand{\entails}[2]{\begin{array}{@{}c@{}}{#1}\\\hline{#2}\end{array}}
\newcommand{\ttlb}{\mbox{\tt \char'173}}
\newcommand{\ttrb}{\mbox{\tt \char'175}}
\newcommand{\ttmid}{\mbox{\tt \char'174}}
\newcommand{\tttld}{\mbox{\tt \char'176}}

\newcommand{\fixme}[1]{{\color{red} #1}}

\ifhevea
\newcommand{\phantom}{}
\newcommand{\air}{   }
\else
\newcommand{\air}{\phantom{xxx}}
\fi

\title{The SmPL Grammar}
\author{Research group on Coccinelle}
\date{\today}

\begin{document}
\maketitle

%\section{The SmPL Grammar}

% This section presents the SmPL grammar.  This definition follows closely
% our implementation using the Menhir parser generator \cite{menhir}.

The grammar uses some rules where the left-hand side is in all capital
letters.  These are macros, which take one or more grammar rule
right-hand-sides as arguments.  The grammar also uses some unspecified
nonterminals, such as {\sf id}, {\sf const}, etc.  These refer to the
sets suggested by the name, {\em i.e.}, {\sf id} refers to the set of
possible C-language identifiers, while {\sf const} refers to the set
of possible C-language constants. \ifhevea A PDF version of this
documention is available at
\url{http://localhost:8080/coccinelle/cocci_syntax.pdf}.\else A HTML
version of this documention is online at
\url{http://localhost:8080/coccinelle/cocci_syntax.html}. \fi

\section{Program}

\begin{grammar}
  \RULE{\rt{program}}
  \CASE{\any{\NT{include}} \some{\NT{changeset}}}

  \RULE{\rt{include}}
  \CASE{using "\NT{string}"}
  \CASE{using <\NT{pathToIsoFile}>}

  \RULE{\rt{changeset}}
  \CASE{\NT{metavariables} \ANY{--- filename +++ filename} \NT{transformation}}

\end{grammar}

Between the metavariables and the transformation rule, there can be a
specification of constraints on the names of the old and new files,
analogous to the filename specifications in the standard patch syntax.
%(see Figure \ref{scsiglue_patch}).

\section{Metavariables}

Fresh metavariables must only be used in {\tt +} code.  Metavariables must
occur at least once in the transformation immediately following their
declaration.  These properties are not expressed in the grammar, but are
checked by a subsequent analysis.

\begin{grammar}
  \RULE{\rt{metavariables}}
  \CASE{@@ \any{\NT{metadecl}} @@}
  \CASE{@ \NT{rulename} @ \any{\NT{metadecl}} @@}

  \RULE{\rt{rulename}}
  \CASE{\T{id} \OPT{extends \T{id}} \OPT{depends on \NT{dep}} \opt{\NT{iso}} \opt{\NT{disable}} \opt{\NT{exists}} \opt{expression}}
  \CASE{script:\T{language} \OPT{depends on \NT{dep}}}

  \RULE{\rt{dep}}
  \CASE{\NT{pnrule}}
  \CASE{\NT{dep} \&\& \NT{dep}}
  \CASE{\NT{dep} || \NT{dep}}

  \RULE{\rt{pnrule}}
  \CASE{\T{id}}
  \CASE{!\T{id}}
  \CASE{ever \T{id}}
  \CASE{never \T{id}}
  \CASE{(\NT{dep})}

  \RULE{\rt{iso}}
  \CASE{using "\T{str}" \ANY{, "\T{str}"}}

  \RULE{\rt{disable}}
  \CASE{disable \T{id} \ANY{, \T{id}}}

  \RULE{\rt{exists}}
  \CASE{exists}
  \CASE{\opt{reverse} forall}
\end{grammar}

\begin{grammar}
  \RULE{\rt{metadecl}}
  \CASE{fresh identifier \NT{ids} ;}
  \CASE{parameter \opt{list} \NT{ids} ;}
  \CASE{expression list \NT{ids} ;}
  \CASE{type \NT{ids} ;}
  \CASE{statement \opt{list} \NT{ids} ;}
  \CASE{typedef \NT{ids} ;}
  \CASE{declarer name \NT{ids} ;}
  \CASE{iterator name \NT{ids} ;}
  \CASE{identifier \NT{pmid\_with\_not\_eq\_list} ;}
  \CASE{\opt{local} function \NT{pmid\_with\_not\_eq\_list} ;}
  \CASE{declarer \NT{pmid\_with\_not\_eq\_list} ;}
  \CASE{iterator \NT{pmid\_with\_not\_eq\_list} ;}
  \CASE{error \NT{pmid\_with\_not\_eq\_list} ; }
  \CASE{\opt{local} idexpression \opt{\NT{ctype}} \NT{pmid\_with\_not\_eq\_list} ;}
  \CASE{\opt{local} idexpression \OPT{\ttlb \NT{ctypes} \ttrb \any{*}} \NT{pmid\_with\_not\_eq\_list} ;}
  \CASE{\opt{local} idexpression \some{*} \NT{pmid\_with\_not\_eq\_list} ;}
  \CASE{expression \some{*} \NT{pmid\_with\_not\_eq\_list} ;}
  \CASE{\NT{ctype} [ ] \NT{pmid\_with\_not\_eq\_list} ;}
  \CASE{\ttlb \NT{ctypes} \ttrb \any{*} [ ] \NT{pmid\_with\_not\_eq\_list} ;}
  \CASE{constant \opt{\NT{ctype}} \NT{pmid\_with\_not\_eq\_list} ;}
  \CASE{constant \OPT{\ttlb \NT{ctypes} \ttrb \any{*}} \NT{pmid\_with\_not\_eq\_list} ;}
  \CASE{expression \NT{pmid\_with\_not\_ceq\_list} ;}
  \CASE{\NT{ctype} \NT{pmid\_with\_not\_ceq\_list} ;}
  \CASE{\ttlb \NT{ctypes} \ttrb \any{*} \NT{pmid\_with\_not\_ceq\_list} ;}
  \CASE{position \opt{any} \NT{pmid\_with\_not\_eq\_mid\_list} ;}
  \CASE{parameter list [ ident ] \NT{ids} ;}
  \CASE{expression list [ ident ] \NT{ids} ;}
\end{grammar}

\begin{grammar}
  \RULE{\rt{ids}}
  \CASE{\NT{pmid} \ANY{, \NT{pmid}}}

  \RULE{\rt{pmid}}
  \CASE{\T{id}}
  \CASE{\NT{mid}}
  \CASE{list}
  \CASE{error}
  \CASE{type}

  \RULE{\rt{mid}}  \CASE{\T{rulename\_id}.\T{id}}

  \RULE{\rt{pmid\_with\_not\_eq\_list}}
  \CASE{\NT{pmid\_with\_not\_eq} \ANY{, \NT{pmid\_with\_not\_eq}}}

  \RULE{\rt{pmid\_with\_not\_eq}}
  \CASE{\NT{pmid} \OPT{!= \T{id}}}
  \CASE{\NT{pmid} \OPT{!= \ttlb \T{id} \ANY{, \T{id}} \ttrb}}

  \RULE{\rt{pmid\_with\_not\_ceq\_list}}
  \CASE{\NT{pmid\_with\_not\_ceq} \ANY{, \NT{pmid\_with\_not\_ceq}}}

  \RULE{\rt{pmid\_with\_not\_ceq}}
  \CASE{\NT{pmid} \OPT{!= \NT{id\_or\_cst}}}
  \CASE{\NT{pmid} \OPT{!= \ttlb \NT{id\_or\_cst} \ANY{, \NT{id\_or\_cst}} \ttrb}}

  \RULE{\rt{id\_or\_cst}}
  \CASE{\T{id}}
  \CASE{\T{integer}}

  \RULE{\rt{pmid\_with\_not\_eq\_mid\_list}}
  \CASE{\NT{pmid\_with\_not\_eq\_mid} \ANY{, \NT{pmid\_with\_not\_eq\_mid}}}

  \RULE{\rt{pmid\_with\_not\_eq\_mid}}
  \CASE{\NT{pmid} \OPT{!= \NT{mid}}}
  \CASE{\NT{pmid} \OPT{!= \ttlb \NT{mid} \ANY{, \NT{mid}} \ttrb}}
\end{grammar}

Subsequently, we refer to arbitrary metavariables as
\mth{\msf{metaid}^{\mbox{\scriptsize{\it{ty}}}}}, where {\it{ty}} indicates
the {\it metakind} used in the declaration of the variable.  For example,
\mth{\msf{metaid}^{\ssf{Type}}} refers to a metavariable that stands for
any type.

The {\it type} nonterminal is used by both the grammar of metavariable
declarations and the grammar of transformations, and is defined on
page~\pageref{types}.

\section{Transformation}

The grammar of the transformation is not actually the grammar of the SmPL
code that can be written by the programmer, but the grammar of the slice of
this consisting of the {\tt -} annotated and the unannotated code (the
context of the transformed lines), or the {\tt +} annotated code and the
unannotated code.  For example, for parsing purposes, the transformation
%presented in Section \ref{sec:seq2}
is split into the two variants shown below and each is parsed
separately.

\begin{center}
\begin{tabular}{c}
\begin{lstlisting}[language=Cocci]
  proc_info_func(...) {
    <...
@--    hostno
@++    hostptr->host_no
    ...>
 }
\end{lstlisting}\\
\end{tabular}
\end{center}

{%\sizecodebis
\begin{center}
\begin{tabular}{p{5cm}p{3cm}p{5cm}}
\begin{lstlisting}[language=Cocci]
  proc_info_func(...) {
    <...
@--    hostno
    ...>
 }
\end{lstlisting}
&&
\begin{lstlisting}[language=Cocci]
  proc_info_func(...) {
    <...
@++    hostptr->host_no
    ...>
 }
\end{lstlisting}
\end{tabular}
\end{center}
}

\noindent
Requiring that both slices parse correctly ensures that the rule matches
syntactically valid C code and that it produces syntactically valid C code.
The generated parse trees are then merged for use in the subsequent
matching and transformation process.

The grammar rule for the minus or plus slice of a transformation is as follows:

\begin{grammar}

  \RULE{\rt{transformation}}
  \CASE{\NT{fundecl}}
  \CASE{\NT{ctype}}
  \CASE{\ttlb \NT{initialize\_list} \ttrb}
  \CASE{\NT{toplevel\_seq\_start\_after\_dots\_init}}

  \RULE{\rt{toplevel\_seq\_start\_after\_dots\_init}}
  \CASE{\NT{stmt\_dots} \NT{toplevel\_after\_dots}}
  \CASE{\NT{expr} \opt{\NT{toplevel\_after\_exp}}}
  \CASE{\NT{decl\_stmt\_expr} \opt{\NT{toplevel\_after\_stmt}}}

  \RULE{\rt{stmt\_dots}}
  \CASE{... \any{\NT{whenppdecls}}}
  \CASE{<... \any{\NT{whenppdecls}} \NT{nest\_after\_dots} ...>}
  \CASE{<+... \any{\NT{whenppdecls}} \NT{nest\_after\_dots} ...+>}

  \RULE{\rt{whenppdecls}}
  \CASE{when != \NT{when\_start}$^\dag$}
  \CASE{when = \NT{rule\_elem\_stmt}$^\dag$}
  \CASE{when \NT{any\_strict} \ANY{, \NT{any\_strict}} $^\dag$}
  \CASE{when true != \NT{exp} $^\ddag$}
  \CASE{when false != \NT{exp} $^\ddag$}

  \RULE{\rt{any\_strict}}
  \CASE{any}
  \CASE{strict}
  \CASE{forall}
  \CASE{exists}

  \RULE{\rt{nest\_after\_dots}}
  \CASE{\NT{decl\_stmt\_exp} \opt{\NT{nest\_after\_stmt}}}
  \CASE{\opt{\NT{exp}} \opt{\NT{nest\_after\_exp}}}

  \RULE{\rt{nest\_after\_stmt}}
  \CASE{\NT{stmt\_dots} \NT{nest\_after\_dots}}
  \CASE{\NT{decl\_stmt} \opt{\NT{nest\_after\_stmt}}}

  \RULE{\rt{nest\_after\_exp}}
  \CASE{\NT{stmt\_dots} \NT{nest\_after\_dots}}

  \RULE{\rt{toplevel\_after\_dots}}
  \CASE{\opt{\NT{toplevel\_after\_exp}}}
  \CASE{\NT{exp} \opt{\NT{toplevel\_after\_exp}}}
  \CASE{\NT{decl\_stmt\_expr} \NT{toplevel\_after\_stmt}}

  \RULE{\rt{toplevel\_after\_exp}}
  \CASE{\NT{stmt\_dots} \opt{\NT{toplevel\_after\_dots}}}

  \RULE{\rt{decl\_stmt\_expr}}
  \CASE{TMetaStmList$^\ddag$}
  \CASE{\NT{decl\_var}}
  \CASE{\NT{stmt}}
  \CASE{(\NT{fun\_start} \ANY{| \NT{fun\_start}})}

  \RULE{\rt{toplevel\_after\_stmt}}
  \CASE{\NT{stmt\_dots} \opt{\NT{toplevel\_after\_dots}}}
  \CASE{\NT{decl\_stmt} \NT{toplevel\_after\_stmt}}

\end{grammar}

$^\dag$ Note
$^\ddag$ Check and fix me

% \noindent{\footnotesize\begin{tabular}{r@{\,\,\,}c@{\,\,\,}l}
% \mita{transformation} & ::= &
%    \begin{tabular}[t]{@{}l}
%    \ANY{\mtt{\#include} \msf{include\_string}} \\
%    \opt{OPTDOTSEQ(\some{\mita{fun\_decl\_statement}} \(\mid\) \mita{expr},
%    \mita{stmt\_whencode})}\end{tabular} \\
% \mita{fun\_decl\_statement} & ::= & \mita{decl\_statement} \(\mid\)
% \mita{fun\_decl}
% \end{tabular}

% \begin{grammar}
%   \PRULE{OPTDOTSEQ(\NT{grammar},\NT{whencode})}
%   \CASE{\opt{... \opt{\NT{whencode}}} \NT{grammar}
%  \ANY{... \opt{\NT{whencode}} \NT{grammar}}
%  \opt{... \opt{\NT{whencode}}}}
% \end{grammar}

\noindent
Lines may be annotated with an element of the set $\{\mtt{-},
\mtt{+}\}$ or an element of the set $\{\mtt{*}, \mtt{?}\}$, or one of
each. \mtt{?} and \mtt{*} represent respectively at most one, and at
least one match of the given pattern.  There are some constraints on
the use of these annotations:
\begin{itemize}
\item Dots, {\em i.e.} \texttt{...}, cannot occur on a line marked
  \texttt{+}.
\item Nested dots, {\em i.e.} dots enclosed in {\tt <} and {\tt >}, cannot
  occur on a line with any marking.
\end{itemize}


\section{Types}
\label{types}

\begin{grammar}

  \RULE{\rt{ctypes}}
  \CASE{\NT{ctype} \ANY{, \NT{ctype}}}

  \RULE{\rt{ctype}}
  \CASE{\opt{\NT{const\_vol}} \NT{generic\_ctype} \any{*}}
  \CASE{\opt{\NT{const\_vol}} void \some{*}}
  \CASE{(\NT{ctype} \ANY{| \NT{ctype}})}

  \RULE{\rt{const\_vol}}
  \CASE{const}
  \CASE{volatile}

  \RULE{\rt{generic\_ctype}}
  \CASE{\NT{ctype\_qualif}}
  \CASE{\opt{\NT{ctype\_qualif}} char}
  \CASE{\opt{\NT{ctype\_qualif}} short}
  \CASE{\opt{\NT{ctype\_qualif}} int}
  \CASE{\opt{\NT{ctype\_qualif}} long}
  \CASE{double}
  \CASE{float}
  \CASE{\OPT{struct\OR union} \T{id} \OPT{\{ \any{\NT{struct\_decl\_list}} \}}}

  \RULE{\rt{ctype\_qualif}}
  \CASE{unsigned}
  \CASE{signed}

  \RULE{\rt{struct\_decl\_list}}
  \CASE{\NT{struct\_decl\_list\_start}}

  \RULE{\rt{struct\_decl\_list\_start}}
  \CASE{\NT{struct\_decl}}
  \CASE{\NT{struct\_decl} \NT{struct\_decl\_list\_start}}
  \CASE{... \opt{when != \NT{struct\_decl}}$^\dag$ \opt{\NT{continue\_struct\_decl\_list}}}

  \RULE{\rt{continue\_struct\_decl\_list}}
  \CASE{\NT{struct\_decl} \NT{struct\_decl\_list\_start}}
  \CASE{\NT{struct\_decl}}

  \RULE{\rt{struct\_decl}}
  \CASE{\NT{ctype} \NT{d\_ident};}
  \CASE{\NT{fn\_ctype} (* \NT{d\_ident}) (\NT{});)}
  \CASE{\opt{\NT{const\_vol}} \NT{pure\_ident} \NT{d\_ident};}

  \RULE{\rt{d\_ident}}
  \CASE{\NT{ident} \any{[\opt{\NT{eexpr}}]}}

  \RULE{\rt{fn\_ctype}}
  \CASE{\NT{generic\_ctype} \any{*}}
  \CASE{void \any{*}}
\end{grammar}

$^\dag$ The optional \texttt{when} construct ends at the end of the line.

% \noindent{\footnotesize\begin{tabular}{r@{\,\,\,}c@{\,\,\,}l}
% \mita{type} & ::= & \opt{\mtt{const} \(\mid\) \mtt{volatile}}
%                     \mita{type\_desc} \ANY{\mtt{*}}\\
% \mita{type\_desc} & ::= & \mita{simple\_type} \(\mid\) \opt{\mtt{signed} \(\mid\)
%        \mtt{unsigned}} \mita{signable\_type} \(\mid\) \opt{\mtt{struct} \(\mid\)
%        \mtt{union}} \msf{id}
% \\&\multicolumn{1}{r}{\(\mid\)}&
%        \mth{\msf{metaid}^{\ssf{Type}}}
% \end{tabular}}

% \noindent{\footnotesize\begin{tabular}{r@{\,\,\,}c@{\,\,\,}l}
% \mita{simple\_type} & ::= & \mtt{void} \(\mid\) \mtt{double} \(\mid\)
%  \mtt{float} \\
% \mita{signable\_type} & ::= & \mtt{char} \(\mid\) \mtt{short} \(\mid\)
%  \mtt{int} \(\mid\) \mtt{long}
% \end{tabular}}

\section{Function declarations}

\begin{grammar}

  \RULE{\rt{fundecl}}
  \CASE{\opt{static} \NT{funid}
    (\opt{\NT{PARAMSEQ}(\NT{param},\mth{\varepsilon})})
    \ttlb~\opt{\NT{stmt\_seq}} \ttrb}

  \RULE{\rt{funid}}
  \CASE{\T{id}}
  \CASE{\mth{\T{metaid}^{\ssf{Func}}}}
  \CASE{\mth{\T{metaid}^{\ssf{LocalFunc}}}}

  \RULE{\rt{param}}
  \CASE{\NT{type} \T{id}}
  \CASE{\mth{\T{metaid}^{\ssf{Param}}}}
  \CASE{\mth{\T{metaid}^{\ssf{ParamList}}}}
\end{grammar}

\begin{grammar}
  \RULE{\rt{PARAMSEQ}(\NT{grammar},\NT{whencode})}
  \CASE{\mth{(}\NT{grammar}\OR \ldots \opt{\NT{whencode}}\mth{)} \ANY{, \NT{grammar}\OR , \ldots \opt{\NT{whencode}}}}
\end{grammar}

%\newpage

\section{Declarations}

\begin{grammar}
  \RULE{\rt{decl}}
  \CASE{\NT{type} \opt{\NT{id} \opt{[\opt{\NT{dot\_expr}}]}
      \ANY{, \NT{id} \opt{[ \opt{\NT{dot\_expr}}]}}};}
  \CASE{\NT{type} \NT{id} \opt{[\opt{\NT{dot\_expr}}]}= \NT{nest\_expr};}
\end{grammar}

% \noindent{\footnotesize\begin{tabular}{r@{\,\,\,}c@{\,\,\,}l}
% \mita{decl\_var} & ::= &
%    \mita{type} \opt{\ANY{\mita{id} \opt{\mtt{[} \opt{\mita{dot\_expr}}
%    \mtt{]}} \mtt{,}} \mita{id} \opt{\mtt{[} \opt{\mita{dot\_expr}} \mtt{]}}}
%    \mtt{;}
% \\&\multicolumn{1}{r}{\(\mid\)}&
%  \mita{type} \mita{id} \opt{\mtt{[} \opt{\mita{dot\_expr}} \mtt{]}}
%    \mtt{=} \mita{nest\_expr} \mtt{;}
% \end{tabular}}

\section{Statements}

The first rule {\em statement} describes the various forms of a statement.
The remaining rules implement the constraints that are sensitive to the
context in which the statement occurs: {\em single\_statement} for a
context in which only one statement is allowed, and {\em decl\_statement}
for a context in which a declaration, statement, or sequence thereof is
allowed.

%\vspace{\baselineskip}
\begin{grammar}
  \RULE{\rt{stmt}}
  \CASE{\NT{includes}}
  \CASE{\mth{\T{metaid}^{\ssf{Stmt}}}}
  \CASE{\NT{expr};}
  \CASE{if (\NT{dot\_expr}) \NT{single\_stmt} \opt{else \NT{single\_stmt}}}
  \CASE{for (\opt{\NT{dot\_expr}}; \opt{\NT{dot\_expr}}; \opt{\NT{dot\_expr}})
    \NT{single\_stmt}}
  \CASE{while (\NT{dot\_expr}) \NT{single\_stmt}}
  \CASE{do \NT{single\_stmt} while (\NT{dot\_expr});}
  \CASE{\NT{iter\_ident} (\any{\NT{dot\_expr}}) \NT{single\_stmt}}
  \CASE{switch (\opt{\NT{dot\_expr}}) \ttlb \any{\NT{case\_line}} \ttrb}
  \CASE{return \opt{\NT{dot\_expr}};}
  \CASE{\ttlb~\opt{\NT{stmt\_seq}} \ttrb}
  \CASE{\NT{NEST}(\some{\NT{decl\_stmt}}, \NT{stmt\_whencode})}
  \CASE{\NT{NEST}(\NT{expr}, \NT{stmt\_whencode})}
  \CASE{break;}
  \CASE{continue;}
  \CASE{\NT{ident}:}
  \CASE{goto \NT{ident};}
  \CASE{\ttlb \NT{fun\_start} \ttrb}

% \noindent{\footnotesize\begin{tabular}{r@{\,\,\,}c@{\,\,\,}l}
% \mita{statement} & ::= &
%   \mth{\msf{metaid}^{\ssf{Stmt}}}
% \\&\multicolumn{1}{r}{\(\mid\)}&
%   \mita{expr} \mtt{;}
% \\&\multicolumn{1}{r}{\(\mid\)}&
%   \mtt{if} \mtt{(} \mita{dot\_expr} \mtt{)} \mita{single\_statement}
%   \opt{\mtt{else} \mita{single\_statement}}
% \\&\multicolumn{1}{r}{\(\mid\)}&
%   \mtt{for} \mtt{(} \opt{\mita{dot\_expr}} \mtt{;} \opt{\mita{dot\_expr}} \mtt{;}
%   \opt{\mita{dot\_expr}} \mtt{)} \mita{single\_statement}
% \\&\multicolumn{1}{r}{\(\mid\)}&
%   \mtt{while} \mtt{(} \mita{dot\_expr} \mtt{)} \mita{single\_statement}
% \\&\multicolumn{1}{r}{\(\mid\)}&
%   \mtt{do} \mita{single\_statement} \mtt{while} \mtt{(} \mita{dot\_expr} \mtt{)}
%   \mtt{;}
% \\&\multicolumn{1}{r}{\(\mid\)}&
%   \mtt{return} \opt{\mita{dot\_expr}} \mtt{;}
% \\&\multicolumn{1}{r}{\(\mid\)}&
%   \ttlb~\opt{\mita{statement\_sequence}} \ttrb
% \\&\multicolumn{1}{r}{\(\mid\)}&
%   NEST(\some{\mita{decl\_statement}} \(\mid\) \mita{expr}, \mita{stmt\_whencode})
% \end{tabular}}

  \RULE{\rt{single\_stmt}}
  \CASE{\NT{stmt}}
  \CASE{\NT{OR}(\NT{stmt})}

  \RULE{\rt{decl\_stmt}}
  \CASE{\mth{\T{metaid}^{\ssf{StmtList}}}}
  \CASE{\NT{decl\_var}}
  \CASE{\NT{stmt}}
  \CASE{\NT{OR}(\NT{stmt\_seq})}

  \RULE{\rt{stmt\_seq}}
  \CASE{\any{\NT{decl\_stmt}}
    \opt{\NT{DOTSEQ}(\some{\NT{decl\_stmt}},
      \NT{stmt\_whencode}) \any{\NT{decl\_stmt}}}}
  \CASE{\any{\NT{decl\_stmt}}
    \opt{\NT{DOTSEQ}(\NT{expr},
      \NT{stmt\_whencode}) \any{\NT{decl\_stmt}}}}

  \RULE{\rt{stmt\_whencode}}
  \CASE{when != \NT{XXXDOTSEQXXX}(\some{\NT{decl\_stmt}}, \NT{stmt\_whencode})}
  \CASE{when != \NT{XXXDOTSEQXXX}(\NT{expr}, \NT{stmt\_whencode})}

  \RULE{\rt{case\_line}}
  \CASE{default : \NT{fun\_start}}
  \CASE{case \NT{dot\_expr} : \NT{fun\_start}}

  \RULE{\rt{fun\_start}}
  \CASE{}

  \RULE{\rt{iter\_ident}}
  \CASE{\T{IteratorId}}
  \CASE{\T{MetaIterator}}
\end{grammar}

\noindent{\footnotesize\begin{tabular}{r@{\,\,\,}c@{\,\,\,}l}
% \mita{single\_statement} & ::= &
%      \mita{statement} \(\mid\) OR(\mita{statement})
% \\
% \mita{decl\_statement} & ::= &
%   \mth{\msf{metaid}^{\ssf{StmtList}}} \(\mid\)
%   \mita{decl\_var} \(\mid\) \mita{statement} \(\mid\)
%   OR(\mita{statement\_sequence})
% \\
% \mita{statement\_sequence} & ::= &\\
%   \multicolumn{3}{r}{\air\air\any{\mita{decl\_statement}}
%   \opt{DOTSEQ(\some{\mita{decl\_statement}} \(\mid\) \mita{expr},
%   \mita{stmt\_whencode}) \any{\mita{decl\_statement}}}}
% \\
\mita{stmt\_whencode} & ::= & \mtt{WHEN} \mtt{!=}
OPTDOTSEQ(\some{\mita{decl\_statement}} \(\mid\) \mita{expr},\mita{stmt\_whencode})
% \\
% OR(\mita{grammar}) & ::= &
%  \mtt{(} \mita{grammar} \ANY{\ttmid \mita{grammar}} \mtt{)}
\end{tabular}}

\begin{grammar}
  \RULE{\rt{OR}(\NT{grammar})}
  \CASE{( \NT{grammar} \ANY{\ttmid \NT{grammar}})}

  \RULE{\rt{DOTSEQ}(\NT{grammar},\NT{whencode})}
  \CASE{\ldots \opt{\NT{whencode}} \ANY{\NT{grammar} \ldots \opt{\NT{whencode}}}}

% \noindent{\footnotesize\begin{tabular}{r@{\,\,\,}c@{\,\,\,}l}
% DOTSEQ(\mita{grammar},\mita{whencode}) & ::= &
% \\ \multicolumn{3}{l}{\air\air\phantom{\(\mid\)}
%  \mtt{\ldots} \opt{\mita{whencode}} \ANY{\mita{grammar} \mtt{\ldots}
%  \opt{\mita{whencode}}}}
% \\ \multicolumn{3}{l}{\air\air\(\mid\)
%  \mtt{ooo} \opt{\mita{whencode}} \ANY{\mita{grammar} \mtt{ooo}
%  \opt{\mita{whencode}}}}
% \\ \multicolumn{3}{l}{\air\air\(\mid\)
%  \mtt{***} \opt{\mita{whencode}} \ANY{\mita{grammar} \mtt{***}
%  \opt{\mita{whencode}}}}
% \end{tabular}}

  \RULE{\rt{NEST}(\NT{grammar},\NT{whencode})}
  \CASE{<\ldots \NT{grammar} \ANY{\ldots \opt{\NT{whencode}} \NT{grammar}} \ldots>}
  \CASE{<+\ldots \NT{grammar} \ANY{\ldots \opt{\NT{whencode}} \NT{grammar}} \ldots+>}

% \noindent{\footnotesize\begin{tabular}{r@{\,\,\,}c@{\,\,\,}l}
% NEST(\mita{grammar},\mita{whencode}) & ::= &
% \\ \multicolumn{3}{l}{\air\air\phantom{\(\mid\)}
%  \mtt{<\ldots} \mita{grammar} \ANY{\ldots \opt{\mita{whencode}}
%  \mita{grammar}} \mtt{\ldots>}}
% \\ \multicolumn{3}{l}{\air\air\(\mid\)
%  \mtt{<ooo} \mita{grammar} \ANY{ooo \opt{\mita{whencode}} \mita{grammar}}
%  \mtt{ooo>}}
% \\ \multicolumn{3}{l}{\air\air\(\mid\)
%  \mtt{<***} \mita{grammar} \ANY{*** \opt{\mita{whencode}} \mita{grammar}}
%  \mtt{***>}}
% \end{tabular}}

\end{grammar}

%\vspace{\baselineskip}

\noindent
OR is a macro that generates a disjunction of patterns.  The three
tokens \T{(}, \T{\ttmid}, and \T{)} must appear in the leftmost
column, to differentiate them from the parentheses and bit-or tokens
that can appear within expressions (and cannot appear in the leftmost
column).  These tokens are furthermore different from (, \(\mid\), and
), which are part of the grammar metalanguage.

\section{Expressions}

A nest or a single ellipsis is allowed in some expression contexts, and
causes ambiguity in others.  For example, in a sequence \mtt{\ldots
\mita{expr} \ldots}, the nonterminal \mita{expr} must be instantiated as an
explicit C-language expression, while in an array reference,
\mtt{\mth{\mita{expr}_1} \mtt{[} \mth{\mita{expr}_2} \mtt{]}}, the
nonterminal \mth{\mita{expr}_2}, because it is delimited by brackets, can
be also instantiated as \mtt{\ldots}, representing an arbitrary expression.  To
distinguish between the various possibilities, we define three nonterminals
for expressions: {\em expr} does not allow either top-level nests or
ellipses, {\em nest\_expr} allows a nest but not an ellipsis, and {\em
dot\_expr} allows both.  The EXPR macro is used to express these variants
in a concise way.

%\vspace{\baselineskip}
\begin{grammar}
  \RULE{\rt{expr}}
  \CASE{\NT{EXPR}(\NT{expr})}

  \RULE{\rt{nest\_expr}}
  \CASE{\NT{EXPR}(\NT{nest\_expr})}
  \CASE{\NT{NEST}(\NT{nest\_expr}, \NT{exp\_whencode})}

  \RULE{\rt{dot\_expr}}
  \CASE{\NT{EXPR}(\NT{dot\_expr})}
  \CASE{\NT{NEST}(\NT{dot\_expr}, \NT{exp\_whencode})}
  \CASE{... \opt{\NT{exp\_whencode}}}

  \RULE{\rt{EXPR}(exp)}
  \CASE{\NT{exp} \NT{assign\_op} \NT{exp}}
  \CASE{\NT{exp}++}
  \CASE{\NT{exp}--}
  \CASE{\NT{unary\_op} \NT{exp}}
  \CASE{\NT{exp} \NT{bin\_op} \NT{exp}}
  \CASE{\NT{exp} ? \NT{dot\_expr} : \NT{exp}}
  \CASE{(\NT{type}) \NT{exp}}
  \CASE{\NT{exp} [\NT{dot\_expr}]}
  \CASE{\NT{exp} . \NT{id}}
  \CASE{\NT{exp} -> \NT{id}}
  \CASE{\NT{exp}(\opt{\NT{PARAMSEQ}(\NT{arg}, \NT{exp\_whencode})})}
  \CASE{\NT{id}}
  \CASE{\mth{\T{metaid}^{\ssf{Func}}}}
  \CASE{\mth{\T{metaid}^{\ssf{LocalFunc}}}}
  \CASE{\mth{\T{metaid}^{\ssf{Exp}}}}
  \CASE{\mth{\T{metaid}^{\ssf{Err}}}}
  \CASE{\mth{\T{metaid}^{\ssf{Const}}}}
  \CASE{\NT{const}}
  \CASE{(\NT{dot\_expr})}
  \CASE{\NT{OR}(\NT{exp})}

  \RULE{\rt{arg}}
  \CASE{\NT{nest\_expr}}
  \CASE{\mth{\T{metaid}^{\ssf{ExpList}}}}

  \RULE{\rt{exp\_whencode}}
  \CASE{when != \NT{dot\_expr}}

  \RULE{\rt{assign\_op}}
  \CASE{= \OR -= \OR += \OR *= \OR /= \OR \%=}
  \CASE{\&= \OR |= \OR \caret= \OR \lt\lt= \OR \gt\gt=}

  \RULE{\rt{bin\_op}}
  \CASE{* \OR / \OR \% \OR + \OR -}
  \CASE{\lt\lt \OR \gt\gt \OR \caret\xspace \OR \& \OR \ttmid}
  \CASE{< \OR > \OR <= \OR >= \OR == \OR != \OR \&\& \OR \ttmid\ttmid}

  \RULE{\rt{unary\_op}}
  \CASE{++ \OR -- \OR \& \OR * \OR + \OR - \OR !}

\end{grammar}

\section{Constant, Identifiers and Types for Transformations}

\begin{grammar}
  \RULE{const}
  \CASE{"\any{[\^{}"]}"}
  \CASE{[0-9]+}
  \CASE{\mth{\cdots}}

  \RULE{id}
  \CASE{\T{id} \OR \mth{\T{metaid}^{\ssf{Id}}}}

  \RULE{type}
  \CASE{\NT{ctype} \OR \mth{\T{metaid}^{\ssf{Type}}}}
\end{grammar}

\section{To be fixed}

\subsection{From Toplevel Transformation}

\begin{grammar}
  \RULE{initialize\_list}\CASE{}
  \RULE{TMetaStmList}\CASE{}
  \RULE{fun\_start}\CASE{Also in statements}
  \RULE{decl\_stmt}\CASE{}
  \RULE{exp}\CASE{}
  \RULE{when\_start}\CASE{}
  \RULE{rule\_elem\_stmt}\CASE{}
\end{grammar}

\subsection{From Types}

\begin{grammar}
  \RULE{ident}\CASE{}
  \RULE{pure\_ident}\CASE{}
  \RULE{eexpr}\CASE{}
\end{grammar}

\subsection{...}

\begin{grammar}
  \RULE{decl\_var}\CASE{From statements::decl\_stmt}
  \RULE{includes}\CASE{From statements}
\end{grammar}


\section{Examples}
%\label{sec:examples}

This section presents a range of examples.  Each
example is presented along with some C code to which it is
applied. The description explains the rules and the matching process.

\subsection{Function renaming}

One of the primary goals of Coccinelle is to perform software
evolution.  For instance, Coccinelle could be used to perform function
renaming. In the following example, every occurrence of a call to the
function \texttt{foo} is replaced by a call to the
function \texttt{bar}.\\

\begin{tabular}{ccc}
Before & Semantic patch & After \\
\begin{minipage}[t]{.3\linewidth}
\begin{lstlisting}
#DEFINE TEST "foo"

printf("foo");

int main(int i) {
//Test
  int k = foo();

  if(1) {
    foo();
  } else {
    foo();
  }

  foo();
}
\end{lstlisting}
\end{minipage}
&
\begin{minipage}[t]{.3\linewidth}
\begin{lstlisting}[language=Cocci]
@M@@

@@@M


@-- foo()
@++ bar()
\end{lstlisting}
\end{minipage}
&
\begin{minipage}[t]{.3\linewidth}
\begin{lstlisting}
#DEFINE TEST "foo"

printf("foo");

int main(int i) {
//Test
  int k = bar();

  if(1) {
    bar();
  } else {
    bar();
  }

  bar();
}
\end{lstlisting}
\end{minipage}\\
\end{tabular}

\newpage
\subsection{Removing a function argument}

Another important kind of evolution is the introduction or deletion of a
function argument. In the following example, the rule \texttt{rule1} looks
for definitions of functions having return type \texttt{irqreturn\_t} and
two parameters. A second \emph{anonymous} rule then looks for calls to the
previously matched functions that have three arguments. The third argument
is then removed to correspond to the new function prototype.\\

\begin{tabular}{c}
\begin{lstlisting}[language=Cocci,name=arg]
@M@ rule1 @
identifier fn;
identifier irq, dev_id;
typedef irqreturn_t;
@@@M

static irqreturn_t fn (int irq, void *dev_id)
{
   ...
}

@M@@
identifier rule1.fn;
expression E1, E2, E3;
@@@M

 fn(E1, E2
@--  ,E3
   )
\end{lstlisting}\\
\end{tabular}

\vspace{1cm}

\begin{tabular}{c}
  \texttt{drivers/atm/firestream.c} at line 1653 before transformation\\
\begin{lstlisting}[language=PatchC]
static void fs_poll (unsigned long data)
{
        struct fs_dev *dev = (struct fs_dev *) data;

@-        fs_irq (0, dev, NULL);
        dev->timer.expires = jiffies + FS_POLL_FREQ;
        add_timer (&dev->timer);
}
\end{lstlisting}\\
\vspace{1cm}
\\


  \texttt{drivers/atm/firestream.c} at line 1653 after transformation\\
\begin{lstlisting}[language=PatchC]
static void fs_poll (unsigned long data)
{
        struct fs_dev *dev = (struct fs_dev *) data;

@+        fs_irq (0, dev);
        dev->timer.expires = jiffies + FS_POLL_FREQ;
        add_timer (&dev->timer);
}
\end{lstlisting}\\
\end{tabular}

\newpage
\subsection{Introduction of a macro}

To avoid code duplication or error prone code, the kernel provides
macros such as \texttt{BUG\_ON}, \texttt{DIV\_ROUND\_UP} and
\texttt{FIELD\_SIZE}. In these cases, the semantic patches look for
the old code pattern and replace it by the new code.\\

A semantic patch to introduce uses of the \texttt{DIV\_ROUND\_UP} macro
looks for the corresponding expression, \emph{i.e.}, $(n + d - 1) /
d$. When some code is matched, the metavariables \texttt{n} and \texttt{d}
are bound to their corresponding expressions. Finally, Coccinelle rewrites
the code with the \texttt{DIV\_ROUND\_UP} macro using the values bound to
\texttt{n} and \texttt{d}, as illustrated in the patch that follows.\\

\begin{tabular}{c}
Semantic patch to introduce uses of the \texttt{DIV\_ROUND\_UP} macro\\
\begin{lstlisting}[language=Cocci,name=divround]
@M@ haskernel @
@@@M

#include <linux/kernel.h>

@M@ depends on haskernel @
expression n,d;
@@@M

(
@-- (((n) + (d)) - 1) / (d))
@++ DIV_ROUND_UP(n,d)
|
@-- (((n) + ((d) - 1)) / (d))
@++ DIV_ROUND_UP(n,d)
)
\end{lstlisting}
\end{tabular}\\

\vspace{1cm}

\begin{tabular}{c}
Example of a generated patch hunk\\
\begin{lstlisting}[language=PatchC]
@---- a/drivers/atm/horizon.c
@++++ b/drivers/atm/horizon.c
@M@@ -698,7 +698,7 @@ got_it:
                if (bits)
                        *bits = (div<<CLOCK_SELECT_SHIFT) | (pre-1);
                if (actual) {
@--                       *actual = (br + (pre<<div) - 1) / (pre<<div);
@++                       *actual = DIV_ROUND_UP(br, pre<<div);
                        PRINTD (DBG_QOS, "actual rate: %u", *actual);
                }
                return 0;
\end{lstlisting}
\end{tabular}\\

\newpage

The \texttt{BUG\_ON} macro makes a assertion about the value of an
expression. However, because some parts of the kernel define
\texttt{BUG\_ON} to be the empty statement when debugging is not wanted,
care must be taken when the asserted expression may have some side-effects,
as is the case of a function call. Thus, we create a rule introducing
\texttt{BUG\_ON} only in the case when the asserted expression does not
perform a function call.

On particular piece of code that has the form of a function call is a use
of \texttt{unlikely}, which informs the compiler that a particular
expression is unlikely to be true.  In this case, because \texttt{unlikely}
does not perform any side effects, it is safe to use \texttt{BUG\_ON}.  The
second rule takes care of this case.  It furthermore disables the
isomorphism that allows a call to \texttt{unlikely} be replaced with its
argument, as then the second rule would be the same as the first one.\\

\begin{tabular}{c}
\begin{lstlisting}[language=Cocci,name=bugon]
@M@@
expression E,f;
@@@M

(
  if (<+... f(...) ...+>) { BUG(); }
|
@-- if (E) { BUG(); }
@++ BUG_ON(E);
)

@M@ disable unlikely @
expression E,f;
@@@M

(
  if (<+... f(...) ...+>) { BUG(); }
|
@-- if (unlikely(E)) { BUG(); }
@++ BUG_ON(E);
)
\end{lstlisting}\\
\end{tabular}\\

For instance, using the semantic patch above, Coccinelle generates
patches like the following one.

\begin{tabular}{c}
\begin{lstlisting}[language=PatchC]
@---- a/fs/ext3/balloc.c
@++++ b/fs/ext3/balloc.c
@M@@ -232,8 +232,7 @@ restart:
                prev = rsv;
        }
        printk("Window map complete.\n");
@--       if (bad)
@--               BUG();
@++       BUG_ON(bad);
 }
 #define rsv_window_dump(root, verbose) \
        __rsv_window_dump((root), (verbose), __FUNCTION__)
\end{lstlisting}
\end{tabular}

\newpage
\subsection{Look for \texttt{NULL} dereference}

This SmPL match looks for \texttt{NULL} dereferences. Once an
expression has been compared to \texttt{NULL}, a dereference to this
expression is prohibited unless the pointer variable is reassigned.\\

\begin{tabular}{c}
    Original \\

\begin{lstlisting}
foo = kmalloc(1024);
if (!foo) {
  printk ("Error %s", foo->here);
  return;
}
foo->ok = 1;
return;
\end{lstlisting}\\
  \end{tabular}

\vspace{1cm}

\begin{tabular}{c}
  Semantic match\\

\begin{lstlisting}[language=Cocci]
@M@@
expression E, E1;
identifier f;
statement S1,S2,S3;
@@@M

@+* if (E == NULL)
{
  ... when != if (E == NULL) S1 else S2
      when != E = E1
@+* E->f
  ... when any
  return ...;
}
else S3
\end{lstlisting}\\
\end{tabular}

\vspace{1cm}

\begin{tabular}{c}
  Matched lines\\

\begin{lstlisting}[language=PatchC]
foo = kmalloc(1024);
@-if (!foo) {
@-  printk ("Error %s", foo->here);
  return;
}
foo->ok = 1;
return;
\end{lstlisting}\\
\end{tabular}

\newpage
\subsection{Reference counter: the of\_xxx API}

Coccinelle can embed Python code. Python code is used inside special
SmPL rule annotated with \texttt{script:python}.  Python rules inherit
metavariables, such as identifier or token positions, from other SmPL
rules. The inherited metavariables can then be manipulated by Python
code.

The following semantic match looks for a call to the
\texttt{of\_find\_node\_by\_name} function. This call increments a
counter which must be decremented to release the resource. Then, when
there is no call to \texttt{of\_node\_put}, no new assignment to the
\texttt{device\_node} variable \texttt{n} and a \texttt{return}
statement is reached, a bug is detected and the position \texttt{p1}
and \texttt{p2} are initialized. As the Python only depends on the
positions \texttt{p1} and \texttt{p2}, it is evaluated. In the
following case, some emacs Org mode data are produced.  This example
illustrates the various fields that can be accessed in the Python code from
a position variable.

\begin{tabular}{c}
\begin{lstlisting}[language=Cocci,breaklines=true]
@M@ r exists @
local idexpression struct device_node *n;
position p1, p2;
statement S1,S2;
expression E,E1;
@@@M

(
if (!(n@p1 = of_find_node_by_name(...))) S1
|
n@p1 = of_find_node_by_name(...)
)
<... when != of_node_put(n)
    when != if (...) { <+... of_node_put(n) ...+> }
    when != true !n  || ...
    when != n = E
    when != E = n
if (!n || ...) S2
...>
(
  return <+...n...+>;
|
return@p2 ...;
|
n = E1
|
E1 = n
)

@M@ script:python @
p1 << r.p1;
p2 << r.p2;
@@@M

print "* TODO [[view:%s::face=ovl-face1::linb=%s::colb=%s::cole=%s][inc. counter:%s::%s]]" % (p1[0].file,p1[0].line,p1[0].column,p1[0].column_end,p1[0].file,p1[0].line)
print "[[view:%s::face=ovl-face2::linb=%s::colb=%s::cole=%s][return]]" % (p2[0].file,p2[0].line,p2[0].column,p2[0].column_end)
\end{lstlisting}
\end{tabular}


\newpage

Lines 13 to 17 list a variety of constructs that should not appear
between a call to \texttt{of\_find\_node\_by\_name} and a buggy return
site. Examples are a call to \texttt{of\_node\_put} (line 13) and a
transition into the then branch of a conditional testing whether
\texttt{n} is \texttt{NULL} (line 15). Any number of conditionals
testing whether \texttt{n} is \texttt{NULL} are allowed as indicated
by the use of a nest \texttt{<...~~...>} to describe the path between
the call to \texttt{of\_find\_node\_by\_name}, the return and the
conditional in the pattern on line 18.\\

The previously semantic match has been used to generate the following
lines. They may be edited using the emacs Org mode to navigate in the code
from a site to another.

\begin{lstlisting}[language=,breaklines=true]
* TODO [[view:/linux-next/arch/powerpc/platforms/pseries/setup.c::face=ovl-face1::linb=236::colb=18::cole=20][inc. counter:/linux-next/arch/powerpc/platforms/pseries/setup.c::236]]
[[view:/linux-next/arch/powerpc/platforms/pseries/setup.c::face=ovl-face2::linb=250::colb=3::cole=9][return]]
* TODO [[view:/linux-next/arch/powerpc/platforms/pseries/setup.c::face=ovl-face1::linb=236::colb=18::cole=20][inc. counter:/linux-next/arch/powerpc/platforms/pseries/setup.c::236]]
[[view:/linux-next/arch/powerpc/platforms/pseries/setup.c::face=ovl-face2::linb=245::colb=3::cole=9][return]]
\end{lstlisting}

Note~: Coccinelle provides some predefined Python functions,
\emph{i.e.}, \texttt{cocci.print\_main}, \texttt{cocci.print\_sec} and
\texttt{cocci.print\_secs}. One could alternatively write the following
SmPL rule instead of the previously presented one.

\begin{tabular}{c}
\begin{lstlisting}[language=Cocci]
@M@ script:python @
p1 << r.p1;
p2 << r.p2;
@@@M

cocci.print_main(p1)
cocci.print_sec("return",p2)
\end{lstlisting}
\end{tabular}\\

The function \texttt{cocci.print\_secs} is used when there are several
positions which are matched by a single position variable and that
every matched position should be printed.

Any metavariable could be inherited in the Python code. However,
accessible fields are not currently equally supported among them.

% \begin{tabular}{ccc}
% Before & Semantic patch & After \\
% \begin{minipage}[t]{.3\linewidth}
% \begin{lstlisting}
% \end{lstlisting}
% \end{minipage}
% &
% \begin{minipage}[t]{.3\linewidth}
% \begin{lstlisting}[language=Cocci]
% \end{lstlisting}
% \end{minipage}
% &
% \begin{minipage}[t]{.3\linewidth}
% \begin{lstlisting}
% \end{lstlisting}
% \end{minipage}\\
% \end{tabular}

%%% Local Variables:
%%% mode: LaTeX
%%% TeX-master: "cocci_syntax"
%%% coding: latin-9
%%% TeX-PDF-mode: t
%%% ispell-local-dictionary: "american"
%%% End:

\end{document}

%%% Local Variables:
%%% mode: LaTeX
%%% TeX-master: "cocci_syntax"
%%% coding: latin-9
%%% TeX-PDF-mode: t
%%% ispell-local-dictionary: "english"
%%% End:


\chapter{\spatch command line options}

%coupling: ../spatch.1
\section{Introduction}

This document describes the options provided by Coccinelle.  The options
have an impact on various phases of the semantic patch application
process.  These are:

\begin{enumerate}
\item Selecting and parsing the semantic patch.
\item Selecting and parsing the C code.
\item Application of the semantic patch to the C code.
\item Transformation.
\item Generation of the result.
\end{enumerate}

\noindent
One can either initiate the complete process from step 1, or
to perform step 1 or step 2 individually.

Coccinelle has quite a lot of options.  The most common usages are as
follows, for a semantic match {\tt foo.cocci}, a C file {\tt foo.c}, and a
directory {\tt foodir}:

\begin{itemize}
\item {\tt spatch --parse-cocci foo.cocci}: Check that the semantic patch
  is syntactically correct.
\item {\tt spatch --parse-c foo.c}: Check that the C file
  is syntactically correct.  The Coccinelle C parser tries to recover
  during the parsing process, so if one function does not parse, it will
  start up again with the next one.  Thus, a parse error is often not a
  cause for concern, unless it occurs in a function that is relevant to the
  semantic patch.
\item {\tt spatch --sp-file foo.cocci foo.c}: Apply the semantic patch {\tt
    foo.cocci} to the file {\tt foo.c} and print out any transformations as
  the changes between the original and transformed code, using the program
  {\tt diff}.  {\tt --sp-file} is optional in this and the following cases.
\item {\tt spatch --sp-file foo.cocci foo.c --debug}:  The same as the
  previous case, but print out some information about the matching process.
\item {\tt spatch --sp-file foo.cocci --dir foodir}:  Apply the semantic
  patch {\tt foo.cocci} to all of the C files in the directory {\tt foodir}.
\item {\tt spatch --sp-file foo.cocci --dir foodir --include-headers}:  Apply
  the semantic patch {\tt foo.cocci} to all of the C files and header files
  in the directory {\tt foodir}. 
\end{itemize}

In the rest of this document, the options are annotated as follows:
\begin{itemize}
\item \FilledBigDiamondshape: a basic option, that is most likely of
  interest to all users.
\item \BigLowerDiamond: an option that is frequently used, often for better
understanding the effect of a semantic patch.
\item \BigDiamondshape: an option that is likely to be rarely used, but
  whose effect is still comprehensible to a user.
\item An option with no annotation is likely of interest only to
  developers.
\end{itemize}

\section{Selecting and parsing the semantic patch}

\subsection{Standalone options}

\normal{--parse-cocci $\langle$file$\rangle$}{ Parse a semantic
patch file and print out some information about it.}

\subsection{The semantic patch}

\minimum{--sp-file $\langle$file$\rangle$, -c $\langle$file$\rangle$, 
-cocci-file $\langle$file$\rangle$}{ Specify the name of the file
  containing the semantic patch.  The file name should end in {\tt .cocci}.
All three options do the same thing.  These options are optional.  If they
are not used, the single file whose name ends in \texttt{.cocci} is
assoumed to be the name of the file containing the semantic patch.}

\rare{--sp ``semantic patch string''}{Specify a semantic match as a
  command-line argument.  See the section ``Command-line semantic match''
  in the manual.}

\subsection{Isomorphisms}

\rare{--iso, --iso-file}{ Specify a file containing isomorphisms to be used in
place of the standard one.  Normally one should use the {\tt using}
construct within a semantic patch to specify isomorphisms to be used {\em
  in addition to} the standard ones.}

\rare{--iso-limit $\langle$int$\rangle$} Limit the depth of application of
isomorphisms to the specified integer.

\rare{--no-iso-limit} Put no limit on the number of times that
isomorphisms can be applied. This is the default.

\rare{--disable-iso}{Disable a specific isomorphism from the command line.
  This option can be specified multiple times.}

\developer{--track-iso}{ Gather information about isomorphism usage.}

\developer{--profile-iso}{ Gather information about the time required for
isomorphism expansion.}

\subsection{Display options}

\rare{--show-cocci}{Show the semantic patch that is being processed before
  expanding isomorphisms.}

\rare{--show-SP}{Show the semantic patch that is being processed after
  expanding isomorphisms.}

\rare{--show-ctl-text}{ Show the representation
of the semantic patch in CTL.}

\rare{--ctl-inline-let}{ Sometimes {\tt let} is used to name
intermediate terms CTL representation.  This option causes the let-bound
terms to be inlined at the point of their reference.
This option implicitly sets {\bf --show-ctl-text}.}

\rare{--ctl-show-mcodekind}{ Show
transformation information within the CTL representation
of the semantic patch. This option implicitly sets {\bf --show-ctl-text}.}

\rare{--show-ctl-tex}{ Create a LaTeX files showing the representation
of the semantic patch in CTL.}

\section{Selecting and parsing the C files}

\subsection{Standalone options}

\normal{--parse-c $\langle$file/dir$\rangle$}{ Parse a {\tt .c} file or all
  of the {\tt .c} files in a directory.  This generates information about
  any parse errors encountered.}

\normal{--parse-h $\langle$file/dir$\rangle$}{ Parse a {\tt .h} file or all
  of the {\tt .h} files in a directory.  This generates information about
  any parse errors encountered.}

\normal{--parse-ch $\langle$file/dir$\rangle$}{ Parse a {\tt .c} or {\tt
    .h} file or all of the {\tt .c} or {\tt .h} files in a directory.  This
  generates information about any parse errors encountered.}

\normal{--control-flow $\langle$file$\rangle$, --control-flow
$\langle$file$\rangle$:$\langle$function$\rangle$}{ Print a control-flow
graph for all of the functions in a file or for a specific function in a
file.  This requires {\tt dot} (http://www.graphviz.org/) and {\tt gv}.}

\rare{--control-flow-to-file $\langle$file$\rangle$,
  --control-flow-to-file
  $\langle$file$\rangle$:$\langle$function$\rangle$}{ Like --control-flow
  but just puts the dot output in a file in the {\em current} directory.
  For PATH/file.c, this produces file:xxx.dot for each (selected) function
  xxx in PATH/file.c.}

\rare{--type-c $\langle$file$\rangle$}{ Parse a C file and pretty-print a
version including type information.}

\developer{--tokens-c $\langle$file$\rangle$}{Prints the tokens in a C
  file.}

\developer{--parse-unparse $\langle$file$\rangle$}{Parse and then reconstruct
  a C file.}

\developer{--compare-c $\langle$file$\rangle$ $\langle$file$\rangle$,
  --compare-c-hardcoded}{Compares one C file to another, or compare the
file tests/compare1.c to the file tests/compare2.c.}

\developer{--test-cfg-ifdef $\langle$file$\rangle$}{Do some special
processing of \#ifdef and display the resulting control-flow graph.  This
requires {\tt dot} and {\tt gv}.}

\developer{--test-attributes $\langle$file$\rangle$,
           --test-cpp $\langle$file$\rangle$}{
Test the parsing of cpp code and attributes, respectively.}

\subsection{Selecting C files}

An argument that ends in {\tt .c} is assumed to be a C file to process.
Normally, only one C file or one directory is specified.  If multiple C
files are specified, they are treated in parallel, {\em i.e.}, the first
semantic patch rule is applied to all functions in all files, then the
second semantic patch rule is applied to all functions in all files, etc.
If a directory is specified then no files may be specified and only the
rightmost directory specified is used.

\normal{--include-headers}{ This option causes header files to be processed
independently.  This option only makes sense if a directory is specified
using {\bf --dir}.}

\normal{--use-glimpse}{ Use a glimpse index to select the files to which
a semantic patch may be relevant.  This option requires that a directory is
specified.  The index may be created using the script {\tt
  coccinelle/scripts/ glimpseindex-cocci.sh}.  Glimpse is available at
http://webglimpse.net/.  In conjunction with the option {\bf --patch-cocci}
this option prints the regular expression that will be passed to glimpse.}

\normal{--use-idutils $[\langle$file$\rangle]$}
{ Use an id-utils index created using lid to select
  the files to which a semantic patch may be relevant.  This option
  requires that a directory is specified.  The index may be created using
  the script {\tt coccinelle/scripts/ idindex-cocci.sh}.  In conjunction
  with the option {\bf --patch-cocci} this option prints the regular
  expression that will be passed to glimpse.

The optional file name option is the name of the file in which to find the
index.  It has been reported that the viewer seascope can be used to
generate an appropriate index.  If no file name is specified, the default
is .id-utils.index.
}

\normal{--use-coccigrep}{ Use a version of grep implemented in Coccinelle
  to check that selected files are relevant to the semantic patch.  This
  option is only relevant to the case of working on a complete directory,
  when parallelism is requested (max and index options).  Otherwise it is
  the default, except when multiple files are requested to be treated as a
  single unit.  In that case grep is used.

  Note that coccigrep or grep is used even if
  glimpse or id-utils is selected, to account for imprecision in the index
  (glimpse at least does not distinguish between underline and space,
  leading to false positives).}

\rare{--selected-only}{Just show what files will be selected for processing.}

\normal{--dir}{ Specify a directory containing C files to process.  A trailing
  {\tt /} is permitted on the directory name and has no impact on the
  result.  By default, the include path will be set to the ``include''
  subdirectory of this directory.  A different include path can be
  specified using the option {\bf -I}.  {\bf --dir} only considers the
  rightmost directory in the argument list.  This behavior is convenient
  for creating a script that always works on a single directory, but allows
  the user of the script to override the provided directory with another
  one.  Spatch collects the files in the directory using {\tt find} and
  does not follow symbolic links.}

\developer{--kbuild-info $\langle$file$\rangle$}{ The specified file
  contains information about which sets of files should be considered in
  parallel.}

\developer{--disable-worth-trying-opt}{Normally, a C file is only
  processed if it contains some keywords that have been determined to be
  essential for the semantic patch to match somewhere in the file.  This
  option disables this optimization and tries the semantic patch on all files.}

\developer{--test $\langle$file$\rangle$}{ A shortcut for running Coccinelle
on the semantic patch ``file{\tt{.cocci}}'' and the C file
``file{\tt{.c}}''.  The result is put in the file {\tt
  /tmp/file{\tt{.res}}}.  If writing a file in /tmp with a non-fresh name
is a concern, then do not use this option.
}

\developer{--testall}{A shortcut for running Coccinelle on all files in a
  subdirectory {\tt tests} such that there are all of a {\tt .cocci} file, a {\tt
    .c} file, and a {\tt .res} file, where the {\tt .res} contains the
  expected result.}

\developer{--test-okfailed, --test-regression-okfailed} Other options for
keeping track of tests that have succeeded and failed.

\developer{--compare-with-expected}{Compare the result of applying
  Coccinelle to file{\tt{.c}} to the file file{\tt{.res}} representing the
  expected result.}

\developer{--expected-score-file $\langle$file$\rangle$}{
which score file to compare with in the testall run}

\subsection{Parsing C files}

\rare{--show-c}{Show the C code that is being processed.}

\rare{--parse-error-msg}{Show parsing errors in the C file.}

\rare{--verbose-parsing}{Show parsing errors in the C file, as well as
  information about attempts to accomodate such errors.  This implicitly
  sets --parse-error-msg.}

\rare{--type-error-msg}{Show information about where the C type checker
  was not able to determine the type of an expression.}

\rare{--int-bits $\langle$n$\rangle$, --long-bits
$\langle$n$\rangle$}{Provide integer size information. n is the number of
bits in an unsigned integer or unsigned long, respectively.  If only the
option {\bf --int-bits} is used, unsigned longs will be assumed to have
twice as many bits as unsigned integers.  If only the option {\bf
-long-bits} is used, unsigned ints will be assumed to have half as many
bits as unsigned integers.  This information is only used in determining
the types of integer constants, according to the ANSI C standard (C89).  If
neither is provided, the type of an integer constant is determined by the
sequence of ``u'' and ``l'' annotations following the constant.  If there
is none, the constant is assumed to be a signed integer.  If there is only
``u'', the constant is assumed to be an unsigned integer, etc.}

\rare{--no-loops}{Drop back edges for loops.  This may make a semantic
  patch/match run faster, at the cost of not finding matches that wrap
  around loops.}

\developer{--use-cache}{Use preparsed versions of the C files that are
stored in a cache.}

\developer{--cache-prefix}{Specify the directory in which to store 
preparsed versions of the C files.  This sets {--use-cache}}

\developer{--cache-limit}{Specify the maximum number of
preparsed C files to store.  The cache is cleared of all files with names
ending in .ast-raw and .depend-raw on reaching this limit.  Only
effective if --cache-prefix is used as well.  This is most useful when
iteration is used to process a file multiple times within a single run of
Coccinelle.}

\developer{--debug-cpp, --debug-lexer, --debug-etdt,
  --debug-typedef}{Various options for debugging the C parser.}

\developer{--filter-msg, --filter-define-error,
  --filter-passed-level}{Various options for debugging the C parser.}

\developer{--only-return-is-error-exit}{In matching ``{\tt{\ldots}}'' in
  a semantic patch or when forall is specified, a rule must match all
  control-flow paths starting from a node matching the beginning of the
  rule.  This is relaxed, however, for error handling code.  Normally, error
  handling code is considered to be a conditional with only a then branch
  that ends in goto, break, continue, or return.  If this option is set,
  then only a then branch ending in a return is considered to be error
  handling code.  Usually a better strategy is to use {\tt when strict} in
  the semantic patch, and then match explicitly the case where there is a
  conditional whose then branch ends in a return.}

\subsubsection*{Macros and other preprocessor code}

\normal{--macro-file $\langle$file$\rangle$}{
  Extra macro definitions to be taken into account when parsing the C
  files.  This uses the provided macro definitions in addition to those in
  the default macro file.}

\normal{--macro-file-builtins $\langle$file$\rangle$}{
  Builtin macro definitions to be taken into account when parsing the C
  files.  This causes the macro definitions provided in the default macro
  file to be ignored and the ones in the specified file to be used instead.}

\rare{--ifdef-to-if,-no-ifdef-to-if}{
The option {\bf --ifdef-to-if}
represents an {\tt \#ifdef} in the source code as a conditional in the
control-flow graph when doing so represents valid code.  {\bf
-no-ifdef-to-if} disables this feature.  {\bf --ifdef-to-if} is the
default.
}

\rare{--noif0-passing}{ Normally code under \#if 0 is ignored.  If this
option is set then the code is considered, just like the code under any
other \#ifdef.}

\rare{--defined $s$}{The string $s$ is a comma-separated list of constants
  that should be considered to be defined, with respect to uses of {\tt
    \#ifdef} and {\tt \#ifndef} in C code.  No spaces should appear in $s$.
  Multiple {\bf --defined} arguments can be provided and the list of
  strings accumulates.  For the provided strings any {\tt \#else}s of {\tt
    \#ifdef}s are ignored and any {\tt \#ifndef}s are ignored, unless the
  argument {\bf --noif0-passing} is also given, in which case {\bf
    --defined} has no effect.  Note that occurrences of {\tt \#define} in
  the C code have no effect on the list of defined constants.}

\rare{--undefined $s$}{Analogous to {\bf --defined} except that the strings
  represent constants that should be considered to be undefined.}

\developer{--noadd-typedef-root}{This seems to reduce the scope of a
  typedef declaration found in the C code.}

\subsubsection*{Include files}

\normal{--recursive-includes, --all-includes, --local-includes,
  --no-includes}{ These options control which include files mentioned in a
  C file are taken into account.  {\bf --recursive-includes} indicates
  that all included files mentioned in the .c file(s) or any included files
  will be processed.  {\bf --all-includes} indicates that all included
  files mentioned in the .c file(s) will be processed.  {\bf
    --local-includes} indicates that only included files in the current
  directory will be processed.  {\bf --no-includes} indicates that no
  included files will be processed.  If the semantic patch contains type
  specifications on expression metavariables, then the default is {\bf
    --local-includes}.  Otherwise the default is {\bf --no-includes}.  At
  most one of these options can be specified.}

\normal{-I $\langle$path$\rangle$}{ This option specifies a directory
  in which to find non-local include files.  This option can be used
  several times to specify multipls include paths.}

\rare{--include $\langle$file$\rangle$}{ This option give the name of a
  file to consider as being included in each processed file.  The file is
  added to the end of the file's list of included files.  The complete path
  name should be given; the {\bf -I} options are not taken into account to
  find the file.  This option can be used
  several times to include multiple files.}

\rare{--relax-include-path}{This option when combined with --all-includes
  causes the search for local
  include files to consider the current directory, even if the include
  patch specifies a subdirectory.  This is really only useful for testing,
  eg with the option {\bf --testall}}

\rare{--c++}{Make an extremely minimal effort to parse C++ code.  Currently,
  this is limited to allowing identifiers to contain ``::'', tilde, and
  template invocations.  Consider testing your code first with spatch
  --type-c to see if there are any type annotations in the code you are
  interested in processing.  If not, then it was probably not parsed.}

\rare{--ibm}{Make a effort to parse IBM C code.  Currently decimal
  declarations are supported.}

\section{Application of the semantic patch to the C code}

\subsection{Feedback at the rule level during the application of the
  semantic patch}

\normal{--show-bindings}{
Show the environments with respect to which each rule is applied and the
bindings that result from each such application.}

\normal{--show-dependencies}{ Show the status (matched or unmatched) of the
rules on which a given rule depends.  {\bf --show-dependencies} implicitly
sets {\bf --show-bindings}, as the values of the dependencies are
environment-specific.}

\normal{--show-trying}{
Show the name of each program element to which each rule is applied.}

\normal{--show-transinfo}{
Show information about each transformation that is performed.
The node numbers that are referenced are the number of the nodes in the
control-flow graph, which can be seen using the option {\bf --control-flow}
(the initial control-flow graph only) or the option {\bf --show-flow} (the
control-flow graph before and after each rule application).}

\normal{--show-misc}{Show some miscellaneous information.}

\rare{--show-flow $\langle$file$\rangle$, --show-flow
  $\langle$file$\rangle$:$\langle$function$\rangle$} Show the control-flow
graph before and after the application of each rule.

\developer{--show-before-fixed-flow}{This is similar to {\bf
    --show-flow}, but shows a preliminary version of the control-flow graph.}

\subsection{Feedback at the CTL level during the application of the
  semantic patch}

\normal{--verbose-engine}{Show a trace of the matching of atomic terms to C
  code.}

\rare{--verbose-ctl-engine}{Show a trace of the CTL matching process.
  This is unfortunately rather voluminous and not so helpful for someone
  who is not familiar with CTL in general and the translation of SmPL into
  CTL specifically.  This option implicitly sets the option {\bf
    --show-ctl-text}.}

\rare{--graphical-trace}{Create a pdf file containing the control flow
  graph annotated with the various nodes matched during the CTL matching
  process.  Unfortunately, except for the most simple examples, the output
  is voluminous, and so the option is not really practical for most
  examples.  This requires {\tt dot} (http://www.graphviz.org/) and {\tt
  pdftk}.}

\rare{--gt-without-label}{The same as {\bf --graphical-trace}, but the PDF
  file does not contain the CTL code.}

\rare{--partial-match}{
Report partial matches of the semantic patch on the C file.  This can
  be substantially slower than normal matching.}

\rare{--verbose-match}{
Report on when CTL matching is not applied to a function or other program
unit because it does not contain some required atomic pattern.
This can be viewed as a simpler, more efficient, but less informative
version of {\bf --partial-match}.}

\subsection{Actions during the application of the semantic patch}

\normal{-D rulename}{Run the patch considering that the virtual rule
  ``rulename'' is satisfied.  Virtual rules should be declared at the
  beginning of the semantic patch in a comma separated list following the
  keyword virtual.  Other rules can depend on the satisfaction or non
  satifaction of these rules using the keyword {\tt depends on} in the
  usual way.}

\normal{-D variable=value}{Run the patch considering that the virtual
  identifier metavariable ``variable'' is bound to ``value''.  Any
  identifier metavariable can be designated as being virtual by giving it
  the rule name {\tt virtual}.  An example is in demos/vm.coci}

\rare{--allow-inconsistent-paths}{Normally, a term that is transformed
  should only be accessible from other terms that are matched by the
  semantic patch.  This option removes this constraint.  Doing so, is
  unsafe, however, because the properties that hold along the matched path
  might not hold at all along the unmatched path.}

\rare{--disallow-nested-exps}{In an expression that contains repeated
  nested subterms, {\em e.g.} of the form {\tt f(f(x))}, a pattern can
  match a single expression in multiple ways, some nested inside others.
  This option causes the matching process to stop immediately at the
  outermost match.  Thus, in the example {\tt f(f(x))}, the possibility
  that the pattern {\tt f(E)}, with metavariable {\tt E}, matches with {\tt
    E} as {\tt x} will not be considered.}

\rare{--no-safe-expressions}{normally, we check that an expression does
       not match something earlier in the disjunction.  But for large
       disjunctions, this can result in a very big CTL formula.  So this
       option give the user the option to say he doesn't want this feature,
       if that is the case.}

\rare{--pyoutput coccilib.output.Gtk, --pyoutput coccilib.output.Console}{
This controls whether Python output is sent to Gtk or to the console.  {\bf
  --pyoutput coccilib.output.Console} is the default.  The Gtk option is
currently not well supported.}

\developer{--loop}{When there is ``{\tt{\ldots}}'' in the semantic patch,
  the CTL operator {\sf AU} is used if the current function does not
  contain a loop, and {\sf AW} may be used if it does.  This option causes
  {\sf AW} always to be used.}

\rare{--ocaml-regexps}{Use the regular expressions provided by the OCaml
  \texttt{Str} library.  This is the default if the PCRE library is not
  available.  Otherwise PCRE regular expressions are used by default.}

\developer{--steps $\langle$int$\rangle$}{
This limits the number of steps performed by the CTL engine to the
specified number.  This option is unsafe as it might cause a rule to fail
due to running out of steps rather than due to not matching.}

\developer{--bench $\langle$int$\rangle$}{This collects various information
  about the operations performed during the CTL matching process.}

% \developer{--popl, --popl-mark-all, --popl-keep-all-wits}{
% These options use a simplified version of the SmPL language.  {\bf
%   --popl-mark-all} and {\bf --popl-keep-all-wits} implicitly set {\bf
%   --popl}.}

\rare{--reverse}{Inverts the semantic patch before applying it.
A potential use case is backporting changes to previous versions. If a
semantic patch represents an API change, then the reverse undoes the
API change. Note that inverting a semantic patch is not always possible.
In particular, the composition of a
semantic patch with its inverse is not guaranteed to be an empty patch.}

\section{Generation of the result}

Normally, the only output is the differences between the original code and
the transformed code obtained using the program {\tt diff} with the unified
format option.  If
stars are used in column 0 rather than {\tt -} and {\tt +}, then the {\tt
  -} lines in the output are the lines that matched the stars.

\normal{--keep-comments}{Don't remove comments adjacent to removed code.}

\normal{--linux-spacing, --smpl-spacing}{Control the spacing within the code
  added by the semantic patch.  The option {\bf --linux-spacing} causes
  spatch to follow the conventions of Linux, regardless of the spacing in
  the semantic patch.  This is the default.  The option {\bf
  --smpl-spacing} causes spatch to follow the spacing given in the semantic
  patch, within individual lines.}

\rare{-o $\langle$file$\rangle$}{ This causes the transformed code to be
  placed in the file {\tt file}.  The difference between the original code
  and the transformed code is still printed to the standard output using
  {\tt diff} with the unified format option.  This option only makes sense
  when {\tt -} and {\tt +} are used.}

\rare{--in-place}{ Modify the input file to contain the transformed code.
  The difference between the original code and the transformed code is
  still printed to the standard output using {\tt diff} with the unified
  format option.  By default, the input file is overwritten when using this
  option, with no backup.  This option only makes sense when {\tt -} and
  {\tt +} are used.}

\rare{--backup-suffix $s$}{The suffix $s$ of the file to use in making a
  backup of the original file(s).  This suffix should include the leading
  ``.'', if one is desired.  This option only has an effect when the option
  {\tt --in-place} is also used.}

\rare{--out-place}{ Store the result of modifying the code in a .cocci-res
  file.  The difference between the original code
  and the transformed code is still printed to the standard output using
  {\tt diff} with the unified format option.  This option only
  makes sense when {\tt -} and {\tt +} are used.}

\rare{--no-show-diff}{ Normally, the difference between the original and
  transformed code is printed on the standard output.  This option causes
  this not to be done.}

\rare{-U}{ Set number of context lines to be provided by {\tt diff}.}

\rare{--patch $\langle$path$\rangle$}{The prefix of the pathname of the
  directory or file name that should dropped from the {\tt diff} line in the
  generated patch.  This is useful if you want to apply a patch only to a
  subdirectory of a source code tree but want to create a patch that can be
  applied at the root of the source code tree.  An example could be {\tt
  spatch --sp-file foo.cocci --dir /var/linuxes/linux-next/drivers --patch
  /var/linuxes/linux-next}.  A trailing {\tt /} is permitted on the
  directory name and has no impact on the result.}

\rare{--save-tmp-files}{Coccinelle creates some temporary
  files in {\tt /tmp} that it deletes after use.  This option causes these
  files to be saved.}

\developer{--debug-unparsing}{Show some debugging information about the
  generation of the transformed code.  This has the side-effect of
  deleting the transformed code.}


\section{Other options}

\subsection{Version information}

\normal{--version}{ The version of Coccinelle is printed on the standard
  output.  No other options are allowed.}

\normal{--date}{ The date of the current version of Coccinelle are printed
  on the standard output. No other options are allowed.}

\subsection{Help}

\minimum{--h, --shorthelp}{ The most useful commands.}

\minimum{--help, --help, --longhelp}{ A complete listing of the available
commands.}

\subsection{Controlling the execution of Coccinelle}

\normal{--timeout $\langle$int$\rangle$}{ The maximum time in seconds for
  processing a single file.}

\rare{--max $\langle$int$\rangle$}{This option informs Coccinelle of the
  number of instances of Coccinelle that will be run concurrently.  This
  option requires {\bf --index}.  It is usually used with {\bf --dir}.}

\rare{--index $\langle$int$\rangle$}{This option informs Coccinelle of
  which of the concurrent instances is the current one.  This option
  requires {\bf --max}.}

\rare{--mod-distrib}{When multiple instances of Coccinelle are run in
  parallel, normally the first instance processes the first $n$ files, the
  second instance the second $n$ files, etc.  With this option, the files
  are distributed among the instances in a round-robin fashion.}

\developer{--debugger}{Option for running Coccinelle from within the OCaml
  debugger.}

\developer{--profile}{ Gather timing information about the main Coccinelle
functions.}

\developer{--disable-once}{Print various warning messages every time some
condition occurs, rather than only once.}

\subsection{External analyses}

\developer{--external-analysis-file}{Loads in the contents of a database
produced by some external analysis tool. Each entry contains the analysis
result of a particular source location. Currently, such a database is a
.csv file providing integer bounds or an integer set for some subset of
the source locations that references an integer memory location.
This database can be inspected with coccilib functions, e.g. to
control the pattern match process.}

\subsection{Miscellaneous}

\rare{--quiet}{Suppress most output.  This is the default.}

%\developer{--pad, -hrule $\langle$dir$\rangle$, -xxx, -l1}{}
\developer{--pad, --xxx, --l1}{}


%%% Local Variables:
%%% mode: LaTeX
%%% TeX-master: "main_options"
%%% coding: utf-8
%%% TeX-PDF-mode: t
%%% ispell-local-dictionary: "american"
%%% End:


%******************************************************************************
% Appendix
%******************************************************************************

%index

{\small
\bibliographystyle{acm}
\bibliography{main}
}


%******************************************************************************
% Postlude
%******************************************************************************

\end{document}
